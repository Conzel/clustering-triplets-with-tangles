%%%%%%%%%%%%%%%%%%%%%%%%%%%%%%%%%%%%%%%%%%%%%%%%%%%%%%%%%%%%%%%%%%%%%%%%%%%%%
%%% LaTeX-Rahmen fuer das Erstellen von Masterarbeiten
%%%%%%%%%%%%%%%%%%%%%%%%%%%%%%%%%%%%%%%%%%%%%%%%%%%%%%%%%%%%%%%%%%%%%%%%%%%%%
%%%%%%%%%%%%%%%%%%%%%%%%%%%%%%%%%%%%%%%%%%%%%%%%%%%%%%%%%%%%%%%%%%%%%%%%%%%%%
%%% allgemeine Einstellungen
%%%%%%%%%%%%%%%%%%%%%%%%%%%%%%%%%%%%%%%%%%%%%%%%%%%%%%%%%%%%%%%%%%%%%%%%%%%%%


\documentclass[twoside,12pt,a4paper]{report}
\usepackage{import}

%\usepackage{reportpage}
\usepackage{epsf}
\usepackage{graphics, graphicx}
\usepackage{latexsym}
\usepackage[margin=10pt,font=small,labelfont=bf]{caption}
\usepackage[utf8]{inputenc}
\usepackage[toc,page]{appendix}
\usepackage{bbm}
\usepackage{amsmath}
\usepackage{amssymb}
\usepackage{placeins}
\usepackage{caption}
\usepackage[subrefformat=parens,labelformat=parens,caption=false]{subfig}
\usepackage{scalefnt}
\usepackage{pgf}
\usepackage{hyperref}
\usepackage{csquotes}
\newcommand{\subfigureautorefname}{\figureautorefname}

\renewcommand{\chapterautorefname}{Chapter}
\renewcommand{\sectionautorefname}{Section}
\renewcommand{\subsectionautorefname}{Subsection}

% continuous equation numbers throughout the chapter
\usepackage{chngcntr}
\counterwithout{equation}{chapter}

% bib settings
\usepackage[round,comma]{natbib}\bibliographystyle{thesis}


\newcommand{\R}{\mathbb{R}}
\newcommand{\C}{\mathbb{C}}
\newcommand{\N}{\mathbb{N}}
\newcommand{\Q}{\mathbb{Q}}
\newcommand{\Z}{\mathbb{Z}}
\newcommand{\E}{\mathbb{E}}

\textwidth 14cm
\textheight 22cm
\topmargin 0.0cm
\evensidemargin 1cm
\oddsidemargin 1cm
%\footskip 2cm
\parskip0.5explus0.1exminus0.1ex

% Kann von Student auch nach pers\"onlichem Geschmack ver\"andert werden.
\pagestyle{headings}

\sloppy


\begin{document}

%%%%%%%%%%%%%%%%%%%%%%%%%%%%%%%%%%%%%%%%%%%%%%%%%%%%%%%%%%%%%%%%%%%%%%%%%%%%
%%% hier steht die neue Titelseite 
%%%%%%%%%%%%%%%%%%%%%%%%%%%%%%%%%%%%%%%%%%%%%%%%%%%%%%%%%%%%%%%%%%%%%%%%%%%%
 
\begin{titlepage}
 \begin{center}
  {\LARGE Eberhard Karls University of T\"ubingen}\\
  {\large Department of Mathematics and Natural Sciences \\
Wilhelm Schickard Institute for Computer Science\\[4cm]}
  {\huge Master Thesis Computer Science\\[2cm]}
  {\Large\bf  Clustering Triplets with Tangles \\[1.5cm]}
 {\large Alexander Conzelmann}\\[0.5cm]
18.10.2022\\[4cm]
{\small\bf Reviewers}\\[0.5cm]
  \parbox{7cm}{\begin{center}{\large Prof. Dr. Ulrike von Luxburg}\\
   Theory of Machine Learning\\
  {\footnotesize Wilhelm Schickard Institute for Computer Science\\
	University of T\"ubingen}\end{center}}\hfill\parbox{7cm}{\begin{center}
  {\large Prof. Felix Wichmann, DPhil}\\
  Neural Information Processing\\
  {\footnotesize Wilhelm Schickard Institute for Computer Science\\
	University of T\"ubingen}\end{center}
 }
  \end{center}
\end{titlepage}

%%%%%%%%%%%%%%%%%%%%%%%%%%%%%%%%%%%%%%%%%%%%%%%%%%%%%%%%%%%%%%%%%%%%%%%%%%%%
%%% Titelr"uckseite: Bibliographische Angaben
%%%%%%%%%%%%%%%%%%%%%%%%%%%%%%%%%%%%%%%%%%%%%%%%%%%%%%%%%%%%%%%%%%%%%%%%%%%%

\thispagestyle{empty}
\vspace*{\fill}
\begin{minipage}{11.2cm}
\textbf{Conzelmann, Alexander}\\
\emph{Clustering Triplets with Tangles}\\ Master Thesis Computer Science\\
Eberhard Karls University of T\"ubingen\\
Thesis period: 04/2022–10/2022
\end{minipage}
\newpage

%%%%%%%%%%%%%%%%%%%%%%%%%%%%%%%%%%%%%%%%%%%%%%%%%%%%%%%%%%%%%%%%%%%%%%%%%%%%

\pagenumbering{roman}
\setcounter{page}{1}

%%%%%%%%%%%%%%%%%%%%%%%%%%%%%%%%%%%%%%%%%%%%%%%%%%%%%%%%%%%%%%%%%%%%%%%%%%%%
%%% Seite I: Zusammenfassug, Danksagung
%%%%%%%%%%%%%%%%%%%%%%%%%%%%%%%%%%%%%%%%%%%%%%%%%%%%%%%%%%%%%%%%%%%%%%%%%%%%


\section*{Abstract}
We investigate a new algorithm for clustering triplets (comparison-based data 
without absolute distance information). This new approach is based on the tangles framework,
a tool previously used to find dense structures in graphs, which 
we extend to work on triplets. Current work on clustering triplets focusses
on embedding the triplets into a euclidean space, resulting possibly in distortions of our data,
and then following up with a classical clustering algorithm. In contrast, our proposed method
does not construct an intermediate embedding, potentially introducing fewer distortions
on our data and achieving a higher clustering accuracy. In addition to being
competitive in performance, our approach provides both explainability as well as a cluster 
hierarchy on top with no added cost. We evaluate the tangles algorithm both on synthetic data 
under a diverse range of settings as well as experimental data from the realm of psychophysics.

\newpage
\section*{Zusammenfassung}
Wir untersuchen einen neuen Algorithmus, um Triplets zu clustern. Triplets sind Daten, die 
Vergleichen zwischen Datenpunkten basieren, jedoch keine absoluten Distanzinformationen beinhalten. Unser neuer Ansatz basiert
auf dem Tangles-Framework, welches in der Vergangenheit verwendet wurde, um dichte Strukturen in Graphen 
zu untersuchen. Wir erweitern in dieser Arbeit den Tangles-Algorithmus, um ihn auch auf Triplets anzuwenden.
Die bisherige Forschung zur Clusterung von Triplet verwendet größtenteils ordinale Einbettungen. In diesen 
werden die Triplet zuerst in einen euklidischen Raum eingebettet, was die Daten verzerren kann. 
Danach wird ein klassischer Clustering-Algorithmus verwendet. Im Gegensatz dazu verwendet unsere Methode keine euklidische Zwischenrepräsentation.
So können wir potentiell Verzerrungen in den Daten umgehen und ein präziseres Clustering generieren. 
Unser Ansatz zeigt nicht nur eine kompetitive Leistung, sondern bietet auch die Möglichkeit, die Ergebnisse zu erklären und 
hierarchisch aufzuschlüsseln. Wir evaluieren den Tangles-Algorithm auf synthetischen Daten in einer Reihe von verschiedenen Szenarien
und außerdem auf experimentellen Daten aus dem Forschungsgebiet der Psychophysik.


\newpage
\section*{Acknowledgements}
I would like to thank Prof. Ulrike von Luxburg for welcoming me so warmly into her group and for providing me room to grow as a researcher. \\
I would also like to extend my deepest gratitude to Solveig Klepper for countless inspiring discussions, 
for always providing valuable insights and for her great attention to detail. \\
Thanks also to David-Elias Künstle for his help whenever I needed to know something about psychophysics.\\
Lastly, I want to thank all my friends who were always there for me, and I especially thank Levin Güver for helping me with all the little details 
of research work and for always providing an interested audience when I needed to explain wild ideas to someone.

\cleardoublepage

%%%%%%%%%%%%%%%%%%%%%%%%%%%%%%%%%%%%%%%%%%%%%%%%%%%%%%%%%%%%%%%%%%%%%%%%%%%%%
%%% Table of Contents
%%%%%%%%%%%%%%%%%%%%%%%%%%%%%%%%%%%%%%%%%%%%%%%%%%%%%%%%%%%%%%%%%%%%%%%%%%%%%

\renewcommand{\baselinestretch}{1.3}
\small\normalsize

\tableofcontents

\renewcommand{\baselinestretch}{1}
\small\normalsize

\cleardoublepage


%%%%%%%%%%%%%%%%%%%%%%%%%%%%%%%%%%%%%%%%%%%%%%%%%%%%%%%%%%%%%%%%%%%%%%%%%%%%%
%%% Der Haupttext, ab hier mit arabischer Numerierung
%%% Mit \input{dateiname} werden die Datei `dateiname' eingebunden
%%%%%%%%%%%%%%%%%%%%%%%%%%%%%%%%%%%%%%%%%%%%%%%%%%%%%%%%%%%%%%%%%%%%%%%%%%%%%

\pagenumbering{arabic}
\setcounter{page}{1}

%% Introduction
\chapter{Introduction}\label{Introduction}
Forschungsfrage, was und wie wird das gemacht, welche grundlegenden Konzepte existieren.

\cleardoublepage


\chapter{Theoretical Background}\label{theory}
Here we write about the theoretical background.

\section{Tangles}\label{theory:tangles}
% What are Tangles? How are they used? What advantages do they
% prove?
% TODO: How to cite correctly here?
Tangles have been a tool in mathematical graph theory, introduced originally by \cite{robertsonGraphMinorsObstructions1991} 
with a diverse range of application %TODO: Citations from Solveig. \\

In recent times, through the work of \cite{klepperClusteringTanglesAlgorithmic2021}, they have been successfully applied
to solve problems of clustering. The mentioned work delivers an algorithmic framework and theoretical guarantees for basic problem settings.
Additionally, it delivers simplified notations, adapted to the domain of computer science. When talking about Tangles,
we will exclusively use the definitions introduced there, not those that might be common in mathematics. \\

In this section, we will now deliver a very brief recap of the basic notions, theory and applications of Tangles in a clustering context.
For more in-depth explanations of the algorithms and exact procedures, refer to \cite{klepperClusteringTanglesAlgorithmic2021}.
\subsection{What is a Tangle?}
The central object in Tangles Theory is a \textbf{bipartition} (which we also refer to as a cut). 
A bipartition is simply a way of dividing a set of elements $V =  \{ v_1, v_2, \ldots \}$ into two distinct subsets $A, B \subset V$, such that
$A \cap B = \O$ and $A \cup B = V$. We can also write a bipartition as $P = \{A, \overline{A}\}$, with $A \subset V$ and $\overline{A}$ being the
complement of $A$ with respect to $V$. \\

For such a bipartition to be useful in clustering, we expect it to hold some degree of information about the cluster 
structure of our data. This means that a good bipartition should not separate groups of data that are tightly coupled.
If we imagine a graph data structure, a good bipartition $P = \{A, \overline{A}\}$ might be a separation of the set of nodes $V$ such that there 
are only a few edges between $A$ and $\overline{A}$. How useful a cut might be for our clustering will be quantified through a \textbf{cost function} 
$c: \mathcal{P}(V) \to \R$, with $\mathcal{P}(V)$ denoting the power set of V. One is free to choose this cost function and it might be dependent on the problem at hand. 
%TODO: Figure of bipartiion
%TODO: insert cost function used?
\\
Assume that for a set of elements $V$ that we are equipped with a set of bipartitions $\mathcal{B} = \{\{A_1, \overline{A_1}\}, \ldots, \{A_n, \overline{A_n}\} \} $ on $V$.
Coupled with the cost function, this set of bipartitions should tell us a lot about the cluster structure of the data:
we know for all bipartitions, how much they do or don't separate dense regions in $V$. The task of the Tangles framework is to aggregate
the information present in the bipartitions and bring it into a useable form. For this, we process $\mathcal{B}$ to a set of so-called \textbf{Tangles}, which
correspond to specific ways of orienting the cuts in a consistent way such that they point to cohesive structures in the data. 
Orienting here means that we pick one specific side of a bipartition. An \textbf{Orientation} of $\mathcal{B}$ is then a set $O = \{o_1, o_2, \ldots o_n\}$, where $o_i$ 
corresponds to either the partition $A_i$ (oriented \textit{left}) or $\overline{A_i}$ (oriented \textit{right}). A consistent orientation (which we also call a Tangle) is an orientation for which:

\begin{align}
    \forall A,B,C \in O: \left| A \cap B \cap B \right| \ge a
.\end{align}
for some fixed parameter $a \in \N$, which we refer to as \textbf{agreement} parameter. This point is also where we need the cost function: Without it, a lot
of reasonably sized sets of bipartitions wouldn't allow for any tangles, as there are simply too many of them to consistently orient. Imagine if our set of bipartitions would 
contain a few random bipartitions: on average, each of these cuts our set of points in half, so we can at most consistently orient on the order of $O(\log(n))$ many of them.
Using the cost function, we can simply restrict our tangles to a set of low-cost (and thus very insightful) cuts $P_{\psi}$, using a threshhold $\psi \in \R$ such that
$P_{\psi} = \{ P c(P) \le \psi \}$. A tangle on $P_{\psi}$ is then said to have order $\psi$.


\begin{figure}[h]
    \centering
    \includegraphics[width=0.8\textwidth]{figures/tangles-example.pdf}
    \caption{An example of how a simple tangle might look like, if we assume a reasonably sized agreement (say $a = 3$). The red lines represent simple cuts, which divide the sets of points into a bipartition.
    A tangle on this set of bipartitions might orient all bipartitions to the left side (indicated by the arrow), so that they point to the dense structure there. Another 
possible tangle might orient all cuts to the left. Notice that a tangle on this set of bipartitions can only either point all bipartitions to the left or to the right, else
the consistency criterion is violated. This might already give a good intuition on why tangles are able to find dense structures in data.}
    \label{fig:tangles-example}
\end{figure}
\FloatBarrier

\subsection{Processing Tangles to a clustering}
As we have seen in \autoref{fig:tangles-example}, a tangle might correspond directly to a cluster. But, a given set of bipartitions usually allows for a wide 
variety of possible tangles, some of them pointing to different or overlapping clusters. We now have to process this set of tangles into a useable clustering.
This step is a bit involved and we aim to only give a rough, intuitional overview here. \\

Given a set of bipartitions $\mathcal{B} = \{b_1, b_2, \ldots b_n\} $, we first want to determine for all possible orders $\psi$ the sets of all tangles of order $\psi$ on $\mathcal{B}$ 
according to a given cost function $c$. Inuitively, the order of the set of tangles determines how coarse the clustering is they define:
If we only use bipartitions with a low cost (so in the case of small $\psi$), then the bipartitions cut only very loosely connected structures. 
If the cost is higher, the bipartitions are allowed to cut through more dense regions. This directly induces a sort of hierarchy, where we go from coarse cluster
structures to fine ones with increasing $\psi$. \\
To better handle this procedure computationally, we build a tree structure on the set of tangles, called the \textbf{Tangle Search Tree}. In the tangles search tree, 
one node represents a possible tangle. Every level of the tree contains all possible tangles of a certain threshhold $\psi_i$ which directly corresponds to the cost of bipartition $b_i$. 
The exact makeup of the tangle is determined by walking the path from the root to the node, and adding bipartition $b_i$ to the tangle in a left-oriented way, if it is a left child and
in a right-oriented way if it is a right child. An exemplary tangle search tree is illustrated in \autoref{fig:tangles-tree-example}.

\begin{figure}[h]
    \centering
    \includegraphics[width=0.8\textwidth]{figures/tangles-tree-example.png}
    \caption{An example of a possible tangles search tree for a set of bipartitions $\mathcal{B} = \{\{A_1, \overline{A_1}\}, \{A_2, \overline{A_2}\}, \{A_3, \overline{A_3}\} \}$. 
        Each level corresponds to the tangles of the order given by the bipartition $P_i$ that is indicated to the left of it.
        Let us take a look at the sole node in level 3. If we would want to find out, which orientations the corresponding tangle $T$ consists of, we just walk from
        the root to it and add the bipartitions in the direction indicated by the tree. We end up with $T = \{\overline{A_1}, \overline{A_2}, A_3\}$. 
        Figure taken with permission from \cite{klepperClusteringTanglesAlgorithmic2021}.}
    \label{fig:tangles-tree-example}
\end{figure}

We can now obtain a soft, hierarchical clustering from this tangle search tree. For this, the interesting nodes are those
where the tree splits up (as this represents a new clustering) and the leaves (which correspond to the clusters). 
For each of the \textit{splitting nodes}, we determine the set of \textit{characterizing cuts}.
A cut belongs to this set, if it is both oriented the same way inside the subtrees, and if it is oriented in a different way between the two subtrees.
To illustrate this, we take a look at the exemplary tree in \autoref{fig:tangles-tree-example}. Here, for the root node, $P_1$ is a characterizing cut (pretty trivially),
while $P_2$ is not: below the node $A_1$, the bipartition is both oriented to the left and to to the right, violating the requirement that the cuts are always oriented
the same way inside the subtrees. 

The characterizing cuts express some kind of agreement on how to align the bipartitions in the subtrees of the node, which we can leverage 
for a soft clustering. If we now want to determine, with what probability a point $a$ belongs to a cluster $C$ represented by a leaf in the tree, we simply walk down 
the tree from the root, and count at every splitting node how many characterizing cuts contain the point $a$, and how many don't. Normalized by the total amount of
characterizing cuts, we can interpret this as a probability to walk down either the left or the right path. To get a total probability that $a$ belongs to $C$, 
we simply multiply the probabilities that we obtain on the path to $C$ at every splitting node. The corresponding hard clustering is obtained by assigning each
node to the cluster it belongs to with the highest probability.
    
% Given a set of bipartitions $\mathcal{B}$, we first sort the bipartitions according to their costs given by the cost function $c$. 
% We now iterate over the set of bipartitions, while building up a binary tree in a level-wise fashion. For every bipartition $P_i$, 
% we add it to each node in level $i-1$, if it is consistent with the tangle inducded by the given node. 

\section{Ordinal Constraints}
A dataset of ordinal constraints consists of object for which we don't know their exact attributes or features, but only how they relate to other objects
in the form of comparisons such as \textit{item $i$ is closer to item $j$ than to item $k$}. Formally, we assume that $i, j, k$ are from 
a set $D$ where we can define a dissimilarity function $d: D \times D \to \R$. Note that $d$ which can be a metric on $D$, but does not have to be.
We can then express our above constraint as $d(i, j) < d(i, k)$. \\

Such data often appears when humans are asked to judge objects, as they are often naturally perform better on comparing objects
than on accurately placing them on an abstract scale \cite{stewartAbsoluteIdentificationRelative2005}. Applications are for example estimation of perceptual scales in psychophysics 
\cite{haghiriEstimationPerceptualScales2020} or crowd-sourcing clustering algorithms \cite{ukkonenCrowdsourcedCorrelationClustering2017}. 
We want to keep our focus here mainly the realm of psychophysics.


\subsection{Formats of Ordinal Constraints}
%TODO: Explain d as metric
Ordinal data usually takes two possible forms. 
The more general form of an ordinal data constraint is the quadruplet form, for example used in \cite{ghoshdastidarFoundationsComparisonBasedHierarchical2019}. 
Here, we have constraints such as $\text{d}(a,b) < \text{d}(c,d)$. A more commonly encountered form is the triplet form, which is simply a quadruplet where $a = c$, such as
$\text{d}(a,b) < \text{d}(a,d)$, used for example in \cite{vankadaraInsightsOrdinalEmbedding2021,haghiriComparisonBasedFrameworkPsychophysics2019}. We refer to datasets
consisting of triplet comparisons as \textit{triplet data}.
To obtain such triplet data, one might present to human participants three images $a, b, d$, with image $a$ as anchor point and ask them, which of the images $b,d$ are closer to $a$. \\

The way that participants are questioned is often varied on the context of the experiment (and might also influence their answers). An
example is presenting the participant with three images, and asking \textit{which is the most central image?}, or \textit{which is the odd one out?},
but the results are then always transformed back to triplet format for further prcoessing. For example, if $a$ is the \textit{odd-one-out} of the three elements $a,b,c$, 
then we know that $d(b,c) < d(b,a)$ and $d(c, b) < d(c,a)$. 

\subsection{Ordinal embeddings}
One of the most central problems consists of finding a so called \textit{ordinal embedding} of the data. If we have a set of triplet comparisons $T = \{t_1, t_2, \ldots t_n\}$, 
of the form $t_i = \left( a,b,c \right)$, encoding that $d(a,b) < d(a,c)$, 
we want to find a set of points $y_1, y_2, \ldots y_n \in \R^{m}$, such that they uphold the most of the original triplet constraints in $\R^{m}$ with the euclidean distance
as metric. More formally, we want to minimize \cite{vankadaraInsightsOrdinalEmbedding2021}:
\[
    \min_{y_1, \ldots y_n \in \R^{m}} \sum_{t=\left( i,j,k \right)  \in T} \mathbbm{1}_{ \|y_i - y_j\| < \|y_i - y_k\| }
.\] 
This problem is difficult to optimize and thus various algorithms have been propose that solve a relaxed or modified version of this objective function. 
These include Soft Ordinal Embedding (SOE) \cite{teradaLocalOrdinalEmbedding2014}, Generalized Non-Metric Multidimensional Scaling (GNDMS) \cite{agarwalGeneralizedNonmetricMultidimensional2007}, 
Crowd Kernel Learning (CKL) \cite{tamuzAdaptivelyLearningCrowd2011}, Fast Ordinal Triplets Embedding (FORTE) \cite{jainFiniteSamplePrediction2016},
T-Stochatic Triplet Embedding (T-STE) \cite{laurensvandermaatenStochasticTripletEmbedding2012} and various others. \\

It has been proven that if the original points come from the space $\R^{m}$, one can recover the points (up to scaling and orthogonal transformations) with $O(mn\log(n)$ triplet comparisons
\cite{jainFiniteSamplePrediction2016}. On this basis, an ordinal embedding can be used together with more classical machine learning algorithms, such as support vector machines or k-Means,
for other machine learning tasks. This approach has for example been used for classification \cite{tamuzAdaptivelyLearningCrowd2011, kleindessnerLensDepthFunction2017} and for clustering \cite{kleindessnerLensDepthFunction2017} on ordinal data. \\

%TODO This might better be something like "methods"?
\section{Applying Tangles to Triplet Data}
As we have made out in \autoref{theory:tangles}, the tangles algorithm operates on bipartitions of data that contain some information about the cluster structure.
If we have obtained a set of triplet data, we are now faced with the task of processing this data to appropriate bipartitions. In this work, we have developed two
methods for this which we call \textit{landmark cuts} and \textit{majority cuts}. We will elaborate on the methods and intuitions in the following sections.

\subsection{Landmark cuts}\label{theory:landmark_cuts}
In recent years, there have been algorithms that hope to speed up ordinal embedding by focussing on so-called \textit{landmarks}\cite{ghoshLandmarkOrdinalEmbedding2019, andertonScalingOrdinalEmbedding2019}, which are objects in the dataset that for which we know all (or all relevant) triplet comparisons.
The definitions of what constitutes landmarks varies a bit, but we will use one that is used in \cite{haghiriComparisonBasedFrameworkPsychophysics2019}.
Assume we have a set of objects $D$, as well as a set of $T$ triplet constraints of the form $(a,b,c)$, indicating that $d(a,b) < d(a,c)$. In a landmark setting, we have a set 
of $m$ objects $L \subset D$, for which $\forall l_1, l_2 \in L \forall x \in D: (x, l_1, l_2) \in T \vee (x, l_2, l_1) \in T$. This admits to a very natural notion of bipartitions: 
for each combination of landmarks $l_1, l_2$, we can make a bipartition $P = \{A, \overline{A}\}$ by assigning all points closer to $l_1$ to $A$ and all 
those closer to $l_2$ to $\overline{A}$. Thus we can define $A_{12} = \{ x \in D \mid \left( x, l_1, l_2 \right) \in T \}$, denoting as $P_L = \{A_ij  \mid i, j \in 1\ldots m\}, i < j\}$ 
all such bipartitions on $L$. These bipartitions are then called \textit{landmark cuts}.
Later in the simulations, as tangles is not reliant on having all triplet constraints for all landmark objects, we will simply sample a subset $P'_{L} \subset P_L$ of bipartitions,
which corresponds to repeatedly picking some objects $a, b$ and sampling all triplet comparisons to all other objects $x \in D$. \\

These landmark cuts intuitively capture a cluster structure: the more close $x$ is to $l_1$ according to $d$, the more likely that $A_{12}$ will contain $x$. In the end, $A_{12}$ will
consist of the points that are in some sense close to $l_1$. In the euclidean space, this notion is very easily captured: A landmark cut between $l_1, l_2$ is
simply a linear cut between the two points, as illustrated in \autoref{theory:landmark_cuts}.

    \begin{figure}[ht]
        \centering
        \includegraphics[width=0.8\textwidth]{figures/landmark_cut.png}
        \caption{Example of a simple landmark cut, with the two triangles being the landmarks $l_1, l_2$, where all red items (left of the line) 
        would be assigned to $A$, all blue items to $\overline{A}$ (or equivalently the other way around).}
        \label{fig:landmark_cut}
    \end{figure}

The landmark approach is a quite unusual way of sampling triplet questions.
In most triplet experiments, the triplets to sample are chosen randomly \cite{kleindessnerLensDepthFunction2017, haghiriEstimationPerceptualScales2020} 
or according to some metric (f.e. maximizing some measure of gained information) \cite{roadsEnrichingImageNetHuman2021}. 
There was no experimental dataset available which is sampled according to a landmark approach, where the objects exhibit a cluster structure, thus we rely on simulations
for testing our methods.

\subsection{Majority cuts}
As explained in \autoref{theory:landmark_cuts}, sampling triplet data in a landmark-fashion is not very widely used in current practice. Due to this, we aimed
to also develop a more general approach to processing tangles to cuts that can be applied to any set of triplet comparisons $T$ regardless of the sampling method.
For this, we again use the intuition that triplets tell us something about the closeness of data point, and thus about the cluster structure. Assume again
that we have a set of objects $D$ and a set of triplet comparisons $T$. We fix two points $a,b \in D$. Assume that $a,b$ are close and we and sample a random point $x$,
then, with high probability, it will be that $d(a,b) < d(a,x)$. The reverse holds, if $a,b$ are far away. Now we can take this the other way around: 
if we observe a triplet $(a,b,c) \in D$, then we can more reasonably assume that $a,b$ are close than $a,c$ being close. This leads us to the following method of defining
a set of close points of $a$: Let $L_x = \{t \in T | t = (x, b,c), b,c \in D\}$ be the set of all available triplets where $a$ is in the left position, and equivalently $M_x$ and $R_x$ 
for set of triplets where $x$ is in the middle and right position, respectively. Then we define: $P_a := \{x \in D \mid \left|   L_a \cap M_x\right| < \left| L_a \cap R_x \right| \}$
which is the set of all points that are more often closer to $a$ than they are farther away. We refer to these bipartitions as \textit{majority cuts}. Again, they capture
some cluster structure by assigning points that are close together to the same bipartition.

%TODO: Examining the quality of the cuts


\cleardoublepage

\chapter{Applying Tangles to Triplet Data}\label{methods}
As we have made out in \autoref{theory:tangles}, the tangles algorithm operates on bipartitions of data that contain some information about the cluster structure.
If we have obtained a set of triplet data, we are now faced with the task of processing this data to appropriate bipartitions. In this work, we have developed two
methods for this which we call \textit{landmark cuts} and \textit{majority cuts}. We will elaborate on the methods and intuitions in the following sections.

\section{Landmark cuts}\label{theory:landmark_cuts}
In recent years, there have been algorithms that hope to speed up ordinal embedding by focussing on so-called \textit{landmarks} \citep{ghoshLandmarkOrdinalEmbedding2019, andertonScalingOrdinalEmbedding2019}, which are objects in the dataset for which we know all (or all relevant) triplet comparisons.  
The definitions of what constitutes landmarks varies a bit, but we will use the one that is used in \cite{haghiriComparisonBasedFrameworkPsychophysics2019}.
Assume we have a set of objects $D$, as well as a set of $T$ triplet constraints of the form $(a,b,c)$, indicating that $d(a,b) < d(a,c)$. In a landmark setting, we have a set 
of $m$ objects $L \subset D$, for which 
\begin{align*}
\forall l_1, l_2 \in L \; \forall x \in D: (x, l_1, l_2) \in T \vee (x, l_2, l_1) \in T. 
\end{align*}
Landmarks make it very easy to define a set of bipartitions on triplet data:
for each combination of landmarks $l_1, l_2$, we can make a bipartition $P = \{A, \overline{A}\}$ by assigning all points closer to $l_1$ to $A$ and all 
those closer to $l_2$ to $\overline{A}$. Thus we can define the bipartition between landmark points $l_1, l_2$ as $A_{12} = \{ x \in D \mid \left( x, l_1, l_2 \right) \in T \}$, 
denoting as $P_L = \{A_{ij}  \mid i, j \in \{1\ldots m\},\, i < j\}$ all such possible bipartitions on $L$. 
These bipartitions are then called \textit{landmark cuts}.
Later in the simulations, as tangles is not reliant on having all triplet constraints for all landmark objects, we will simply sample a subset $P'_{L} \subset P_L$ of bipartitions,
which corresponds to repeatedly picking some objects $a, b$ and sampling all triplet comparisons to all other objects $x \in D$. \\

Landmark cuts intuitively capture a cluster structure: the more close $x$ is to $l_1$ according to $d$, the more likely that $A_{12}$ will contain $x$. In the end, $A_{12}$ will
consist of the points that are in some sense close to $l_1$ (how strong this closeness is depends on how close $l_1$ and $l_2$ are). 
In the euclidean space, this notion is very easily captured: A landmark cut between $l_1, l_2$ is
simply a linear cut between the two points, as illustrated in \autoref{fig:landmark_cut}.

    \begin{figure}[ht]
        \centering
        \includegraphics[width=0.8\textwidth]{figures/landmark_cut.png}
        \caption{Example of a landmark cut on a euclidean data set, with the two triangles being the landmarks $l_1, l_2$, where all blue items (left of the line) 
        would be assigned to $A$, all red items to $\overline{A}$ (or equivalently the other way around).}
        \label{fig:landmark_cut}
    \end{figure}

The landmark approach is a quite unusual way of sampling triplet questions.
In most triplet experiments, the triplets to sample are chosen randomly \citep{kleindessnerLensDepthFunction2017, haghiriEstimationPerceptualScales2020} 
or according to some metric, for example maximizing some measure of gained information \citep{roadsEnrichingImageNetHuman2021}. 
There was no experimental dataset available which is both sampled according to a landmark approach and exhibits a cluster structure, thus we rely on simulations
for testing landmark cuts.

\section{Majority cuts}\label{theory:majority_cuts}
As explained in \autoref{theory:landmark_cuts}, sampling triplet data in a landmark-fashion is not very widely used in current practice. Due to this, we aimed
to also develop a more general approach to processing tangles to cuts that can be applied to any set of triplet comparisons $T$ regardless of the sampling method.
For this, we again use the intuition that triplets tell us something about the closeness of data points, and thus about the cluster structure. 

Assume again that we have a set of objects $D$ and a set of triplet comparisons $T$. We fix two points $a,b \in D$. Assume that $a,b$ are close and we sample a random point $x$.
With high probability, it will be that $d(a,b) < d(a,x)$. The reverse holds if $a,b$ are far away. Now we can take this the other way around: 
if we observe a triplet $(a,b,c) \in D$, then we can more reasonably assume that $a,b$ are close than $a,c$ being close. This leads us to the following method of defining
a set of close points of $a$: Let $L_x = \{t \in T | t = (x, b,c), b,c \in D\}$ be the set of all available triplets where $a$ is in the left position, and equivalently $M_x$ and $R_x$ 
the set of triplets where $x$ is in the middle and right position. Then we define: $P_a := \{x \in D \mid \left|   L_a \cap M_x\right| < \left| L_a \cap R_x \right| \}$
which is the set of all points that are more often closer to $a$ than they are farther away. We refer to these bipartitions as \textit{majority cuts}, and to the point $a$ that
as the \textit{anchor point} of the cut $P_a$. These majority cuts capture
some cluster structure by assigning points that are close together to the same bipartition. Majority cuts can be made more flexible by including a ratio $r$ that controls the size of the cuts. 
We then define $P_a(r) := \{x \in D \mid \left|   L_a \cap M_x\right| < r \cdot \left| L_a \cap R_x \right| \}$ and call $r$ the \textit{radius} of the cut. 
The definition of majority cuts given before is then simply the case $r = 1$. 

Next, we want to gain some intuition about the form of the majority cuts. What we would expect, assuming that we had all triplets, is that each cut
$P_a(r)$ is a ball around $a$ that contains the $n \cdot \frac{r}{r+1}$ points that are closest to $a$ according to the distance measure $d$ that defined the triplets. In a euclidean setting for 
$r = 0.5$ we thus expect $P_a(1)$ to be a ball around $a$ that contains the $\frac{n}{3}$ points that are closest to $a$. To test this, we plot a majority cut with a fixed anchor point on a 
simple mixture of gaussians in \autoref{fig:majority_cut}. 
We can see that we get closer to the ideal form of a ball around $a$ (the point marked with a cross) with radius $\frac{n}{3}$ as we increase the amount of sampled triplets.
When we have fewer than all triplets available, our cut contains some noise in the form of points that in the cut together with $a$, but are outside of the $\frac{n}{3}$ ball around $a$ (and might in fact be very far away).

This intuition already gives us hints how we might want to choose the radius: if we pick a smaller radius, we will detect smaller clusters versus when we pick a larger radius. 
%TODO: Investigate this claim?
In particular, we should not pick a radius smaller than the smallest cluster we want to detect, else the tangles algorithm will quickly have troubles consistently aligning
the cuts. On the contrary, we are safe if we pick a radius that is a bit larger, as the tangles algorithm can work well with cuts that contain a cluster together with some additional data points 
(as long as they are not always the same points on all cuts).

\onecolumn
\begin{figure}[ht]
    \centering
    \subfloat[500 triplets]{%
      \resizebox{0.5\textwidth}{!}{\input{figures/results/majority-cut-8-n_triplets-500.pgf}}
  }
    \subfloat[5000 triplets]{%
      \resizebox{0.5\textwidth}{!}{\input{figures/results/majority-cut-8-n_triplets-5000.pgf}}}
    \hfill
    \subfloat[20000 triplets]{%
      \resizebox{0.5\textwidth}{!}{\input{figures/results/majority-cut-8-n_triplets-20000.pgf}}}
    \subfloat[All (106200) triplets]{%
      \resizebox{0.5\textwidth}{!}{\input{figures/results/majority-cut-8-n_triplets-106200.pgf}}}
    \caption{Depiction of how majority cuts look like on a simple mixture of gaussians, with a varying number of triplets available. 
        We plot just one bipartition $P_a$ with radius $r=0.5$, which was generated according to the procedure in \autoref{theory:majority_cuts}. The big X marks the anchor point $a$. 
        The bipartition $P_a$ consists of the orange triangles, which are the points that are twice as often closer to $a$ than they are not (according to the drawn triplets). 
        The blue points are those not in $P_a$.}
    \label{fig:majority_cut}
\end{figure}

\cleardoublepage

%% 
\chapter{Simulations}\label{simulations}
In this chapter and the following one, we want to explore how tangles perform on triplet data with the methods we proposed in \autoref{methods}. 
At first, we focus on simulations, as this allows us to control our data precisely. 
We will show in which cases Tangles perform well, in which ones they don't and what to keep in mind when applying the algorithm. \\

As datasets, we have decided on a mixture of gaussians and a hierarchical setup. In both cases, we will then generate triplets from the data points.
The gaussian setup serves as first baseline: it is a de-facto standard in clustering and a lot of real-life data is approximately gaussian. 
We have decided against using more complex data sets such as two-moons, as the focus lies on how the algorithms
interact with the triplets generated on the data and deal with additional complexities in that domain (such as triplets being corrupted, triplets missing, et cetera...). \\

The hierarchical setup takes the form of a noisy hierarchical block matrix, introduced by \cite{balakrishnanNoiseThresholdsSpectral2011}.
We use it to illustrate a second property of Tangles: in addition to pure clustering, we also produce a hierarchical tree, 
which can be used for hierarchical clustering. 

\section{Terms and methods used}
To evaluate the performance of the Tangles algorithm, we will need some metrics and other methods to use as a baseline. To compare a clustering against a ground truth,
we will use the \textit{normalized mutual information score} (NMI), which is independent of the cluster labels. 
$1.0$ indicates the same clustering (up to label permutations), $0.0$ indicates absolutely no mutual information between the clusterings (such as when our prediction
puts all data points in a single cluster). To compare the quality of different hierarchies, we make use of the \textit{average adjusted rand index} (AARI), 
introduced by \cite{ghoshdastidarFoundationsComparisonBasedHierarchical2019}. The AARI extends the \textit{adjusted rand index} (ARI), which is a clustering performance measurement
similar to the NMI, to compare hierarchies. In AARI, we calculate the ARI over all levels of the hierarchies we want to compare and average over all the obtained scores. 
% A further discussion on how we use the AARI for Tangles can be found in \label{sec:hierarchical_data}.


To generate the triplets, we first take the data and calculate a (dis)similarity on it, for example the euclidean distance between two data points. This allows
us to determine whether $(a,b,c)$ or $(a,c,b)$.
Then, we can use two approaches of drawing triplets. The first one is sampling triplets randomly and uniformly from the set of all triplets. The second one is a landmark approach:
we fix two randomly sampled points $a,b$, with $a \neq b$, sample all triplet that have the form $(x,a,b)$ (or equivalently $(x,b,a)$.  For the Landmark Tangles, only the 
second kind of triplets are useable, as discussed in \autoref{methods}, thus we will mostly stick to this format. 
We now have to differ between two possible ways of altering the triplets. We can add \textit{noise}
to the triplets. If we have a noise level of $p$, then every triplet is flipped with probability $p$, $(a,b,c)$ would be turned into $(a,c,b)$. We can also reduce
the \textit{density}, meaning that instead of sampling all triplets, we sample only a fraction $d$ of all triplets. In this section, when we use the term density, we
will refer to triplets sampled in a landmark approach, while when explicitly using the number of drawn triplets, we refer to a uniform sampling.

As a baseline, we will be using a combination of ordinal embedding together with a clustering algorithm. 
We first use an ordinal embedding algorithm to get an embedding of the triplet data. This approach has also been used in \cite{kleindessnerLensDepthFunction2017}. 
It has the advantage of also working with triplet data as opposed to clustering on the data directly, giving a more fair baseline, while still being very straightforward.
Afterwards, we use a standard clustering algorithm for euclidean data on the obtained embedding.  
There exist numerous algorithms for ordinal embedding (see \cite{vankadaraInsightsOrdinalEmbedding2021} for an overview) and for clustering.

For the ordinal embeddings, we have used Soft Ordinal Embedding (SOE \cite{teradaLocalOrdinalEmbedding2014}), as this has been identified by \cite{vankadaraInsightsOrdinalEmbedding2021} as one of the 
top-performing ordinal embedding algorithm for a variety of use cases. When testing out different baselines, we have also found SOE to consistently have top performance among 
all tested ordinal embedding algorithms. Additionally, we have included T-Stochatic Triplet Embedding (T-STE \cite{laurensvandermaatenStochasticTripletEmbedding2012}) 
to have a second baseline. For the clustering, have decided to go with k-Means, as one of the most basic clustering algorithms with usually good performance.

Additionally, we have included ComparisonHC \citep{ghoshdastidarFoundationsComparisonBasedHierarchical2019} as another baseline. This is another algorithm that doesn't
construct an embedding before clustering, and thus might allow for a more fair comparison to the tangles algorithm. Especially in the setting of a mixture of gaussians, 
the fact ordinal embedding algorithms aim to reconstruct a euclidean embedding seems to already introduce some bias that might benefit the ordinal embedding algorithms.
Normally, ComparisonHC is a hierachical clustering algorithm, which produces a dendrogram as its output. To use it as a baseline for clustering, we simply
cut the dendrogram off such that it produces the desired amount of clusters.

In the experiment figures, we will  use the abbreviations L-Tangles for Landmark Tangles, and M-Tangles for Majority Tangles.

\section{Gaussian data}\label{sec:gaussian_data}
\subsection{Experimental setup}
We generate a mixture of gaussians as follows: We draw a number of points $n$ each from $k$ different gaussian distributions with means $\mu_1, \mu_2, \ldots \mu_k$ and 
variances $\nu_1, \nu_2, \ldots \nu_k$. Each point gets assigned a label $y_i$ that corresponds to the number $i$ of the gaussian distribution it was drawn from.
For all of the experiments, unless mentioned otherwise, we use $n=20$, $k=3$, fixed means $\mu_1 = \begin{bmatrix} -6.0 & 3.0 \end{bmatrix}, 
\mu_2 = \begin{bmatrix} -6.0 & -3.0 \end{bmatrix},  \mu_2 = \begin{bmatrix} 6.0 & 3.0 \end{bmatrix}$ and a constant variance for all distributions of $\nu = I$, 
with $I$ as the identity matrix. A plot of the data set can be seen in \autoref{fig:dataset-gauss}. We generate the triplets via the euclidean distance between the points, so that
the triplet $(a,b,c)$ implies that $\|a - b\| < \|a - c\|$. For the Tangles algorithms, we use an agreement of $a=7$ (around 1/3 the size of the smallest
clusters we want to detect, in accordance with \cite{klepperClusteringTanglesAlgorithmic2021}), and a radius of $r=\frac{1}{3}$, for the majority
tangles, such that the cuts roughly have the diameter of the clusters.

\begin{figure}[h]
    \centering
    \resizebox{0.8\textwidth}{!}{\input{figures/results/gaussian_small_tangles_clustering.pgf}}
    \caption{}
    \label{fig:dataset-gauss}
    \caption{An example draw of the gaussian mixture used for our experiments.}
\end{figure}

\subsection{Lowering density}\label{lower_density}
First, we want to observe how our methods behave under different numbers of triplets present. As the number of triplet grows on the order of $O(N^3)$, it is only feasible to obtain
all triplets for very small datasets. Even with $N = n \cdot k = 60$, as in our case, there are already $106200$ triplets possible. If we imagine that the triplets have to be obtained 
through experiments with real people (such as in psychophysics settings), we might be able to get a few thouand triplets at most. If an algorithm performs better with a lower amount
of triplets, this can quickly translate into really large time, labor and money savings. To test this, we have drawn an increasing amount of triplets from our dataset in a landmark format.
We have plotted the results in \autoref{fig:lower_density_small}. We have also repeated this experiment for a larger number of datapoints, as this is a particularly interesting case. 
Usually, the larger the dataset, the smaller the percentage of all triplets we use, as it grows with $O(N^3)$. However, ordinal embedding algorithms empirically perform well with a lot
less triplets (for example requiring on the order of $O(n d \log(n))$ for euclidean data \citep{jainFiniteSamplePrediction2016}). 
Ideally, we would like for the Tangles algorithm to behave similarly. To test this, we have repeated the experiment with $n=200$ in \autoref{fig:density-change}, and lowered densities.
All other parameters were kept the same.

When looking at the plots, we can see that L-Tangles is performing at least as well as SOE, and even significantly better for a lot of densities in the case of $n=200$. 
As we would have expected, T-STE performs about as well as SOE, albeit a bit worse (and interestingly a lot worse for the larger dataset). ComparisonHC and M-Tangles
perform about the same level, but both stay far behind L-Tangles and SOE. With the larger dataset, we couldn't test ComparisonHC, as our obtained implementation requires 
constructing a $n^4$ matrix during the training step (this could possibly be remedied using a different implementation).

\onecolumn
\begin{figure}[ht]
    \centering
    \subfloat[20 points per cluster]{%
    \resizebox{0.5\textwidth}{!}{\input{figures/results/lower_density_small.pgf}}
    }
    \subfloat[200 points per cluster]{%
    \resizebox{0.5\textwidth}{!}{\input{figures/results/lower_density_large.pgf}}
    }
    \caption{
        We plot the NMI of different clustering method against the percentage of the triplets generated from a draw of a gaussian mixture with $3$ clusters. 
        We draw $20$ data points from each cluster for the left plot, and $200$ data points for the right plot 
        On the x-axis we have the density, where a density of $0.1$ means that we only use 10\% of the total number of triplets. The embedding methods (SOE, T-STE) are 
        followed by k-Means. The Tangles methods (L-Tangles, M-Tangles), are applied with $a=7$. ComparisonHC was left out of the right plot due to computational issues.}
    \label{fig:density-change}
\end{figure}


\subsection{Adding noise}\label{sec:adding-noise}
Next, we want to observe how our algorithm behaves under added noise. This is an important model: most applications of triplet data use triplets that are generated from
real humans. They might disagree on which objects are closer and which are not, which can be modelled as noise on the responses of our triplets. The higher the noise, 
the more disagreement is there about the similarities of objects, so it would matter more about which person you ask than which objects you present them. 
%TODO M-Tangles ComparisonHC
We have plotted our results in \autoref{fig:adding-noise}. 
We observe that L-Tangles falls off a lot quicker in performance than SOE, and falls off a bit harder than T-STE, but they are in the same area. We observe however, that until
relatively large noise levels ($>0.1$), all algorithms performs the same, meaning that L-Tangles can still perform well with low to medium levels of noise.

\begin{figure}[h]
    \centering
    \resizebox{0.7\textwidth}{!}{\input{figures/results/adding_noise_small.pgf}}
    \caption{}
    \label{fig:adding-noise}
    \caption{
        We plot the NMI of our chosen clustering methods against the noise that we introduce on the triplets.  We use $3$ clusters and $20$ data points per cluster, sampling all possible triplets. On a noise level of $0.1$, this means that we flip 10\% of the triplets around (turning for example $(a,b,c)$ to $(a,c,b)$). 
    }
\end{figure}

\subsection{Adding noise and lowering density}
In this experiment, we compare \autoref{sec:lowering-density} and \autoref{sec:adding-noise}. We have seen that L-Tangles can perform well with a low
amount of triplets (a bit better than SOE), and have reasonable performance for noisy triplets (worse than SOE). We are now interested to see how large the area is
where L-Tangles can still outperform SOE when we consider both a low density and noisy triplets, as these two factors are probably the most interesting variables
when choosing an algorithm to evaluate triplet data. We have generated two heatmaps, which can be found in \autoref{fig:noise-density-heatmaps}. 
There we can see that L-Tangles outperforms SOE in quite a large region in the low-noise , low-density regime. We would imagine this effect to be even larger
for a $n=200$ setup, but found this computationally too intensive. On the contrary, SOE performs better in the high-density, high-noise regions, with about similar performance
for the cases of low-noise high-density (perfect clustering) and high-noise low-density (random clustering).
% TODO: Repeat for n=200
% TODO: green border around the parts where L-Tangles outperforms?

\onecolumn
\begin{figure}[ht]
    \centering
    \subfloat[SOE]{%
    %\resizebox{0.5\textwidth}{!}{\input{figures/results/gaussian_small_lower_density_noise_heatmap_soe.pdf}}
        \resizebox{0.5\textwidth}{!}{\includegraphics{figures/results/gaussian_small_lower_density_noise_heatmap_soe.pdf}}
    }
    \subfloat[L-Tangles]{%
        \resizebox{0.5\textwidth}{!}{\includegraphics{figures/results/gaussian_small_lower_density_noise_heatmap_l_tangles.pdf}}
    }
    \caption{
        We plot a heatmap of the NMI of SOE and L-Tangles over changing noise and the density of our the triplets. 
        The regions with a darker shade of blue indicate better performance of the algorithm. 
        Note that the noise increases as we move down, and the
        density decreases as we move right.  We use $3$ clusters and $20$ data points per cluster, sampling all possible triplets. 
    }
    \label{fig:noise-density-heatmaps}
\end{figure}

\subsection{Missing triplets}
% TODO
In this experiment, we will use the small gaussian data set and gradually sample an increasing number of triplets for it. 
This is very similar to the setup in \autoref{lower_density}, but this time we sample triplets uniformly at random without replacement
from the set of all triplets. In this setup, we will not be able to use Landmark Tangles, as there will be missing values in the cuts. 
In principle, one could imagine using some kind of imputation to fill in the missing triplets. This is not an effective method,
as there are simply way too little triplets compared to the set of all triplets used. We have shown in \autoref{?} 
how Landmark Tangles behaves under different imputation methods with a decreasing number of triplets. One ca see that all methods fall 
off very quickly in performance, a lot faster than the other clustering methods.

In \autoref{?}, we have shown the performance of our other algorithms over the number of triplets present.


% TODO: Experiment that shows how quickly LT starts falling off, maybe with different imputations?
\subsection{The case of weird geometry}

\subsection{Discussion}

\section{Hierarchical data}\label{sec:hierarchical_data}
\subsection{Experimental setup}
We generate a noise hierarchical block matrix \citep{\cite{balakrishnanNoiseThresholdsSpectral2011}.
The dendrogram described by this model has the form of a complete binary tree, and the similarities of the data points are described via a similarity matrix $M$.
In this matrix the elements in the same cluster have the highest similarity $mu_0$, and for each level $l$ in the dendrogram that two classes are removed
from each other, their similarity decreases by $delta$. We can then add a noise matrix $R$ to the similarity matrix and receive the noisy hierarchical block matrix $M' = M + R$. 
In our setup, we use a noise matrix $R$ where every entry is simply drawn from a normal distribution with mean $0$ and standard deviation $\sigma$.
More about the generation process can be read in \cite{ghoshdastidarFoundationsComparisonBasedHierarchical2019}.
%TODO: Image of similarity matrix?
% can we make this bigger?
We choose a relatively simple setup of $4$ clusters with $10$ data points each, an initial class similarity $\mu_0 = 5$ and a similarity decrease of $\delta = 1$.
In this setup, there are two kinds of noise we encounter: the noise that is injected into the hierarchy itself via the noise matrix and the noise that is added to the triplets itself.
The triplet noise has the same form as in \autoref{sec:gaussian_data}, on noise $p$ we simply flip every triplet with probability $p$. We will investigate how our algorithm does
under both noise models, as well as under a lowered density.

When evaluating, we compare both the final clustering (the lowest level of the hierarchical block matrix) as well as the obtained hierarchy to each other.
% TODO: Explain more about the AARI?
%As Tangles does not produce a dendrogram, we might have a problem obtaining the desired amount of clusters in a certain level. In this case, we simply fill up 
%#For comparing the hierarchy, we can only use Tangles and ComparisonHC, as the embedding methods do not 

\subsection{Adding triplet noise}
Similar to the gaussian setup, we have simply increased the noise on the sampled triplets and have evaluated the performance of our algorithms.

\subsection{Adding hierarchy noise}

\subsection{Lowering density}

\subsection{Discussion}



\cleardoublepage

\chapter{Real World Data from Psychophysics}\label{real}
Psychophysics is a field of study that investigates the influence of physical stimuli on human perception. An example would be the relationship between the wavelength of 
light and the color sensation that the light produces, see \cite{shepardAnalysisProximitiesMultidimensional1962}. To investigate this relationship, researchers in psychophysics
often set up experiments to collect triplets from human participants. For the example of color perception, a researcher can place a participant in front of a computer 
screen, and repeatedly show them three different colors, $a$, $b$, $c$, together with the question \textit{is a more similar to b or c?} 
The answers of the participants then constitute a triplet data set, which is often analyzed using ordinal embeddings afterward.

In this chapter, we choose a data set consisting of triplets from psychophysics and analyze it using tangles. 
We compare the results of our analysis with the results that other researchers have achieved using more established methods in psychophysics, such as ordinal embeddings.
This chapter thus shows the strengths and weaknesses of tangles on real experimental data.

\section{Data background}
The data we use was collected by Schönmann during her bachelor thesis in \cite{inesschonmannSimilarityJudgementsNatural2021}. Schönmann originally
constructed the dataset to investigate how the formulation of a triplet question influences 
the perception of a person. It consists of the triplet data obtained from multiple
participants using different questions and image sets.  The resulting data sets were then analyzed to show differing similarity perceptions of the
participants depending on how the triplet questions were formulated and which image set was used.
To collect the data, the participants were presented with three images randomly drawn from the specific set of images, together with one out of four different questions. 

The possible questions are: \textit{Which is the odd one out?}, \textit{Which one is more similar?},
\textit{Which one looks more similar?} and \textit{Which concept is more similar?}, out of which we only study the question \textit{Which is the odd one out?} 
The images that are shown to the participants come from three different image sets: \textit{action}, \textit{taxonomic} and \textit{thematic}, out of which we only use the \textit{thematic} image set.
It consists of 5 classes which can be roughly divided into two themes: barn (straw, hay, pitchforks) and kitchen (forks, dishwashers).

The responses of the participants are converted to triplets as follows: if the participant is presented with images $a, b, c$ and signals that $a$ is the odd one out, 
we know that $b$ must be more similar to $c$ than to $a$ and $c$ must be more similar to $b$ than to $a$. Thus, we can gain two triplets from this answer: $(b,c,a)$ and $(c,b,a)$.
For each combination of a participant, image set, and question, the data set consists of $462$ unique triplets. 
Participant $2$ repeated the experiment over a month later, but we have decided not to include those triplets to stay faithful to the original evaluation.

The data set is particularly suited for our tangles experiment, as the image set consists of images from different classes (for example straw, hay, et cetera). 
We expect that this makes the resulting triplets particularly suited for clustering. Additionally, the analysis done by Schönmann already identified certain 
clusters in the data, which we can expect to see in the tangles analysis as well.

\section{Applying Tangles}
\subsection{Setup}
In this section, we show how to apply tangles to the triplet data set of Schönmann.
As there is a lot of data present, it is not feasible to repeat our evaluations for all possible data points.
We choose the triplets of participant 2 to step through rigorously, and then briefly repeat our evaluations for participant 3. We expect similar results for both observers,
as we keep the question and the image set the same.

For the image set, we use the \textit{thematic} one, as Schönmann has reported embeddings that can be cleanly 
separated into different categories.  We also use the \textit{odd-one-out} triplets, as these have been reported in the work by Schönmann as 
having the highest embedding accuracy. As the triplets are not in a landmark format, but uniformly sampled from the set of all triplets, we use majority cuts. 

We process the triplets to cuts using the majority cuts approach with a radius of 1 and apply tangles using an agreement of 3 and with the mean manhattan cost function (see \autoref{mmcf} in \autoref{sims-methods}). 
As a visualization aid, we plot the clustering onto a 2d-embedding from SOE. Care must be taken: the embedding
from SOE is not a ground-truth embedding and just serves as a visualization aid. If the elements of a cluster are far 
apart in this visualization, this does not mean that the clustering is wrong – it could just as well be that the distances in the embedding do not correctly 
represent the ground truth similarity of the data points.

In the analysis, Schönmann qualitatively identified different clusters for participant 2 on the \textit{thematic} image set with the \textit{odd-one-out} triplets. 
To do this, she identified a separating line in the SOE embedding of the triplets, which divided the objects into the two categories \textit{barn} (hay, pitchfork, straw) and 
\textit{kitchen} (dishwasher, fork). We expect to see a similar divide in the clusters produced by tangles. 
As a follow-up, we use the hierarchical clusterings and the explanations produced by tangles to get further insights into the produced clustering. 

\subsection{Embedding and clustering}
We first reproduce the results by Schönmann by embedding the data with SOE into two dimensions 
(see \autoref{fig:ines-embedding-a}). In this embedding, one can linearly separate two sets of clusters, which correspond to a divide between kitchen objects 
(dishwasher, fork) and barn objects (hay, straw, pitchfork), as reported by Schönmann in her thesis. Then we process the triplets to majority cuts, apply tangles 
and obtain a clustering. These cluster labels are visualised in \autoref{fig:ines-embedding-b}) on the embedding from SOE. 

\begin{figure}[h]
    \centering
    \subfloat[Original classes]{%
    \resizebox{0.5\textwidth}{!}{\input{figures/results/ines_vp2_odd_thematic_classes.pgf}}
    \label{fig:ines-embedding-a}
    }
    \subfloat[Tangles clusters]{%
    \resizebox{0.5\textwidth}{!}{\input{figures/results/ines_vp2_odd_thematic_tangles.pgf}}
    \label{fig:ines-embedding-b}
    }
    \hfill
    \subfloat[Embedding with images]{%
    \centering
    \resizebox{0.85\textwidth}{!}{\includegraphics{figures/results/vp2_embedding_thematic_stimuli_images.pdf}}
    \label{fig:ines-embedding-c}
    }
    \caption{
        Embedding of the \textit{odd-one-out} triplets from participant 2 on the thematic image set.
        In a), we see the original classes, and in b) we see the predictions that we receive by
        applying majority tangles with an agreement of 3 and a radius of 1.
        In c), we plotted the original images at the coordinates of their respective SOE embedding.
    }
    \label{fig:ines-embedding}
\end{figure}

We can see that the clustering from tangles (\autoref{fig:ines-embedding-b}) also produces a similar divide between kitchen (orange triangles) and barn (blue circles) items. 
However, we see a third cluster structure (grey squares), which is a mix between two pitchforks and a kitchen fork. We want to determine whether this is an erroneous clustering, which might arise
from too few triplets sampled, or possibly a new insight that we gained.

When we look at items in the grey squares cluster (depicted in \autoref{fig:thematic-images-forks-a}), we notice that they look more dissimilar to their
counterparts in the kitchen or barn cluster. The fork (left in \autoref{fig:thematic-images-forks-a}) has a design that is more reminiscent of pitchforks, and the two pitchforks
look cleaner than the other pitchforks in \autoref{fig:thematic-images-forks-c} (no dirt on them, not depicted lying in grass). Thus, it makes sense that these three items are judged
as more similar to each other than to their counterparts in the kitchen and barn cluster and thus get put into a separate cluster. 
Interestingly, that is an insight that could not be reached from the SOE embedding alone, as the items are relatively far away from each other in the embedding. Thus, tangles might provide valuable information for a researcher. 

\begin{figure}[h]
    \centering
    \subfloat[Third cluster of different forks (gray squares)]{%
        \label{fig:thematic-images-forks-a}
        \resizebox{0.333\textwidth}{!}{\includegraphics{figures/thematic_stimuli/fork_03s.jpg}}
        \resizebox{0.333\textwidth}{!}{\includegraphics{figures/thematic_stimuli/pitchfork_09s.jpg}}
        \resizebox{0.333\textwidth}{!}{\includegraphics{figures/thematic_stimuli/pitchfork_16s.jpg}}
    }
    \hfill
    \subfloat[Other kitchen forks]{%
        \resizebox{0.2\textwidth}{!}{\includegraphics{figures/thematic_stimuli/fork_05s.jpg}}
        \resizebox{0.2\textwidth}{!}{\includegraphics{figures/thematic_stimuli/fork_06s.jpg}}
        \resizebox{0.2\textwidth}{!}{\includegraphics{figures/thematic_stimuli/fork_many.jpg}}
        \resizebox{0.2\textwidth}{!}{\includegraphics{figures/thematic_stimuli/fork_plastic_light.jpg}}
        \resizebox{0.2\textwidth}{!}{\includegraphics{figures/thematic_stimuli/fork_wood.jpg}}
    }
    \hfill
    \subfloat[Other pitchforks]{%
        \label{fig:thematic-images-forks-c}
        \resizebox{0.25\textwidth}{!}{\includegraphics{figures/thematic_stimuli/pitchfork_02s.jpg}}
        \resizebox{0.25\textwidth}{!}{\includegraphics{figures/thematic_stimuli/pitchfork_03s.jpg}}
        \resizebox{0.25\textwidth}{!}{\includegraphics{figures/thematic_stimuli/pitchfork_04s.jpg}}
        \resizebox{0.25\textwidth}{!}{\includegraphics{figures/thematic_stimuli/pitchfork_15s.jpg}}
    }
    \caption{
        The images of all forks that are present in the thematic data set. In a), we show the pitchforks and forks that landed in
        the gray squares cluster in the tangles clustering (see \autoref{fig:ines-embedding}), which contained a mixture of kitchen and barn items.
        In b), we have depicted all other kitchen forks (orange triangle cluster, kitchen items) and in c) we depict all other pitchforks 
        (blue circle cluster, barn items).
    }
    \label{fig:thematic-images-forks}
\end{figure}

\FloatBarrier
\subsection{Hierarchical clustering}
Next, we explore what the hierarchy of the clusters looks like.
For this, we plot the hierarchy that we receive from the tangles algorithm in \autoref{fig:soft-tree-vp2}. 
As expected, we first see a coarse, thematic divide between the kitchen and the barn cluster, with 
the clean-looking pitchforks being placed together with the kitchen cluster. 
We then see a finer clustering of the kitchen cluster (node 2), which is split into a set of various kitchen items (node 5), and the cluster of special forks (node 6). 
The fact that the special forks belong to the coarse kitchen cluster and only get separated in a later step could be interpreted as the participant thinking of the 
clean-looking pitchforks as belonging more to a kitchen than a barn. This is an insight that we didn't see from the ordinal embedding alone,
highlighting a possible strength of tangles.

% contains the color scale
%TODO: Increase bar font, remove x,y font.
\begin{figure}[ht]
    \centering
    \resizebox{\textwidth}{!}{\includegraphics{figures/results/vp2_soft_tree.pdf}}
    \caption{
        The hierarchy produced by the tangles algorithm on the thematic image set for participant 2. 
        Each node represents a (soft) clustering, which we plot in its place. The points are embedded via SOE as a visualization aid.
        The color of a point corresponds to the probability that the point belongs to the given cluster, the darker, the more probable. 
    }
    \label{fig:soft-tree-vp2}
\end{figure}

\subsection{Explainability}\label{real-explain}
So far we have inspected the clusters that tangles produced. Now, we show how to use tangles to explain the clustering:
Why does a particular image belong to the kitchen or barn cluster? For this, we can look at the characterizing cuts of the clusters. As a reminder, 
these are the cuts at the splitting nodes that contribute to a meaningful decision between the left and the right subtree.

We visualize the characterizing cuts in \autoref{fig:characterising_cuts}, together with the images that induce the particular cuts. As our cuts are interpretable (a majority cut 
with anchor point $a$ contains points that are close to $a$), we can directly use this interpretation for our clustering. As we can see in \autoref{fig:characterising_cuts_a}, 
the items in the barn cluster are there because they are similar to the two straw/hay images shown in \autoref{fig:characterising_cuts_b}. 
%This is similar to our intuition, straw and hay intuitively belong in a barn setting and not in a kitchen, while pitchforks might be a bit more ambiguous.

Our interpretation is that the items that are in the cluster of other kitchen items are there because they are similar to
the dishwashers we have plotted in \autoref{fig:characterising_cuts_d}. This also makes sense, as we would interpret the dishwashers to very clearly belong in a kitchen environment, while the forks might be more ambiguous.

Overall, there is a small caveat to our explanation. In a majority cut, we only have a satisfying explanation in one direction: We know that if a point $b$ is in the majority cut that has anchor point $a$, 
then $b$ is close to $a$. However, the reverse direction might be unsatisfying: if a point $c$ is not in the majority cut, we know that it is not close to $a$. 
If we want to know how the cluster of forks and pitchforks formed, saying that they are dissimilar to the dishwashers plotted in \autoref{fig:characterising_cuts_d} is not a strong argument.
For example, we would rather know that they are similar to a certain item.  To remedy this, we can use more interpretable cuts. 
If the data would have been suitable for landmark cuts, we would have been able to make statements of the form: $a$ is in a certain cluster because it is closer to $b$ than to $c$.


\begin{figure}[ht]
    \centering
    \subfloat[Characterising cuts coarse cluster (kitchen-barn)] {
        \label{fig:characterising_cuts_a}
        \resizebox{0.333\textwidth}{!}{\import{figures/results}{node_18T_characterizing_12.pgf}}
        \resizebox{0.333\textwidth}{!}{\import{figures/results}{node_18T_characterizing_23.pgf}}
    }
    \hfill
    \subfloat[Images of data points point inducing characterizing cuts for coarse cluster] {
        \label{fig:characterising_cuts_b}
        \resizebox{0.25\textwidth}{!}{\includegraphics{figures/thematic_stimuli/straw_hay_01b.jpg}}
        \resizebox{0.25\textwidth}{!}{\includegraphics{figures/thematic_stimuli/hay_02s.jpg}}
    }
    \hfill
    \subfloat[Characterising cuts fine cluster (kitchen/special forks)]{
        \label{fig:characterising_cuts_c}
        \resizebox{0.333\textwidth}{!}{\import{figures/results}{node_19F_characterizing_0.pgf}}
        \resizebox{0.333\textwidth}{!}{\import{figures/results}{node_19F_characterizing_1.pgf}}
        \resizebox{0.333\textwidth}{!}{\import{figures/results}{node_19F_characterizing_3.pgf}}
    }
    \hfill
    \subfloat[Images of data points inducing characterizing cuts for fine cluster] {
        \label{fig:characterising_cuts_d}
        \resizebox{0.25\textwidth}{!}{\includegraphics{figures/thematic_stimuli/dishwasher_03s.jpg}}
        \resizebox{0.25\textwidth}{!}{\includegraphics{figures/thematic_stimuli/dishwasher_09s.jpg}}
        \resizebox{0.25\textwidth}{!}{\includegraphics{figures/thematic_stimuli/dishwasher_05s.jpg}}
    }
    \caption{
        Depiction of the characterizing cuts of all splitting nodes on the thematic image data set.  We draw the orientation that corresponds to the left subtree 
        (which is inverted for the right subtree). This means, that if a datapoint is often contained in the cuts that we depicted above, it is placed in the left subtree with high probability. In a), we plotted the characterizing cuts for the root node and in c) have plotted
        the characterizing cuts for the kitchen/special forks cluster. As these cuts come from our set of majority cuts, we have marked the data point that 
        induced the particular cut with a black X. Below the characterizing cuts in b) and d), we have plotted the images corresponding to the marked points.
    }
    \label{fig:characterising_cuts}
\end{figure}

\FloatBarrier
\subsection{Evaluation of another participant}
Next, we check if the results can be repeated for another participant. 
We select the odd triplets generated from the thematic image set of participant 3 and expect to see similar behavior to our evaluations for participant 2. 
In \autoref{fig:ines-embedding-vp3} we plot the embedding
from Schönmann again together with the tangles clustering. This time, we see three clusters (hay/straw, pitchforks and dishwashers/forks). These clusters coincide nicely
with our classes, aside from one fork being clustered together with the pitchforks. We note that this is not the fork from \autoref{fig:thematic-images-forks-a} that was clustered together 
with the pitchforks, so this might be a misclassification. 
If we look at the hierarchy in \autoref{fig:soft-tree-vp3}, we see that (contrary to participant 2) the pitchforks first get split off, 
and then the kitchen from the straw-hay cluster. This could indicate that participant 3 deems the straw/hay to be more similar to the kitchen items
than to the pitchforks, which could provide valuable insights for further analysis.
%TODO: Ask David
%This could again lead to interesting conclusions, which should probably be discussed by someone with more expert knowledge in the field, possibly using other evaluation data.

\begin{figure}[ht]
    \centering
    \subfloat[Original classes]{%
    \resizebox{0.5\textwidth}{!}{\input{figures/results/ines_vp3_odd_thematic_classes.pgf}}
    }
    \subfloat[Tangles clusters]{%
    \resizebox{0.5\textwidth}{!}{\input{figures/results/ines_vp3_odd_thematic_tangles.pgf}}
    }
    \hfill
    \subfloat[Embedding with images]{%
    \centering
    \resizebox{0.85\textwidth}{!}{\includegraphics{figures/results/vp3_embedding_thematic_stimuli_images.pdf}}
    }
    \caption{
        Analog to \autoref{fig:ines-embedding}. In a), we plot the original classes over an SOE embedding, in b) we plot the prediction of majority tangles with agreement 3 and radius 1 over 
        a 2-dimensional SOE embedding. In c), we see the original images at the coordinates of their respective SOE embedding.
    }
    \label{fig:ines-embedding-vp3}
\end{figure}

\begin{figure}[ht]
    \centering
    \subfloat[]{%
    \resizebox{\textwidth}{!}{\includegraphics{figures/results/vp3_soft_tree.pdf}}
    }
    \caption{
        Tangles produce similar clusters on participant 3 and participant 2. We plot the hierarchy of a tangles clustering on the triplets from participant 3 on the thematic image set. 
        On each node, we plot the soft clustering corresponding to it.
    }
    \label{fig:soft-tree-vp3}
\end{figure}
\FloatBarrier
\section{Discussion}
In this chapter, we evaluated tangles on real-world data.  
We clustered the data with majority tangles, which agreed well with the qualitative analysis that Schönmann has done on the data using more established methods. 
In addition, the more flexible clustering by tangles can show new dependencies between data points, as we 
made out a cluster of forks being perceived differently by participant 2. 

The hierarchy provided by tangles can help put these new dependencies into a better perspective, 
and allowed us to gather that the pitchforks in the special fork cluster were perceived to 
belong more in a kitchen setting than in a barn setting.
The explanations by tangles were then used to discover the more defining items of a cluster (straw/hay for
a barn, dishwashers for a kitchen). To our knowledge, hierarchical and explainable methods
have not been used on triplets in psychophysics to this date, allowing tangles to fill a potential niche.

\FloatBarrier

\cleardoublepage

\chapter{Conclusion}\label{conclusion}
\section{Where Tangles can be useful}
\section{Further developments}
\section{What this work has achieved}

\cleardoublepage

%%%%%%%%%%%%%%%%%%%%%%%%%%%%%%%%%%%%%%%%%%%%%%%%%%%%%%%%%%%%%%%%%%%%%%%%%%%%%
%%% Appendix
%%%%%%%%%%%%%%%%%%%%%%%%%%%%%%%%%%%%%%%%%%%%%%%%%%%%%%%%%%%%%%%%%%%%%%%%%%%%%
%\appendix
%
%%\setcounter{secnumdepth}{-1}
%%\section{Tables}\label{chap:App}
%\chapter{Further Tables and Figures}\label{chap:App}
%Viele Arbeiten haben einen Appendix. Besondere Sorgfalt muss beim Nummerieren der Tabellen und Abbildungen gewährleistet sein.
%\begin{table}[htb]
%\begin{tabular}{cc}
%Nummer & Datum \\
%\hline
%1 & 1.1.80\\
%2 & 1.1.90 \\
%\end{tabular}
%\caption{Erste Appendix-Tabelle}\label{tab:app1}
%\end{table}
%
%%\chapter{Figures}\label{chap:App2}
%
%\begin{table}[htb]
%\begin{tabular}{cc}
%Nummer & Datum \\
%\hline
%1 & 1.1.80\\
%2 & 1.1.90 \\
%\end{tabular}
%\caption{Zweite Appendix-Tabelle}\label{tab:app2}
%\end{table}
%%\end{appendices)
%
%\cleardoublepage

%%%%%%%%%%%%%%%%%%%%%%%%%%%%%%%%%%%%%%%%%%%%%%%%%%%%%%%%%%%%%%%%%%%%%%%%%%%%%
%%% Bibliographie
%%%%%%%%%%%%%%%%%%%%%%%%%%%%%%%%%%%%%%%%%%%%%%%%%%%%%%%%%%%%%%%%%%%%%%%%%%%%%

\addcontentsline{toc}{chapter}{Bibliography}
\bibliography{Masterthesis}
%% Obige Anweisung legt fest, dass BibTeX-Datei `mylit.bib' verwendet
%% wird. Hier koennen mehrere Dateinamen mit Kommata getrennt aufgelistet
%% werden.

\cleardoublepage
%%%%%%%%%%%%%%%%%%%%%%%%%%%%%%%%%%%%%%%%%%%%%%%%%%%%%%%%%%%%%%%%%%%%%%%%%%%%%
%%% Erklaerung
%%%%%%%%%%%%%%%%%%%%%%%%%%%%%%%%%%%%%%%%%%%%%%%%%%%%%%%%%%%%%%%%%%%%%%%%%%%%%
\thispagestyle{empty}
\section*{Selbst\"andigkeitserkl\"arung}

Hiermit versichere ich, dass ich die vorliegende Masterarbeit 
selbst\"andig und nur mit den angegebenen Hilfsmitteln angefertigt habe und dass alle Stellen, die dem Wortlaut oder dem 
Sinne nach anderen Werken entnommen sind, durch Angaben von Quellen als 
Entlehnung kenntlich gemacht worden sind. 
Diese Masterarbeit wurde in gleicher oder \"ahnlicher Form in keinem anderen 
Studiengang als Pr\"ufungsleistung vorgelegt. 

\vskip 3cm

Ort, Datum	\hfill Unterschrift \hfill 
%%%%%%%%%%%%%%%%%%%%%%%%%%%%%%%%%%%%%%%%%%%%%%%%%%%%%%%%%%%%%%%%%%%%%%%%%%%%%
%%% Ende
%%%%%%%%%%%%%%%%%%%%%%%%%%%%%%%%%%%%%%%%%%%%%%%%%%%%%%%%%%%%%%%%%%%%%%%%%%%%%

\end{document}

