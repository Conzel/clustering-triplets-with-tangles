%%%%%%%%%%%%%%%%%%%%%%%%%%%%%%%%%%%%%%%%%%%%%%%%%%%%%%%%%%%%%%%%%%%%%%%%%%%%%
%%% LaTeX-Rahmen fuer das Erstellen von Masterarbeiten
%%%%%%%%%%%%%%%%%%%%%%%%%%%%%%%%%%%%%%%%%%%%%%%%%%%%%%%%%%%%%%%%%%%%%%%%%%%%%

%%%%%%%%%%%%%%%%%%%%%%%%%%%%%%%%%%%%%%%%%%%%%%%%%%%%%%%%%%%%%%%%%%%%%%%%%%%%%
%%% allgemeine Einstellungen
%%%%%%%%%%%%%%%%%%%%%%%%%%%%%%%%%%%%%%%%%%%%%%%%%%%%%%%%%%%%%%%%%%%%%%%%%%%%%


\documentclass[twoside,12pt,a4paper]{report}
\usepackage{import}

%\usepackage{reportpage}
\usepackage{epsf}
\usepackage{graphics, graphicx}
\usepackage{latexsym}
\usepackage[margin=10pt,font=small,labelfont=bf]{caption}
\usepackage[utf8]{inputenc}
\usepackage[toc,page]{appendix}
\usepackage{bbm}
\usepackage{amsmath}
\usepackage{amssymb}
\usepackage{placeins}
\usepackage{caption}
\usepackage[subrefformat=parens,labelformat=parens,caption=false]{subfig}
\usepackage{scalefnt}
\usepackage{pgf}
\usepackage{hyperref}
\usepackage{csquotes}
\newcommand{\subfigureautorefname}{\figureautorefname}

% continuous equation numbers throughout chapter
\usepackage{chngcntr}
\counterwithout{equation}{chapter}

% bib settings
\usepackage[round,comma]{natbib}\bibliographystyle{thesis}


\newcommand{\R}{\mathbb{R}}
\newcommand{\C}{\mathbb{C}}
\newcommand{\N}{\mathbb{N}}
\newcommand{\Q}{\mathbb{Q}}
\newcommand{\Z}{\mathbb{Z}}
\newcommand{\E}{\mathbb{E}}

\textwidth 14cm
\textheight 22cm
\topmargin 0.0cm
\evensidemargin 1cm
\oddsidemargin 1cm
%\footskip 2cm
\parskip0.5explus0.1exminus0.1ex

% Kann von Student auch nach pers\"onlichem Geschmack ver\"andert werden.
\pagestyle{headings}

\sloppy


\begin{document}

%%%%%%%%%%%%%%%%%%%%%%%%%%%%%%%%%%%%%%%%%%%%%%%%%%%%%%%%%%%%%%%%%%%%%%%%%%%%
%%% hier steht die neue Titelseite 
%%%%%%%%%%%%%%%%%%%%%%%%%%%%%%%%%%%%%%%%%%%%%%%%%%%%%%%%%%%%%%%%%%%%%%%%%%%%
 
\begin{titlepage}
 \begin{center}
  {\LARGE Eberhard Karls Universit\"at T\"ubingen}\\
  {\large Mathematisch-Naturwissenschaftliche Fakult\"at \\
Wilhelm-Schickard-Institut f\"ur Informatik\\[4cm]}
  {\huge Master Thesis Computer Science\\[2cm]}
  {\Large\bf  Tangles on Ordinal Data\\[1.5cm]}
 {\large Alexander Conzelmann}\\[0.5cm]
Datum\\[4cm]
{\small\bf Reviewers}\\[0.5cm]
  \parbox{7cm}{\begin{center}{\large Prof. Dr. Ulrike von Luxburg}\\
   Theory of Machine Learning\\
  {\footnotesize Wilhelm-Schickard-Institut f\"ur Informatik\\
	Universit\"at T\"ubingen}\end{center}}\hfill\parbox{7cm}{\begin{center}
  {\large Prof. Felix Wichmann, DPhil}\\
  Neural Information Processing\\
  {\footnotesize Wilhelm-Schickard-Institut f\"ur Informatik\\
	Universit\"at T\"ubingen}\end{center}
 }
  \end{center}
\end{titlepage}

%%%%%%%%%%%%%%%%%%%%%%%%%%%%%%%%%%%%%%%%%%%%%%%%%%%%%%%%%%%%%%%%%%%%%%%%%%%%
%%% Titelr"uckseite: Bibliographische Angaben
%%%%%%%%%%%%%%%%%%%%%%%%%%%%%%%%%%%%%%%%%%%%%%%%%%%%%%%%%%%%%%%%%%%%%%%%%%%%

\thispagestyle{empty}
\vspace*{\fill}
\begin{minipage}{11.2cm}
\textbf{Conzelmann, Alexander}\\
\emph{Tangles on Ordinal Data}\\ Master Thesis Computer Science\\
Eberhard Karls Universit\"at T\"ubingen\\
Thesis period: 04/2022-10/2022
\end{minipage}
\newpage

%%%%%%%%%%%%%%%%%%%%%%%%%%%%%%%%%%%%%%%%%%%%%%%%%%%%%%%%%%%%%%%%%%%%%%%%%%%%

\pagenumbering{roman}
\setcounter{page}{1}

%%%%%%%%%%%%%%%%%%%%%%%%%%%%%%%%%%%%%%%%%%%%%%%%%%%%%%%%%%%%%%%%%%%%%%%%%%%%
%%% Seite I: Zusammenfassug, Danksagung
%%%%%%%%%%%%%%%%%%%%%%%%%%%%%%%%%%%%%%%%%%%%%%%%%%%%%%%%%%%%%%%%%%%%%%%%%%%%


\section*{Abstract}
We investigate a new algorithm for clustering triplet data (comparison-based data 
without absolute distance information). This new approach is based on the tangles framework,
a tool previously used to finde dense structures in graphs, which 
we extend to work on triplet data. Current work focusses
on embedding triplet data into a euclidean space, resulting possibly in distortions of our data,
and then following up with a classical clustering algorithm. In contrast, our proposed method
does not construct an intermediate embedding, potentially introducing less distortions
on our data and achieving a higher clustering accuracy. Additionally to being
competitive in performance, our approach provides both explainability as well as a cluster 
hierarchy on top with no added cost. We validate the tangles algorithm both on synthetic data 
under a diverse range of external influences as well as experimental data from the realm of psychophysics.

\newpage
\section*{Zusammenfassung}

Bei einer englischen Masterarbeit muss zus\"atzlich eine deutsche Zusammenfassung verfasst werden.

\newpage
\section*{Acknowledgements}

Write here your acknowledgements.

\cleardoublepage

%%%%%%%%%%%%%%%%%%%%%%%%%%%%%%%%%%%%%%%%%%%%%%%%%%%%%%%%%%%%%%%%%%%%%%%%%%%%%
%%% Table of Contents
%%%%%%%%%%%%%%%%%%%%%%%%%%%%%%%%%%%%%%%%%%%%%%%%%%%%%%%%%%%%%%%%%%%%%%%%%%%%%

\renewcommand{\baselinestretch}{1.3}
\small\normalsize

\tableofcontents

\renewcommand{\baselinestretch}{1}
\small\normalsize

\cleardoublepage


%%%%%%%%%%%%%%%%%%%%%%%%%%%%%%%%%%%%%%%%%%%%%%%%%%%%%%%%%%%%%%%%%%%%%%%%%%%%%
%%% Der Haupttext, ab hier mit arabischer Numerierung
%%% Mit \input{dateiname} werden die Datei `dateiname' eingebunden
%%%%%%%%%%%%%%%%%%%%%%%%%%%%%%%%%%%%%%%%%%%%%%%%%%%%%%%%%%%%%%%%%%%%%%%%%%%%%

\pagenumbering{arabic}
\setcounter{page}{1}

%% Introduction
\chapter{Introduction}\label{Introduction}
We consider the task of clustering data $X = \{x_1, x_2, \ldots x_n \}$ for which neither 
explicit features nor concrete distances between the data points are known to us. 
The only form of information we have on the data are so-called 
\textit{triplet comparisons} or \textit{triplets} for short. A triplet 
on $X$ is written as $(x_i, x_j, x_k)$ and tells us that $x_i$ is closer to $x_j$ than to $x_k$. 
This problem setting often arises when one works with data from human observers. 
An example, investigated among others by \cite{shepardAnalysisProximitiesMultidimensional1962}, 
is human color perception. It would be hard for humans to accurately rate colors in 
terms of features, let alone define sensible features in the first place.  It would also be hard to rate colors in terms of concrete distances to each other.
Additionally, the experiment designer would have to deal with uniting differing internal scales of different observers during data evaluation. 
A preferred approach might be to gather triplets on the data. 
To obtain these triplets, the experiment designer can repeatedly draw
three different colors, which are presented to a human, together with a 
suitable question. For example, we might draw the colors violet, red and yellow,
and ask the observer \textit{is violet more similar to red or yellow?} These questions
are comparatively easy to answer for humans. 
What remains is the question of how to evaluate the obtained triplets.

A small research community has formed around the task of dealing with triplets.
Most of this community focuses on ordinal embeddings
\citep{agarwalGeneralizedNonmetricMultidimensional2007, tamuzAdaptivelyLearningCrowd2011,
laurensvandermaatenStochasticTripletEmbedding2012,   teradaLocalOrdinalEmbedding2014, jainFiniteSamplePrediction2016, ghoshLandmarkOrdinalEmbedding2019, andertonScalingOrdinalEmbedding2019}.
An ordinal embedding is an algorithm that aims to place the data points into a euclidean space
such that the euclidean distances between the embedded points respect as many of the original triplets as possible. 
However, this approach is not always perfect: we often cannot satisfy all triplet comparisons, 
no matter the dimension of the embedding space or how the points are placed in it. 
This can for example be the case if the triplets are created using a distance function that does not obey the triangle inequality. Another possible complication can be contradicting triplets,
which often occur when gathering data from human observers. 

However, a big advantage of ordinal embeddings is exactly that we obtain a euclidean representation of the data. 
For euclidean data, there are many good, readily available algorithms for almost all tasks. 
Thus, clustering on triplets can be tackled by getting a euclidean representation of the data
from an ordinal embedding and applying a classical clustering algorithm, such as k-means \citep{lloydLeastSquaresQuantization1982}, on 
this representation. An example of this approach can be seen in \cite{kleindessnerLensDepthFunction2017}. But, as mentioned before, an ordinal embedding 
often cannot satisfy all triplet comparisons on the original data and is therefore not 
necessarily a faithful representation. 
An alternative approach is devising an algorithm to solve the desired task directly 
using the triplets, which has shown promising results.
\cite{kleindessnerLensDepthFunction2017} estimated the lens-depth function
of the data using triplets (of a slightly altered format) and used this for 
medoid estimation, outlier identification, clustering and classification. 
\cite{kleindessnerKernelFunctionsBased2017} constructed a kernel function from
the triplets, which they demonstrated to work well with 
different kernel-based clustering algorithms.
\cite{ghoshdastidarFoundationsComparisonBasedHierarchical2019} used the 
triplets to estimate a similarity function between the data points and applied 
a linkage algorithm to obtain a hierarchical clustering from the data.

In a recent paper, \cite{klepperClusteringTanglesAlgorithmic2021} proposed a novel framework
for clustering based on tangles, which are a mathematical tool originating from graph theory 
\citep{robertsonGraphMinorsObstructions1991}. The tangles algorithm has been shown to have 
interesting properties, such as inherent explainability (under certain conditions) and is suitable for hierarchical clustering.
Central objects in this framework are cuts, which are ways of dividing a set into
two distinct, non-overlapping subsets. To cluster data using the tangles framework, one first
needs to obtain a set of cuts on the data. These cuts are required to
hold a little bit of information about the cluster structure of the data. The 
tangles framework can then aggregate the information contained in these cuts to a
clustering. 

In this thesis, we present two novel methods to process triplets to cuts suitable for the
tangles algorithm. Using these methods, we 
demonstrate that the tangles algorithm can be successfully applied to (hierarchically) cluster triplets without creating an intermediate ordinal embedding. 
We evaluate our approach by simulated and experimental data and show that it is competitive 
in performance to approaches based on ordinal embeddings, while providing explainable results. 

This thesis is organized as follows: In \autoref{theory}, we give an introduction to tangles and ordinal data, which lays the necessary foundations for the rest of the work. 
In \autoref{methods}, we present our two cut finding algorithms, named \textit{landmark cuts} and \textit{majority cuts}.
We use simulated data in \autoref{simulations} to show the performance of tangles using our cuts under different circumstances, such as the noise level or availability of the triplets. 
In \autoref{real}, we choose a real-world triplet data set from the realm of psychophysics and evaluate our algorithms on this data.

\cleardoublepage


\chapter{Theoretical Background}\label{theory}
% chapter is not capitalized here, as it is not a title: https://academia.stackexchange.com/questions/9454/capitalisation-of-section-and-chapter-in-a-ph-d-thesis
In this chapter we introduce the theoretical concepts used in this work. 
In particular, we give an in-depth explanation about tangles, triplet data, and state-of-the-art methods of evaluating triplet data.
% This explanation should suffice to understand the rest of this work.

\section{Tangles}\label{theory:tangles}
% What are Tangles? How are they used? What advantages do they
% prove?
Tangles have been used in mathematical graph theory to study highly cohesive structures, 
introduced originally by \cite{robertsonGraphMinorsObstructions1991}. 
Recently, interesting areas of application have been proposed:
\cite{diestelTanglesMonaLisa2017} makes a proof of concept that tangles could 
be used in image analysis.
\cite{diestelTanglesSocialSciences2019} describes how tangles can be used in social sciences, for 
example to identify different mindsets in people answering questionnaires.
\cite{Fluck2019} shows that, under specific circumstances, tangles can be
used to reconstruct the dendrogram in a hierarchical clustering setting. 

In recent times, \cite{klepperClusteringTanglesAlgorithmic2021} describes a very flexible setup,
where tangles are successfully applied to solve problems of clustering. 
The mentioned work develops an algorithmic framework and gives theoretical guarantees for 
basic problem settings.
Additionally, it introduces simplified notations, adapted to the domain of computer science. 
When talking about tangles, we will use the definitions introduced therein. 

In this section, we give an introduction to the basic notions, theory and applications of tangles in a clustering context.
For in-depth explanations of the algorithms and exact procedures, refer to 
\cite{klepperClusteringTanglesAlgorithmic2021}.


\subsection{Definitions}
\textbf{Cuts.} The central object in tangles theory is a \textit{cut} (also referred to as a bipartition in other works). 
A cut is a division of a set $V =  \{ v_1, v_2, \ldots \}$ into two distinct subsets $A, B \subset V$, such that
\begin{align*}
A \cap B = \emptyset\;\text{and}\;A \cup B = V.
\end{align*}
We usually write a cut as $P = \left( A, \overline{A} \right)$, with $A \subset V$ and $\overline{A}$ being the
complement of $A$ with respect to $V$. As a side note, $\left( A, \overline{A} \right)$ and $\left( \overline{A}, A \right)$ are
equivalent cuts, the order of the two elements only matters when it comes to orientations (see below). \\

\noindent
\textbf{Cost function.} For a cut to be useful in clustering, we expect it to hold some degree of information about the cluster 
structure of our data. This means that an informative cut should not separate groups of data that are tightly coupled.
On a graph, an informative cut $P = \{A, \overline{A}\}$ might separate the set of nodes $V$ such that there 
are only a few edges between $A$ and $\overline{A}$. How informative a cut is, is quantified by a \textit{cost function} $c: \mathcal{P}(V) \to \R$, 
with $\mathcal{P}(V)$ denoting the power set of V. 
One needs to choose an appropriate cost function beforehand and the performance of tangles will also depend on how well the cost function and the problem setting fit together. 
For example, on an unweighted graph we might want to choose $c(P)$ as the number of edges
between the nodes of the two sets $A, \overline{A}$. \\

\noindent
\textbf{Orientations.}
An \textit{orientation} $o$ of a cut $P = \left(   A, \overline{A} \right) $ is a choice of either $A$ or $\overline{A}$. We call a cut \textit{left} oriented
if we pick $A$ and \textit{right} oriented if we pick $\overline{A}$.
An \textit{orientation} of a set of cuts 
$\mathcal{B} = \{\{A_1, \overline{A_1}\}, \ldots, \{A_n, \overline{A_n}\} \}$ is  
then a set of orientations of cuts $O = \{o_1, o_2, \ldots o_n\}$
where $o_i$ corresponds to either the partition $A_i$ or $\overline{A_i}$. \\

\noindent
\textbf{Consistency condition.} Assume that for a set of elements $V$ we have a set of cuts $\mathcal{B} = \{\{A_1, \overline{A_1}\}, \ldots, \{A_n, \overline{A_n}\} \} $ on $V$.
This set of cuts, together with the cost function, should tell us a lot about the cluster structure of the data:
for all cuts, we know how much they do or don't separate dense regions in $V$. 
This information in the cuts is aggregated and brought into a useable form by the tangles framework. 
For this, we find in $\mathcal{B}$ the set of \textit{tangles}.  
These correspond to specific ways of orienting the cuts such that they point to cohesive structures in the data.  
A tangle is an orientation for which:
\begin{align}\label{eq:consistency}
    \forall A,B,C \in O: \left| A \cap B \cap C \right| \ge a
\end{align}
for some fixed parameter $a \in \N$, which we refer to as \textit{agreement} parameter. \autoref{eq:consistency} is also called the \textit{consistency condition}. \\

\noindent
\textbf{Order.} Using this definition of tangles, a lot of sets of cuts wouldn't allow for any tangles, as there are too many cuts to consistently orient them. 
Imagine that a set of cuts would contain a few random cuts. 
In expectation, each of these halve our set of points, so we can at most orient on the order of $O(\log(n))$ many random cuts consistently.
This is resolved using the cost function: one restricts the tangles to a set of low-cost (and thus very insightful) cuts $P_{\psi}$, using a threshhold $\psi \in \R$ such that
\begin{align*}
P_{\psi} = \{ c(P) \le \psi \}   
.\end{align*}
A tangle on $P_{\psi}$ is said to have \textit{order} $\psi$. \\

\noindent
To illustrate some of the concepts better, we include a schematic drawing of a tangle with agreement $3$ on 
a simple data set composed of two clusters in \autoref{fig:tangles-example}. 
Here, we might already gain some intuition on why tangles are able to find dense structures in data. 
The tangle that is depicted in the figure orients all cuts left 
(indicated by the arrows), so that they point towards the cluster on the left. 
Another possible tangle might orient all cuts to the right, pointing to the right cluster. 
Notice that a tangle on this set of cuts can only either orient all cuts to the left or to the 
right, else the consistency criterion is violated, as else the intersection of the orientations of the cuts contains at most one point. 
All in all, 
there is exactly one tangle for each cluster.

\begin{figure}[h]
    \centering
    \includegraphics[width=0.8\textwidth]{figures/tangles-example.pdf}
    \caption{A simple tangle for a reasonably sized agreement ($a = 3$). 
        The data set consists of two clusters, one left (blue), one right (orange).
        We assume that we have obtained three cuts on the data set, represented 
        by red lines.
        We draw a possible orientation on the cuts, indicated by the red arrows on them. 
        The entirety of all orientations makes up one possible tangle for the cuts 
        on this data set. 
    }
    \label{fig:tangles-example}
\end{figure}

\subsection{Processing tangles to a clustering}
In \autoref{fig:tangles-example}, each of the tangles we found clearly pointed in the direction 
of exactly one cluster. However, tangles on real data sets
are usually much more complex. In this section, we explain an algorithmic procedure how clusters 
can be identified through tangles. We assume that we have chosen an appropriate
cost function $c$ and an agreement parameter $a$. 

We are given a tuple of cuts $\mathcal{B} = \left(  \left( A_1, \overline{A_1} \right) , \ldots, 
\left( A_n, \overline{A_n} \right)\right) $ which are sorted in  
ascending order according to the cost function $c$.  Next, we build a tree structure
on these cuts, the so-called \textit{tangle search tree}. 
In the tree, the value of each node of level $i$ is an orientation on the set of cuts $\mathcal{B}_{1:i} = \{\{A_1, \overline{A_1}\}, \ldots \{A_i, \overline{A_i}\}\}$, with the value of 
$n_0 = \emptyset$.
We build the tree in an iterative manner: to determine the nodes of level $k$, we 
iterate through all nodes $n_j$ of level $k-1$. 
We then try to add the cut $\{A_k, \overline{A_k}\}$ in left orientation to $n_j$. If this
orientation is consistent with respect to $a$, we add the orientation $n_j \cup A_k$ as a left child. 
We then try to add the cut in right orientation as well, and append $n_j \cup \overline{A_k}$ as right child if it is consistent.
By construction, each node in the tangle search tree then represents a tangle, and 
every level of the tree contains all possible tangles of threshhold $\le \psi_k$ which directly corresponds to the cost of the $k$-th cut $c(\{A_k, \overline{A_k}\})$. 

An exemplary tangle search tree is illustrated in \autoref{fig:tangles-tree-example}. 
By the construction above, we can now determine the value of each node. 
As an example for the node in level 3, we start with $\emptyset$ at the root node. To
get to the node, we have to go right from the root, adding cut $P_1$ in a right-oriented
way. We then go right again for $P_2$, and left for $P_3$. Thus, we know that the node (and the 
corresponding tangle) in level 3 has the value $T = \{\overline{A_1}, \overline{A_2}, A_3\}$. 
By the construction of the tangle tree, we also know that this tangle is now the only one on
on the set $\{P_1, P_2, P_3\}$; however, there exist 3 tangles on $\{P_1, P_2\}$.


\begin{figure}[h]
    \centering
    \includegraphics[width=0.8\textwidth]{figures/tangles-tree-example.png}
    \caption{A possible tangles search tree for a set of cuts $\mathcal{B} = \{\{A_1, \overline{A_1}\}, \{A_2, \overline{A_2}\}, \{A_3, \overline{A_3}\} \}$. 
        Each level corresponds to the tangles of the order given by the cut $P_i$ that is written on the dashed line to the left of it.
        Figure taken with permission from \cite{klepperClusteringTanglesAlgorithmic2021}.}
    \label{fig:tangles-tree-example}
\end{figure}

We now discuss how to use the tangle search tree for clustering. 
First, observe that a cheaper cut cuts through more loosely connected structures
of the data set, while a more expensive cut can cut through more densely connected structures.
Thus, in the tangles search tree we will start with coarse divisions of our data
at the root, and proceed to finer divisions as we go down the tree. With respect to clustering, the interesting nodes in the tangle tree are the \textit{leaves} and
the \textit{splitting nodes} (nodes with two children). \\
\textbf{Leaves} are the final clusters, as we cannot add more cuts to the tangles and thus cannot subdivide the structure the tangle points to further. In the end, 
there will each leaf node will correspond to exactly one cluster and vice versa.\\
\textbf{Splitting nodes} represent meaningful paths in our tree. If a node only has one child,
the cut that was added last could only be added in one orientation. This cut might make it harder to add more cuts further down the tree, but it does not present
a meaningful division of our cluster. If a node has two children, 
the cut however presents a decision, as we can either follow the left or the right
child of the cut when deciding how to subdivide our cluster further. 

For the splitting nodes, we are interested in which cuts are really the ones determining 
which cluster a point belongs to. An initial idea would be to only cut that produced the split,
and assign all points that are contained in the left orientation of the cut to the
left child and all that are contained in the right orientation to the right child. This
however seems to waste information contained in the cuts further down the tree. 
Thus, one determines the set of \textit{characterizing cuts}.  
A cut belongs to this set, if it is both oriented the same way inside the subtrees, and oriented in a different way between the two subtrees. Thus, all characterizing cuts make a meaningful decision between left and right subtree and are coherent in their decision inside the subtree. 
To illustrate this, we take a look at the exemplary tree in \autoref{fig:tangles-tree-example}. Here, for the root node, $P_1$ is a characterizing cut,
while $P_2$ is not: below the node $A_1$, the cut is both oriented to the left and to to the right, violating the requirement that the cuts are always oriented
the same way inside the subtrees. 

With this knowledge, we can now obtain a soft, hierarchical clustering from the tangles search tree. 
\textit{Soft} means that every data point is assigned
a probability of belonging to each cluster. 
\textit{Hierarchical} means that we have a hierarchy on the resulting clusters, which arises naturally
from the form of the tangles search tree. 
For a soft clustering, we determine the probability of a point $x$ belonging to a cluster $C$. 
To do this, we start from the root. At every splitting node of the tree (which includes the root), we examine the set of characterising cuts.
We then orient them in the same way they are oriented in the left subtree and count how many of these so oriented cuts contain $x$. Divided by the total amount of characterising cuts at the
splitting node, we receive the probability $p_L$ that $x$ belongs to the left cluster. As the characterising cuts are always oriented differently in the two subtrees, the probability
of $a$ belonging to the right cluster is then given by $1 - p_L$. We can include these probabilites on the edges of our tree. 
To find out with what probability $x$ belongs to $C$, 
we now take the product of all edge probabilities on the path from the root to the leaf node that corresponds to $C$. 
By assigning each node to the cluster it belongs to with the highest probability, we can also obtain the corresponding \textit{hard} clustering.

\section{Ordinal constraints and triplet data}
Assume that we have a set of objects for which we don't know absolute distance information
between them. 
A dataset of ordinal constraints is then a set of comparisons on these
objects such as \textit{item $i$ is closer to item $j$ than to item $k$}. 
Formally, we assume that $i, j, k$ are from 
a set $D$ where we can define a dissimilarity function $d: D \times D \to \R$. Note that $d$ can be a proper metric on $D$, but does not have to be.
Using $d$, we can express the constraint \textit{item $i$ is closer to item $j$ than to item $k$} 
as $d(i, j) < d(i, k)$. Such data is often encountered when humans are asked to judge objects, 
as they naturally are better at comparing objects
than at accurately placing them on an abstract scale \citep{demiralpLearningPerceptualKernels2014}. 
Applications consist of estimating perceptual scales in psychophysics 
\citep{haghiriEstimationPerceptualScales2020} or crowd-sourcing clustering algorithms \citep{ukkonenCrowdsourcedCorrelationClustering2017}. 
We focus mainly on the realm of psychophysics.

Ordinal constraints are usually presented in one of two forms: quadruplets (used for example in \cite{ghoshdastidarFoundationsComparisonBasedHierarchical2019}) and triplets (used for example in \cite{vankadaraInsightsOrdinalEmbedding2021,haghiriComparisonBasedFrameworkPsychophysics2019}).
Let $D$ be a set of objects, and $a,b,c,d \in D$, and $d$ be a dissimilarity function on $D$.
A quadruplet (a,b,c,d) expresses the following constraint on our data points: 
\begin{align*}
\text{d}(a,b) < \text{d}(c,d)
.\end{align*}
Analogously, a triplet (a,b,c) expresses 
\begin{align*}
\text{d}(a,b) < \text{d}(a,c) 
.\end{align*}
A triplet (a,b,c) can also be expressed by the quadruplet (a,b,a,c), making quadruplets strictly more general. 

Datasets of triplets (which we also simply call triplet data) are almost always obtained by asking humans participants. For example, we might 
collect triplet data on images by presenting human participants with three images $a,b,c$, 
ask them: \textit{is $a$ more similar to $b$ or $c$?}.
The way that this question is formulated varies on the context of the experiment (and might also influence their answers). 
Other possible experiment setups are for example presenting the participant with three images, and asking \textit{which is the most central image?}, or \textit{which is the odd one out?},
but the results can then always transformed back to triplet format for further prcoessing. 
For example, if $a$ is the \textit{odd-one-out} of the three elements $a,b,c$, 
then we know that $d(b,c) < d(b,a)$ and $d(c, b) < d(c,a)$. 

\section{Algorithms on triplet data}
Most of the algorithms that the machine learning community uses require feature-based data 
($k$-Means, support vector machines, neural networks, et cetera...).
Triplet data is an unusual format, which is why one of the most common evaluation
methods is to first use an ordinal embedding algorithm on the triplet data to transform
it into euclidean space, before processing it further.  
We therefore divide the algorithmic approaches presented in this section into two parts, 
ordinal embeddings, and other algorithms, which achieve
end tasks directly without applying an ordinal embedding beforehand. 
The latter category is where the tangles algorithm belongs to. Not using an ordinal embedding
as a first processing step can have distinct advantages: 
we introduce additional distortions to the data and we avoid eventual biases that the ordinal embedding algorithms might have.

\subsection{Ordinal embeddings}
One of the most central problems when dealing with triplet data consists of finding a so called \textit{ordinal embedding} of the data. If we have a set of triplet comparisons $T = \{t_1, t_2, \ldots t_n\}$, 
of the form $t_i = \left( a,b,c \right)$, encoding that $d(a,b) < d(a,c)$, 
we want to find a set of points $y_1, y_2, \ldots y_n \in \R^{m}$, such that they uphold most of the original triplet constraints in $\R^{m}$ with the euclidean distance
as metric. Formally, we want to minimize \citep{vankadaraInsightsOrdinalEmbedding2021}
\begin{align*}
    \min_{y_1, \ldots y_n \in \R^{m}} \sum_{t=\left( i,j,k \right)  \in T} \mathbbm{1}_{ \|y_i - y_j\|_2 < \|y_i - y_k\|_2 }
.\end{align*}
This problem is difficult to optimize and thus various algorithms have been proposed that solve a relaxed or modified version of this objective function. The algorithms are mostly formulated
for quadruplets, as this is more general: we can convert triplets to quadruplets but not 
necessarily vice versa.

\begin{itemize}
    \item Soft Ordinal Embedding  \citep[SOE,][]{teradaLocalOrdinalEmbedding2014} 
        introduces a scale parameter $\delta$ and not only punishes violated ordinal constraints
        with a binary value, but also by how much they are violated using the
        actual distance between embedded points.
        The authors consider a set of quadruplets $Q$ on a data set. 
        They propose the following error function, which their algorithm minimizes:
        \begin{align*}
            \text{Err}_{\text{soft}}(X  \mid  \delta) = \sum_{i<j} \sum_{k<l} o_{i,j,k,l} 
            \max [0, d_{ij}(X) + \delta - d_{kl}(X)]
        ,\end{align*}
        where $X$ is the embedding of the points, $d_{ij}(X)$ is the euclidean distance 
        of points with index $i$ and $j$ in $X$, $\delta$ is a scale parameter and $o_{i,j,k,l}$
        is $1$ if $i$ is closer to $j$ than $k$ to $l$ according to $Q$, and $0$ else.

    \item Generalized Non-Metric Multidimensional Scaling \citep[GNDMS,][]{agarwalGeneralizedNonmetricMultidimensional2007} finds a gram matrix of an embedding. They cast
        the constraints given by a set of quadruplets $Q$ as constraints on the gram matrix 
        and use these as inequalities in a constrained optimization problem. The dimension
        of the embedding is controlled via a regularization parameter $\lambda$ on
        the trace of the embedding, which functions as a relaxation of the rank. 
        They end up with the following optimization problem:
        \begin{align*}
            &\min_{K, \xi_{ijkl}}          &       & \sum_{(i,j,k,l) \in Q} \xi_{ijkl} + \lambda 
            \text{tr}(K),\\
            &\text{subject to} &       & k_{kk} - 2k_{kl} + k_{ll} - k_{ii} + 2k_{ij} - k_{jj}
            \ge 1 - \xi_{ijkl}\\
            &                  &       & \sum_{ab} k_{ab} = 0, K \succeq 0,
        \end{align*}
        where $K$ is the gram matrix of the embedding, $\xi_{ijkl}$ are the slack variables
        for the constraints on $K$, and $K \succeq 0$ indicates that $K$ must be
        positive semidefinite. The actual embedding $X$ can be recovered from the gram matrix by
        spectral decomposition. 
    \item t-Stochatic Triplet Embedding \citep[t-STE,][]{laurensvandermaatenStochasticTripletEmbedding2012} uses an approach similar to the well-known t-stochastic neighbour embedding 
        \citep[t-SNE,][]{maatenVisualizingDataUsing2008}. 
        The authors measure the similarities between points in their embedding using a Student-t kernel 
        with $\alpha$ degrees of freedom: 
        \begin{align*}
            p_ijl = \frac
            { \left( 1+ \frac{  \|x_i - x_j\|^2 }{\alpha}  \right)^{- \frac{\alpha + 1}{2}}  
            }
            { \left( 1+ \frac{  \|x_i - x_j\|^2 }{\alpha}  \right)^{- \frac{\alpha + 1}{2}} +
              \left( 1+ \frac{  \|x_i - x_k\|^2 }{\alpha}  \right)^{- \frac{\alpha + 1}{2}}  
            }
        .\end{align*}
        They then maximize the sum of the log probabilities over all triplets $T$:
        \begin{align*}
            \max_{X} \sum_{(i,j,k) \in T} \log p_{ijk}
        \end{align*}
        using gradient descent.
        This formulation of the objective not only ensures that the 
        triplet constraints are satisifed: if $(i,j,k) \in T$, the algorithm 
        also decreases the distance between $x_i$ and $x_j$, and increases the distance
        between $x_i$ and $x_k$. 
\end{itemize}
Other approaches include for example 
Crowd Kernel Learning \citep[CKL,][]{tamuzAdaptivelyLearningCrowd2011}, Fast Ordinal Triplets Embedding \citep[FORTE,][]{jainFiniteSamplePrediction2016} and various others. 

It has been proven that if the original points come from the space $\R^{m}$, one can recover the points (up to scaling and orthogonal transformations) with $O(mn\log(n))$ many triplet comparisons
\citep{jainFiniteSamplePrediction2016}. On this basis, an ordinal embedding can be used together with more classical machine learning algorithms, such as support vector machines or k-Means,
for other machine learning tasks. This approach has for example been demonstrated for classification \citep{tamuzAdaptivelyLearningCrowd2011, kleindessnerLensDepthFunction2017} or clustering \citep{kleindessnerLensDepthFunction2017}.

\subsection{Other algorithmic approaches}
Other algorithmic approaches rely on extracting information from the triplet data directly.
These algorithms are hand-crafted for the desired target tasks, such as classification, clustering, et cetera... We collect some example algorithms and briefly describe their approaches.

\begin{itemize}
    \item \cite{kleindessnerLensDepthFunction2017} uses lenses and the lens-depth function, 
        introduced by \cite{liuLensDataDepth2011}. Assume we have a set of points $D$ equipped
        with a dissimilarity $d: X \times X \to \R$. 
        For two points $x_i, x_j \in D$, their lens is the intersection of two
        spheres with radius $d(x_i, x_j)$ centered at $x_i$ and $x_j$. 
        Thus, the lens of two close points has a smaller volume than the lense of
        two far away points. Using a specialised form of triplets that
        indicate the most central object out of $i,j,k$, the lens can be used to 
        estimate the proximity of two data points. If, for two points $x_i, x_j$ there are 
        a lot of triplet statements that contain $x_i, x_j$ and another object as the central 
        object, this indicates that their lens must be larger. 

        For clustering, the authors 
        build a $k$-relative neighbourhood graph on $D$ by connecting two points $x_i$ and $x_j$
        if and only if
        \begin{align*}
            V(x_i, x_j) = \frac{N(x_i, x_j)}{M(x_i, x_j)}  < \frac{k}{\left| D \right| - 2}
        .\end{align*}
        with $N(x_i, x_j)$ being the number of statements that contain both $x_i$ and $x_j$ and
        have another object $x_k$ as the central object, and $M(x_i, x_j)$ being
        the total number of statements that contain both $x_i$ and $x_j$.  
        The obtained $k$-relative neighbourhood graph is then used for clustering 
        together with spectral clustering.

        The authors use the insights obtained into lenses together with previous work on 
        lens-depth function to extend the approach also to classification, 
        medoid (most central object) estimation and outlier detection.
         
        % The lens-depth function $LD(x; D)$ of a point $x$ is then the number of pairs of points 
        % whose lens contains $x$:
        % \begin{align*}
        %     LD(x;D) =  \left|   \{(x_i, x_j): x_i, x_j \in D, i < j, x \in \text{Lens}(x_i, x_j)  \}   \right|
        % .\end{align*}

        % They rely on specialised form of triplets that indicate the most central object out 
        % of $i,j,k$ which can be direcctly used to estimate the lens-depth function as:

        % \begin{align*}
        %     LD(x; D) \approx \frac{\text{\# of statements in } S \text{ with } x \text{ as most central point}  }{ \text{\# of statements that contain } x}
        % .\end{align*}

        % with $S$ being the set of triplet statements that indicate the most central object.

        %The lens depth function can for example be used to obtain a $k$-nearest neighbour 
        %graph. As the lens close-by points will have

    \item %TODO 
        \cite{kleindessnerKernelFunctionsBased2017} uses the triplets directly to estimate a kernel function between the points in the dataset. 
        They present two different kernel functions. In both approaches, they start by determining a similarity between all pairs of points $x$ and $y$ out of the dataset $X$.
        For the first kernel function $k_1$, they rank all other points $x \in X$ by their closeness to $x$, and repeat the same for $y$. Then, they take 
        the Kendall tau correlation coefficient (which is a kernel function on the set of total rankings) between the two rankings and use this to generate a feature
        map that serves as the value of the kernel function $k_1(x,y)$. 

        For the second kernel function $k_2$, they determine how similar two points $x$ and $y$ are by counting the number of triplets that they agree on. 
        They do this by determining all pairs of points $x_i, x_j$ for which both $(x_i, x_a, x_j)$ and $(x_i, x_b, x_j)$ hold. From the number of these pairs, 
        they subtract the number of all pairs for which either $(x_i, x_j, x_a)$ and $(x_i, x_b, x_j)$ or $(x_i, x_a, x_j)$ and $(x_i, x_j, x_b)$. This similarity is used in 
        a feature map from which they construct $k_2(x,y)$.

        The kernel functions can then be used in any kernel machine, such as kernel SVM for classification or regression. 

    \item %TODO 
        \cite{kleindessnerKernelFunctionsBased2017} presents methods to hierarchically cluster data from a set of quadruplets $Q$. They argue that single-linkage and complete-linkage can be
        naturally implemented for a quadruplet setting, as they only require us to know which objects are the closest together or farthest away from each other, without requiring
        absolute distance values. As single-linkage and complete-linkage have weak statistical guarantees, they present two methods to implement average linkage for quadruplet data. 
        For the first one, they use a kernel similar to \cite{kleindessnerKernelFunctionsBased2017} to estimate similarities between the objects using the quadruplets $Q$, and then use standard
        linkage procedure (iteratively merge the two most similar clusters). For the second one, they estimate whether two clusters $G_1, G_2$ are more similar to each other than two clusters $G_3, G_4$ directly using the quadruplets. This is done
        by counting the number of quadruplets $(i,j,k,l) \in Q$ and subtracting the number of quadruplets $(k, l, i, j)$ for which $x_i \in G_1, x_j \in G_2, x_k \in G_3, X_l \in G_4$. 
        The similarity between clusters can then again be used in a standard linkage procedure.

\end{itemize}
% Kleindessner Lenses
% Kleindessner Kernel
% Ghoshdastidar Comparison Based Hierarchical Clustering
% Ukkonen? Search for some more


\cleardoublepage

\chapter{Clustering Triplet Data with Tangles}\label{methods}
As we described in \autoref{theory:tangles}, the tangles algorithm operates on cuts of data that contain some information about the cluster structure.
If we want to cluster a set of triplet data using tangles, we are first faced with the task of processing the triplets to appropriate cuts. In this work, we present two 
methods for this which we call \textit{landmark cuts} and \textit{majority cuts}. We elaborate on the methods and their motivations in the following sections.

\section{Landmark cuts}\label{theory:landmark_cuts}
In recent years, algorithms have been developed that hope to speed up ordinal embedding by focussing on so-called \textit{landmarks} 
\citep{ghoshLandmarkOrdinalEmbedding2019, andertonScalingOrdinalEmbedding2019}. Landmarks are objects in the dataset for which we know all triplet comparisons.  
The definition of what constitutes a landmark varies in the literature, but we use the one by \cite{haghiriComparisonBasedFrameworkPsychophysics2019}.
Assume we have a set of objects $D$, as well as a set of triplet constraints $T$. The triplets 
have the form $(a,b,c)$, indicating that $d(a,b) < d(a,c)$. In a landmark setting, we have a set 
of $m$ objects $L \subset D$, for which 
\begin{align*}
\forall l_i, l_j \in L \; \forall x \in D: (x, l_i, l_j) \in T \vee (x, l_j, l_i) \in T. 
\end{align*}
If we have landmarks, they make it very easy to define a set of cuts on triplet data:
for a combination of landmarks $l_i, l_j$, we can make a cut $P = \{A, \overline{A}\}$ by assigning all points closer to $l_i$ to $A$ and all those closer to $l_j$ to $\overline{A}$. 
Formally, for two landmark points $l_i, l_j$, we define the set
\begin{align*}
    \text{Land} _{ij} = \{ x \in D \mid \left( x, l_i, l_j \right) \in T \}
,\end{align*}
and call the corresponding cut $P_{ij} = \{\text{Land}_{ij}, \overline{\text{Land}_{ij}}\}$ a \textit{landmark cut}.

For the tangles algorithm, we can relax the requirement on the landmarks, as we do not actually
need to know the triplet comparisons between every possible combination of landmarks. Rather, 
we require that there exists a set of tuples $L' = \{ (y, z)  \mid y, z \in D \}$ for which:

\begin{align*}
    \forall (x,y) \in L' \; \forall x \in D: (x, y, z) \in T \vee (x, z, y) \in T.
.\end{align*}
This will be used in the simulations, as then we can create landmark cuts for tangles by 
repeatedly picking some objects $y, z \in D$ and sampling all triplet comparisons to all other objects $x \in D$. \\

Landmark cuts intuitively capture some cluster information: 
the closer $x$ is to $l_i$ according to $d$, the more likely it is that $\text{Land}_{ij}$ contains $x$. 
In the euclidean space, this notion is easily captured: A landmark cut between $l_i, l_j$ is
a linear cut between the two points, as illustrated in \autoref{fig:landmark_cut}.

    \begin{figure}[ht]
        \centering
        \includegraphics[width=0.8\textwidth]{figures/landmark_cut.pdf}
        \caption{A landmark cut on a euclidean data set. The two diamonds 
            are the landmarks $l_1$ (red) $l_2$ (blue). All red points (left of the line) 
            are contained in $\text{Land}_{1,2}$, all blue points in $\overline{\text{Land} _{1,2}}$.}
        \label{fig:landmark_cut}
    \end{figure}

The landmark approach is an unusual way of sampling triplets.
In most experiments involving triplets, the triplets are either sampled uniformly at random
\citep{kleindessnerLensDepthFunction2017, haghiriEstimationPerceptualScales2020} 
or according to a chosen metric, for example, maximizing a measure of gained information \citep{roadsEnrichingImageNetHuman2021}. 
There is no experimental dataset available that is both sampled according to a landmark approach and exhibits a cluster structure. 
We thus rely on simulations for testing landmark cuts. These simulations can be found in \autoref{simulations}.

To make landmark tangles work with a set of triplets that are sampled uniformly at random, one can impute the missing triplets with standard methods.
An example would be random imputation. There, one would randomly either add $(a,b,c)$ or $(a,c,b)$ to the set of triplets, if none of them were present.
We will explore this more in-depth in \autoref{simulations}.

\section{Majority cuts}\label{theory:majority_cuts}
As explained in \autoref{theory:landmark_cuts}, sampling triplet data in a landmark-fashion is not very widely used in current practice. Due to this, we 
also present a more general approach to processing triplets to cuts that can be applied to any set of triplets $T$ regardless of the sampling method.
%For this, we again use the intuition that triplets tell us something about the closeness of data points, and thus about the cluster structure. 

As a motivation, we first think about cuts that are well suited for clustering with tangles.
We assume that the objects are clustered based on their similarities to each other so that 
similar objects also belong to the same cluster. Based on this, a cut $P = \left(   A, \overline{A} \right) $ is more informative for clustering if the objects in $A$ are similar to each other and
dissimilar to the objects in $\overline{A}$. If we fix a point $a$, one way
to generate such a cut is as follows: 
we center a ball of radius $r$ on $a$ and add all objects contained by the ball to $A$, 
and all others to $\overline{A}$. This is not perfect, but we know that the objects
contained in $A$ have at least some level of closeness (they all have 
a maximum distance of $2r$ from each other). 

Next, assume that we have a set of $n$ objects $D$ and a set of triplets $T$ on the
objects. We now develop a method of approximating a ball around a point $a$ using the
triplet information given. Let 
\begin{align*}
L_x = \{t \in T  \mid  t = (x, b,c), b,c \in D\} 
.\end{align*}
be the set of all available triplets where $x$ is in the left position. Equivalently, 
we define 
\begin{align*}
    M_x &=  \{t \in T  \mid  t = (a, x, c), a,c \in D\} \\
    R_x &=  \{t \in T  \mid  t = (a, b, x), a,b \in D\} 
,\end{align*}
as the sets of triplets where $x$ is in the middle and right position. 
Using these sets, we define  
\begin{align*}
    \text{Maj} _a := \{x \in D \mid \left| L_a \cap M_x\right| < \left| L_a \cap R_x \right| \}
,\end{align*}
as the set of all points that are more often closer to $a$ than they are farther away. 
We can then define the corresponding cut $P_a := \{ \text{Maj}_a, \overline{\text{Maj}_a } \}$, which
we refer to as a \textit{majority cut} with \textit{anchor point} $a$.

Assuming we have all triplets, meaning that 
\begin{align*}
    \forall a,b,c \in D: (a,b,c) \in T \vee (a,c,b) \in T
.\end{align*}
then $\text{Maj}_a$ is a ball around $a$ whose radius is the median distance of $a$ to all other points $x \in D$, as only those points will appear more often closer to $a$ in the triplets
than they appear farther away. When $T$ does not contain all possible triplets, we introduce 
noise on our cuts, and $\text{Maj}_a$ then contains some points outside of a median-sized ball
around $a$ as well as it does not contain some points inside the ball.

Majority cuts can be made more flexible by including a ratio $r$ that controls the size of the cuts. 
We then define 
\begin{align*}
\text{Maj}_a(r) := \{x \in D \mid \left|   L_a \cap M_x\right| < r \cdot \left| L_a \cap R_x \right| \}
.\end{align*}
and call $r$ the \textit{radius} of the cut. 
$\text{Maj} _a(r)$ is then a ball around $a$ that contains the $n \cdot \frac{r}{r+1}$ points that are closest to $a$ . In a euclidean setting for 
$r = 0.5$ we thus expect $P_a(0.5)$ to be a ball around $a$ that contains the $\frac{n}{3}$ points that are closest to $a$. To visualize this, we plot a majority cut with a fixed anchor point on a 
mixture of gaussians in \autoref{fig:majority_cut}. 
As the number of triplets increases, we get closer to a true ball around $a$ containing the $\frac{n}{3}$ closest points. 
When we have fewer than all triplets available, the ball around $a$ becomes corrupted by noise.

What remains is the question of how to choose the radius. We can imagine that if we pick a smaller radius, we will detect smaller clusters, and if we pick a larger radius, we will detect larger
clusters. 
%TODO: Investigate this claim?
In particular, we should not pick a radius smaller than the smallest cluster we want to detect, else there might exist no tangles on the cuts. 
On the contrary, we are safe if we pick a radius that is a bit larger, as the tangles algorithm shows good performance on cuts that contain a cluster together with some additional data points.

\onecolumn
\begin{figure}[ht]
    \centering
    \subfloat[500 (0.47\%) triplets]{%
      \resizebox{0.5\textwidth}{!}{\input{figures/results/majority-cut-8-n_triplets-500.pgf}}
  }
    \subfloat[5000 (4.7\%) triplets]{%
      \resizebox{0.5\textwidth}{!}{\input{figures/results/majority-cut-8-n_triplets-5000.pgf}}}
    \hfill
    \subfloat[20000 (18.8\%) triplets]{%
      \resizebox{0.5\textwidth}{!}{\input{figures/results/majority-cut-8-n_triplets-20000.pgf}}}
    \subfloat[106200 (100\%) triplets]{%
      \resizebox{0.5\textwidth}{!}{\input{figures/results/majority-cut-8-n_triplets-106200.pgf}}}
    \caption{Majority cuts more closely resemble a ball around their anchor point as the number of 
        sampled triplets increases. 
        We plot $\text{Maj}_a$ with radius $r=0.5$, generated according to the procedure in \autoref{theory:majority_cuts}. The black X marks the anchor point $a$. 
        The set $\text{Maj}_a$ consists of the orange triangles, which are the points that are twice as often closer to $a$ than they are not (according to the drawn triplets). 
        The blue points are those not in $P_a$.}
    \label{fig:majority_cut}
\end{figure}

\cleardoublepage

%% 
\chapter{Simulations}\label{simulations}
In this chapter and the following one, we want to explore how tangles perform on triplet data with the methods we proposed in \autoref{methods}. 
At first, we focus on simulations, as this allows us to control our data precisely. 
We will show in which cases Tangles perform well, in which ones they don't and what to keep in mind when applying the algorithm. \\

As datasets, we have decided on a mixture of gaussians and a hierarchical setup. In both cases, we will then generate triplets from the data points.
The gaussian setup serves as first baseline: it is a de-facto standard in clustering and a lot of real-life data is approximately gaussian. 
We have decided against using more complex data sets such as two-moons, as the focus lies on how the algorithms
interact with the triplets generated on the data and deal with additional complexities in that domain (such as triplets being corrupted, triplets missing, et cetera...). \\

The hierarchical setup takes the form of a noisy hierarchical block matrix, introduced by \cite{balakrishnanNoiseThresholdsSpectral2011}.
We use it to illustrate a second property of Tangles: in addition to pure clustering, we also produce a hierarchical tree, 
which can be used for hierarchical clustering. 

\section{Terms and methods used}
To evaluate the performance of the Tangles algorithm, we will need some metrics and other methods to use as a baseline. To compare a clustering against a ground truth,
we will use the \textit{normalized mutual information score} (NMI), which is independent of the cluster labels. 
$1.0$ indicates the same clustering (up to label permutations), $0.0$ indicates absolutely no mutual information between the clusterings (such as when our prediction
puts all data points in a single cluster). To compare the quality of different hierarchies, we make use of the \textit{average adjusted rand index} (AARI), 
introduced by \cite{ghoshdastidarFoundationsComparisonBasedHierarchical2019}. The AARI extends the \textit{adjusted rand index} (ARI), which is a clustering performance measurement
similar to the NMI, to compare hierarchies. In AARI, we calculate the ARI over all levels of the hierarchies we want to compare and average over all the obtained scores. 
% A further discussion on how we use the AARI for Tangles can be found in \label{sec:hierarchical_data}.


To generate the triplets, we first take the data and calculate a (dis)similarity on it, for example the euclidean distance between two data points. This allows
us to determine whether $(a,b,c)$ or $(a,c,b)$.
Then, we can use two approaches of drawing triplets. The first one is sampling triplets randomly and uniformly from the set of all triplets. The second one is a landmark approach:
we fix two randomly sampled points $a,b$, with $a \neq b$, sample all triplet that have the form $(x,a,b)$ (or equivalently $(x,b,a)$.  For the Landmark Tangles, only the 
second kind of triplets are useable, as discussed in \autoref{methods}, thus we will mostly stick to this format. 
We now have to differ between two possible ways of altering the triplets. We can add \textit{noise}
to the triplets. If we have a noise level of $p$, then every triplet is flipped with probability $p$, $(a,b,c)$ would be turned into $(a,c,b)$. We can also reduce
the \textit{density}, meaning that instead of sampling all triplets, we sample only a fraction $d$ of all triplets. In this section, when we use the term density, we
will refer to triplets sampled in a landmark approach, while when explicitly using the number of drawn triplets, we refer to a uniform sampling.

As a baseline, we will be using a combination of ordinal embedding together with a clustering algorithm. 
We first use an ordinal embedding algorithm to get an embedding of the triplet data. This approach has also been used in \cite{kleindessnerLensDepthFunction2017}. 
It has the advantage of also working with triplet data as opposed to clustering on the data directly, giving a more fair baseline, while still being very straightforward.
Afterwards, we use a standard clustering algorithm for euclidean data on the obtained embedding.  
There exist numerous algorithms for ordinal embedding (see \cite{vankadaraInsightsOrdinalEmbedding2021} for an overview) and for clustering.

For the ordinal embeddings, we have used Soft Ordinal Embedding (SOE \cite{teradaLocalOrdinalEmbedding2014}), as this has been identified by \cite{vankadaraInsightsOrdinalEmbedding2021} as one of the 
top-performing ordinal embedding algorithm for a variety of use cases. When testing out different baselines, we have also found SOE to consistently have top performance among 
all tested ordinal embedding algorithms. Additionally, we have included T-Stochatic Triplet Embedding (T-STE \cite{laurensvandermaatenStochasticTripletEmbedding2012}) 
to have a second baseline. For the clustering, have decided to go with k-Means, as one of the most basic clustering algorithms with usually good performance.

Additionally, we have included ComparisonHC \citep{ghoshdastidarFoundationsComparisonBasedHierarchical2019} as another baseline. This is another algorithm that doesn't
construct an embedding before clustering, and thus might allow for a more fair comparison to the tangles algorithm. Especially in the setting of a mixture of gaussians, 
the fact ordinal embedding algorithms aim to reconstruct a euclidean embedding seems to already introduce some bias that might benefit the ordinal embedding algorithms.
Normally, ComparisonHC is a hierachical clustering algorithm, which produces a dendrogram as its output. To use it as a baseline for clustering, we simply
cut the dendrogram off such that it produces the desired amount of clusters.

In the experiment figures, we will  use the abbreviations L-Tangles for Landmark Tangles, and M-Tangles for Majority Tangles.

\section{Gaussian data}\label{sec:gaussian_data}
\subsection{Experimental setup}
We generate a mixture of gaussians as follows: We draw a number of points $n$ each from $k$ different gaussian distributions with means $\mu_1, \mu_2, \ldots \mu_k$ and 
variances $\nu_1, \nu_2, \ldots \nu_k$. Each point gets assigned a label $y_i$ that corresponds to the number $i$ of the gaussian distribution it was drawn from.
For all of the experiments, unless mentioned otherwise, we use $n=20$, $k=3$, fixed means $\mu_1 = \begin{bmatrix} -6.0 & 3.0 \end{bmatrix}, 
\mu_2 = \begin{bmatrix} -6.0 & -3.0 \end{bmatrix},  \mu_2 = \begin{bmatrix} 6.0 & 3.0 \end{bmatrix}$ and a constant variance for all distributions of $\nu = I$, 
with $I$ as the identity matrix. A plot of the data set can be seen in \autoref{fig:dataset-gauss}. We generate the triplets via the euclidean distance between the points, so that
the triplet $(a,b,c)$ implies that $\|a - b\| < \|a - c\|$. For the Tangles algorithms, we use an agreement of $a=7$ (around 1/3 the size of the smallest
clusters we want to detect, in accordance with \cite{klepperClusteringTanglesAlgorithmic2021}), and a radius of $r=\frac{1}{3}$, for the majority
tangles, such that the cuts roughly have the diameter of the clusters.

\begin{figure}[h]
    \centering
    \resizebox{0.8\textwidth}{!}{\input{figures/results/gaussian_small_tangles_clustering.pgf}}
    \label{fig:dataset-gauss}
    \caption{An example draw of the gaussian mixture used for our experiments.}
\end{figure}

\subsection{Lowering density}\label{lower_density}
First, we want to observe how our methods behave under different numbers of triplets present. As the number of triplet grows on the order of $O(N^3)$, it is only feasible to obtain
all triplets for very small datasets. Even with $N = n \cdot k = 60$, as in our case, there are already $106200$ triplets possible. If we imagine that the triplets have to be obtained 
through experiments with real people (such as in psychophysics settings), we might be able to get a few thouand triplets at most. If an algorithm performs better with a lower amount
of triplets, this can quickly translate into really large time, labor and money savings. To test this, we have drawn an increasing amount of triplets from our dataset in a landmark format.
We have plotted the results in \autoref{fig:lower_density_small}. We have also repeated this experiment for a larger number of datapoints, as this is a particularly interesting case. 
Usually, the larger the dataset, the smaller the percentage of all triplets we use, as it grows with $O(N^3)$. However, ordinal embedding algorithms empirically perform well with a lot
less triplets (for example requiring on the order of $O(n d \log(n))$ for euclidean data \citep{jainFiniteSamplePrediction2016}). 
Ideally, we would like for the Tangles algorithm to behave similarly. To test this, we have repeated the experiment with $n=200$ in \autoref{fig:density-change}, and lowered densities.
All other parameters were kept the same.

When looking at the plots, we can see that L-Tangles is performing at least as well as SOE, and even significantly better for a lot of densities in the case of $n=200$. 
As we would have expected, T-STE performs about as well as SOE, albeit a bit worse (and interestingly a lot worse for the larger dataset). ComparisonHC and M-Tangles
perform about the same level, but both stay far behind L-Tangles and SOE. With the larger dataset, we couldn't test ComparisonHC, as our obtained implementation requires 
constructing a $n^4$ matrix during the training step (this could possibly be remedied using a different implementation).

\onecolumn
\begin{figure}[ht]
    \centering
    \subfloat[20 points per cluster]{%
    \resizebox{0.5\textwidth}{!}{\input{figures/results/lower_density_small.pgf}}
    }
    \subfloat[200 points per cluster]{%
    \resizebox{0.5\textwidth}{!}{\input{figures/results/lower_density_large.pgf}}
    }
    \caption{
        We plot the NMI of different clustering method against the percentage of the triplets generated from a draw of a gaussian mixture with $3$ clusters. 
        We draw $20$ data points from each cluster for the left plot, and $200$ data points for the right plot 
        On the x-axis we have the density, where a density of $0.1$ means that we only use 10\% of the total number of triplets. The embedding methods (SOE, T-STE) are 
        followed by k-Means. The Tangles methods (L-Tangles, M-Tangles), are applied with $a=7$. ComparisonHC was left out of the right plot due to computational issues.}
    \label{fig:density-change}
\end{figure}


\subsection{Adding noise}\label{sec:adding-noise}
Next, we want to observe how our algorithm behaves under added noise. This is an important model: most applications of triplet data use triplets that are generated from
real humans. They might disagree on which objects are closer and which are not, which can be modelled as noise on the responses of our triplets. The higher the noise, 
the more disagreement is there about the similarities of objects, so it would matter more about which person you ask than which objects you present them. 
%TODO M-Tangles ComparisonHC
We have plotted our results in \autoref{fig:adding-noise}. 
We observe that L-Tangles falls off a lot quicker in performance than SOE, and falls off a bit harder than T-STE, but they are in the same area. We observe however, that until
relatively large noise levels ($>0.1$), all algorithms performs the same, meaning that L-Tangles can still perform well with low to medium levels of noise.

\begin{figure}[h]
    \centering
    \resizebox{0.7\textwidth}{!}{\input{figures/results/adding_noise_small.pgf}}
    \label{fig:adding-noise}
    \caption{
        We plot the NMI of our chosen clustering methods against the noise that we introduce on the triplets.  We use $3$ clusters and $20$ data points per cluster, sampling all possible triplets. On a noise level of $0.1$, this means that we flip 10\% of the triplets around (turning for example $(a,b,c)$ to $(a,c,b)$). 
    }
\end{figure}

\subsection{Adding noise and lowering density}
In this experiment, we compare \autoref{sec:lowering-density} and \autoref{sec:adding-noise}. We have seen that L-Tangles can perform well with a low
amount of triplets (a bit better than SOE), and have reasonable performance for noisy triplets (worse than SOE). We are now interested to see how large the area is
where L-Tangles can still outperform SOE when we consider both a low density and noisy triplets, as these two factors are probably the most interesting variables
when choosing an algorithm to evaluate triplet data. We have generated two heatmaps, which can be found in \autoref{fig:noise-density-heatmaps}. 
There we can see that L-Tangles outperforms SOE in quite a large region in the low-noise , low-density regime. We would imagine this effect to be even larger
for a $n=200$ setup, but found this computationally too intensive. On the contrary, SOE performs better in the high-density, high-noise regions, with about similar performance
for the cases of low-noise high-density (perfect clustering) and high-noise low-density (random clustering).
% TODO: Repeat for n=200
% TODO: green border around the parts where L-Tangles outperforms?

\onecolumn
\begin{figure}[ht]
    \centering
    \subfloat[SOE]{%
    %\resizebox{0.5\textwidth}{!}{\input{figures/results/gaussian_small_lower_density_noise_heatmap_soe.pdf}}
        \resizebox{0.5\textwidth}{!}{\includegraphics{figures/results/gaussian_small_lower_density_noise_heatmap_soe.pdf}}
    }
    \subfloat[L-Tangles]{%
        \resizebox{0.5\textwidth}{!}{\includegraphics{figures/results/gaussian_small_lower_density_noise_heatmap_l_tangles.pdf}}
    }
    \caption{
        We plot a heatmap of the NMI of SOE and L-Tangles over changing noise and the density of our the triplets. 
        The regions with a darker shade of blue indicate better performance of the algorithm. 
        Note that the noise increases as we move down, and the
        density decreases as we move right.  We use $3$ clusters and $20$ data points per cluster, sampling all possible triplets. 
    }
    \label{fig:noise-density-heatmaps}
\end{figure}

\subsection{Missing triplets}
% TODO
In this experiment, we will use the small gaussian data set and gradually sample an increasing number of triplets for it. 
This is very similar to the setup in \autoref{lower_density}, but this time we sample triplets uniformly at random without replacement
from the set of all triplets. In this setup, we will not be able to use Landmark Tangles as is, as there will be missing values in the cuts. 

A direct idea would be to impute the missing triplets for the cuts.  For, this we just build up our landmark cuts as we did before,
but we mark the information as missing for which we don't have triplet information. These missing values can then be imputed via different methods.
We have used random, $k$-nearest neighbour ($k$-NN) and mean. Random simply sets all missing values to 0 or 1 with equal probability, $k$-NN imputes
a missing entry in a landmark cut with the value that the most similar cut has in that position (closeness being calculated via the manhattan distance), and mean 
imputes the value with the mean of all other cuts in that position. 
We have shown the results for our imputation method in \autoref{fig:missing-triplet-imputations}.  One can see that $k$-NN performs quite reasonably, and $k=1$ even achieves the best performance.
However, the methods fall off more quickly in performance than when we sample landmark cuts directly. This gets more apparent in \autoref{fig:missing-triplet-performance}, where we plot
the performance of $1$-NN imputed L-Tangles against our other methods. We can see that it is competitive against M-Tangles and ComparisonHC, but performs worse than the ordinal embedding algorithms.

\begin{figure}[ht]
    \centering
    \resizebox{0.7\textwidth}{!}{\input{figures/results/imputing_missing_small.pgf}}
    \caption{
        We plot the performance of L-Tangles over the number of triplets available for different imputation methods over our small gaussian dataset with $3$ clusters
        and $n=20$. We use different imputation methods to fill in the missing values and then cluster the resulting cuts with L-Tangles.
        The \textit{k-NN} methods use k-nearest neighbour imputation, \textit{random} assigns 0 or 1 to the missing values with equal probability, and \textit{mean} assigns the missing
        values the mean value over all other cuts at that position.
    }
    \label{fig:missing-triplet-imputations}
\end{figure}

\begin{figure}[ht]
    \centering
    \resizebox{0.7\textwidth}{!}{\input{figures/results/reducing_triplets_small.pgf}}
    \caption{
        Analog to \autoref{fig:missing-triplet-imputations}, but this time we plot the performance of various other evaluation methods on the small gaussian data set against L-Tangles.
        We impute the missing values in the L-Tangles algorithm with $1$-NN.
    }
    \label{fig:missing-triplet-performance}
\end{figure}


% TODO: Experiment that shows how quickly LT starts falling off, maybe with different imputations?
% \subsection{The case of weird geometry}\label{sec:weird-geometry}
% We have made perhaps a bit of a particular choice for the means of our gaussian distribution. This was motivated by a particular behavior we noticed during experimenting. 
% When a cluster center lies exactly between two other cluster centers, we experienced significantly lower performance for the tangles algorithm. Such a setup of clusters can 
% be seen in \autoref{}

\subsection{Discussion}

\section{Hierarchical data}\label{sec:hierarchical_data}
\subsection{Experimental setup}
We generate a noise hierarchical block matrix \citep{balakrishnanNoiseThresholdsSpectral2011}.
The dendrogram described by this model has the form of a complete binary tree, and the similarities of the data points are described via a similarity matrix $M$.
In this matrix the elements in the same cluster have the highest similarity $mu_0$, and for each level $l$ in the dendrogram that two classes are removed
from each other, their similarity decreases by $delta$. We can then add a noise matrix $R$ to the similarity matrix and receive the noisy hierarchical block matrix $M' = M + R$. 
In our setup, we use a noise matrix $R$ where every entry is simply drawn from a normal distribution with mean $0$ and standard deviation $\sigma$.
More about the generation process can be read in \cite{ghoshdastidarFoundationsComparisonBasedHierarchical2019}.
%TODO: Image of similarity matrix?
% can we make this bigger?
We choose a relatively simple setup of $4$ clusters with $10$ data points each, an initial class similarity $\mu_0 = 5$ and a similarity decrease of $\delta = 1$.
In this setup, there are two kinds of noise we encounter: the noise that is injected into the hierarchy itself via the noise matrix and the noise that is added to the triplets itself.
The triplet noise has the same form as in \autoref{sec:gaussian_data}, on noise $p$ we simply flip every triplet with probability $p$. We will investigate how our algorithm does
under both noise models, as well as under a lowered density.

When evaluating, we compare both the final clustering (the lowest level of the hierarchical block matrix) as well as the obtained hierarchy to each other. 
To produce a hierarchy with SOE, we have applied an agglomerative clustering algorithm with average linkage (AL) on the obtained embedding.
% TODO: Explain more about the AARI?
%As Tangles does not produce a dendrogram, we might have a problem obtaining the desired amount of clusters in a certain level. In this case, we simply fill up 
%#For comparing the hierarchy, we can only use Tangles and ComparisonHC, as the embedding methods do not 

\subsection{Lowering density}
As in the gaussian setup, we investigate how our algorithms perform under a lowered density, similar to \autoref{lower_density}. We draw the cuts in a landmark-format. The results can be seen in 
\autoref{fig:hierarchy-lowering-density}. L-Tangles performs about on-par with SOE in the final clustering case (not taking the hierarchy into account), and greatly outperforms all the other 
algorithms, while M-Tangles and ComparisonHC perform about equally well, wtih ComparisonHC getting better results in the high triplet case.

\onecolumn
\begin{figure}[ht]
    \centering
    \subfloat[Comparing Clustering]{%
    \resizebox{0.5\textwidth}{!}{\input{figures/results/hierarchical_lower_density.pgf}}
    }
    \subfloat[Comparing Hierarchies]{%
    \resizebox{0.5\textwidth}{!}{\input{figures/results/hierarchical_lower_densityh.pgf}}
    }
    \caption{
        We plot the performance of our algorithms against the density of the triplets drawn. We draw the triplets in a landmark format 
        and they are generated from a hierarchical noise matrix with $4$ clusters and $10$ data points each. In the left, we see the NMI of 
        the obtained clusterings versus the ground truth, where we only use the final clustering to evaluate. The ordinal embedding algorithms
        (T-STE, SOE) have been followed by k-Means. On the right, 
        we plot the AARI of our methods against the density, where we take the whole hierarchy into account. To obtain a hierarchy for SOE
        we have applied agglomerative clustering with average linkage to the obtained embedding instead of k-Means.
    }
    \label{fig:hierarchy-add-triplet-noise}
\end{figure}

\subsection{Adding triplet noise}\label{sec:h-triplet-noise}
Similar to the gaussian setup in \autoref{sec:adding_noise}, we have simply increased the noise on the sampled triplets and have evaluated the performance of our algorithms. The results
can be seen in \autoref{fig:hierarchy-add-triplet-noise}. We have have repeated the setup under two different densities, with 100\% and with 10\% to see if the density had influence
on the noise suspectibility of the algorithms.
%TODO: Describe the plot
\onecolumn
\begin{figure}[ht]
    \centering
    \subfloat[Comparing Clustering, density  0.1]{%
    \resizebox{0.5\textwidth}{!}{\input{figures/results/hierarchical_add_triplet_noise_01.pgf}}
    }
    \subfloat[Comparing Hierarchies, density 0.1]{%
    \resizebox{0.5\textwidth}{!}{\input{figures/results/hierarchical_add_triplet_noise_01h.pgf}}
    }
    \hfill
    \subfloat[Comparing Clustering, density  0.05]{%
    \resizebox{0.5\textwidth}{!}{\input{figures/results/hierarchical_add_triplet_noise_005.pgf}}
    }
    \subfloat[Comparing Hierarchies, density 0.05]{%
    \resizebox{0.5\textwidth}{!}{\input{figures/results/hierarchical_add_triplet_noise_005h.pgf}}
    }
    \caption{
        Here we plot the performance of our algorithms against the noise introduced on the triplets, meaning that if we
        have a noise of 0.1, we flip a triplet with probability 10\%.  We again use a hierarchical block matrix with $4$ clusters and $10$ points each. 
        On the top row, we draw all triplets in a landmark format, on the bottom row we drawn 10\% of them.    
        We draw $20$ data points from each cluster in the left column, and $200$ data points in the right one.
    }
    \label{fig:hierarchy-add-triplet-noise}
\end{figure}



\subsection{Adding hierarchy noise}\label{sec:adding-hierarchy-noise}
This setup is a bit different than the experiments done on the gaussian data. Here, we vary the noise that we add directly to the hierarchical block matrix $M' = M + R$. 
$R$ is a matrix whose entries all consist of gaussian noise that is drawn from a normal distribution with mean $0$ and variance $\sigma^2$. We now set $\sigma^2$ to various 
values and evaluate the performance of our algorithms. The result can be seen in \autoref{fig:hierarchy-add-hierarchy-noise}. In the normal clustering case, SOE performs the best over the board, with L-Tangles
falling off pretty sharply on the introduction of noise into the hierarchy, but still outperforming M-Tangles and ComparisonHC. In the hierarchical case, L-Tangles again outperforms all other methods.

We can see one interesting effect here: the steep decline in L-Tangles was not present in our other noie model, where we added triplet noise, see \autoref{sec:h-triplet-noise}.
There, we saw a much smoother decline in performance with added noise on the triplets. In fact, we have tested this, and even when adding a vanishingly small amount of noise (say $1e-6$) we
see the decline in performance. This could illustrate a potential shortcoming of the tangles algorithm.

The hierarchical model is a bit hard to imagine.  We have plotted a representation of our hierarchical model in \autoref{fig:hier_noise_repr} for better visualiation
and will now look at the landmark cuts that we retrieve under our two noise models, with a noise of $0.1$ in the triplet noise case and vanishingly small noise in
the hierarchy noise case, which is a lot smaller than the similarities and similarity distances between elements in the hierarchical block matrix, say $1e-6$.

Assume at first that we select two points, one from one of the left clusters, and one from one of the right clusters. When generate the landmark cut that is associated by those two points
(by putting all points that are closer to the left point in the cut, and taking all points that are closer to the the right point out of the cut) we receive is a \textit{coarse} cut, that divides
a higher level of the hierachy (roughly into left and right). As we see in \autoref{fig:hierarchy-add-hierarchy-noise}, this cut looks pretty similar under both noise models. 
We can see that some of the data points have been assigned to the wrong cluster in the triplet noise case (as we would expect), but the hierarchical noise case looks essentially the same.

Next, we will take two points, $a$ from the bottom-left, $b$ from the top-left cluster. The resulting landmark cut is a \textit{fine cut}, that separates between lower levels of the hierarchy.
In this setting, something interesting happens in the hierarchical noise model: the points from the right clusters get randomly assigned to in-cut or out-cut. To understand this, let's first 
look at what happens when we have no hierarchy noise. As in the hierarchical block model, all distances from are only dependent on how far removed the points are in the hierarchy, the bottom-left 
and bottom-right clusters are correctly separated by the cut, but the distance between $a$ and $b$ to a point $c$ from the right clusters is identical. 
We have decided to break ties by assigning the point to out of cluster (so if we build the landmark cut associated
with $(a,b)$ and $d(c,a) = d(c,b)$, then we assign $c$ as not belonging to the landmark cut $(a,b)$, but it doesn't matter if we assign in-cluster, out-cluster or just subdivide the two right clusters correctly as well. 
In our case, it means that all points from the bottom-left cluster will be in the landmark cut $(a,b)$
and all other cuts will be out of the cluster. However, when we add the tiniest amount of noise to the hierarchical model, then for all points $c'$ from the right cluster 
it will randomly be either $d(c,a) > d(c,b)$ or $d(c,b) > d(c,a)$. This means that those points will be randomly assigned to the landmark cut $(a,b)$ as either in-cut or out-cut. 
Thsi can also be seen in \autoref{fig:hierarchy-add-hierarchy-noise} c). For the triplet noise case, we have the ideal assignment (all points from the right clusters are assigned out-cut), 
but some points are again randomly assigned wrongly.

Now, how does that influence our clustering? Let us step through an example clustering that L-Tangles would make with a hierarchical noise model. 
At first, Tangles would receive the coarse cuts (they are cheaper as they are more similar and thus preferred by the cost function) and subdivide the points into the
left and the right clusters. Next, at some point, we would need to align one of the fine cuts. This will subdivide either the left or right clusters (depending on which cut we receive). but
the random assignment in the other cluster (that one which is not subdivided) prevents us from orienting the fine cut of the other clusters consistently (depending on agreement, but if we 
have to align two fine cuts of the same cluster after another, even a very small agreement will not allow consistent alignment). Thus, we will end up with three clusters, the two clusters that are
subdivided by the first fine cut that appeared, and the other two clusters merged to one. 

On the other hand, if we only have to deal with (low) triplet noise, we can align the first coarse cut in any way we want, and the fine cut then gets aligned in one direction in the left subtree
and in the other direction in the other subtree. We end up with the correct amount of clusters, and the missclassifications will only be a few random points that are assigned to a wrong cluster due to triplet noise.

% TODO: Add more information?

\begin{figure}[ht]
    \centering
    \resizebox{0.8\textwidth}{!}{\input{figures/results/hierarchical_model_repr.pgf}}
    \caption{
        Here we plot a euclidean representation of our hierarchical model with a few more data points drawn. We can see a sort-of hierarchy between the clusters, where
        the left and right side are two far removed clusters, which can be subdivided in bottom-left, top-left and bottom-right, top-right.
        Keep in mind that this representation is only a visualisation aid and does not accurately reflect the actual similarity between data points.
    }
    \label{fig:hier_noise_repr}
\end{figure}


\onecolumn
\begin{figure}[ht]
    \centering
    \subfloat[Coarse cut, hierarchical noise]{%
    \resizebox{0.5\textwidth}{!}{\input{figures/results/hierarchical_model_repr_cut_vert_hier_noise.pgf}}
    }
    \subfloat[Coarse cut, triplet noise]{%
    \resizebox{0.5\textwidth}{!}{\input{figures/results/hierarchical_model_repr_cut_vert_triplet_noise.pgf}}
    }
    \hfill
    \subfloat[Fine cut, hierarchical noise]{%
    \resizebox{0.5\textwidth}{!}{\input{figures/results/hierarchical_model_repr_cut_horz_hier_noise.pgf}}
    }
    \subfloat[Fine cut, triplet noise]{%
    \resizebox{0.5\textwidth}{!}{\input{figures/results/hierarchical_model_repr_cut_horz_triplet_noise.pgf}}
    }
    \caption{
        Visualisations of the cuts we would receive under both noise models we deal with in a hierarchical setting, analog to \autoref{fig:hier_noise_repr}.
        Hierarchical noise means noise that we add directly to the hierarchical similarity matrix, while triplet noise means the 
        percents of triplets answered wrongly. We assume landmark cuts. The coarse cuts are those that separated higher levels of the hierarchy (so left and right clusters), while
        the fine cuts further subdivide left and right into bottom-left, top-left, bottom-right and top-right.
    }
    \label{fig:hier_noise_cuts}
\end{figure}

\onecolumn
\begin{figure}[ht]
    \centering
    \subfloat[Comparing Clustering]{%
    \resizebox{0.5\textwidth}{!}{\input{figures/results/hierarchical_add_hierarchy_noise.pgf}}
    }
    \subfloat[Comparing Hierarchies]{%
    \resizebox{0.5\textwidth}{!}{\input{figures/results/hierarchical_add_hierarchy_noiseh.pgf}}
    }
    \caption{
        We plot the performance of our algorithms against the noise on the hierarchical block matrix. A noise of $1$ means that the each 
        entry in the similarity matrix of the hierarchies is independently corruped with additive gaussian noise $r_{ij} \sim \mathcal{N}(0, 1)$.
        We again use a hierarchical block matrix with $4$ clusters and $20$ data points. On the left, we report the NMI of the final clustering against the ground
        truth. On the right, we also take the hierarchies into account, reporting the AARI.
    }
    \label{fig:hierarchy-add-hierarchy-noise}
\end{figure}

\subsection{Discussion}
In this section, we have reported the performance of the Tangles algorithm on two synthetic data sets, gaussian and hierarchical data. 
We could see that with landmark triplets, L-Tangles consistently had the best or among the best performance, even outperforming the 
state-of-the-art triplet embedding algorithm, SOE in regimes of low noie. Additionally, L-Tangles proved to be the best performing algorithm when 
trying to determine the hierarchy in a hierarchical model.  If we are not presented with landmark triplets, we could observe that L-Tangles is still capable of reaching 
an acceptable performance (a bit better than ComparisonHC, which is a state-of-the-art clustering method for clustering triplets without creating an intermediate representation).

Overall, M-Tangles did fare worse than L-Tangles, but it still had acceptable performance, that was overall comparable to ComparisonHC. Also, M-Tangles introduces another hyperparameter, 
the radius, which needs to be tuned. Thus, if triplets are present in landmark format, we strongly prefer L-Tangles, and even if they are not present in landmark format, we imputing the triplet responses
with a simple $1$-NN method and then using L-Tangles seems preferrable to using M-Tangles. We have still included M-Tangles because we think it can serve as an interesting 
baseline from which to build more sophisticated methods, for example one could think about weighing the triplets in a certain way when counting them up (if we have two points $A$, $B$ and we already know
they are very close, then $(A, C, B)$ is a strong indicator of $C$ and $A$ being in the same cluster). 

We have also shown some cases where the tangles algorithms performs subpar: for example in the case of high noise or a large amount of missing triplets (non-landmark format).
We have also raised some issue with the noise in the hierarchical model, and we now want to discuss whether this points to an artifact in the model or a more serious flaw in the tangles
algorithm.

In \autoref{sec:adding-hierarchy-noise}, we have used two different noise models: adding noise to the triplets and adding noise to the hierarchical block model. This points to two fundamentally different
assumptions about our data. Let's think of our data as a hierarchy of 4 clusters, fruits: apples, bananas and vegetables: zucchini and potatoes.
If we wanted to cluster this data using triplet data, we would gather a lot of people, present them with three images of our objects, and ask them whether the first image is more similar to the second or third one.
In the triplet noise model, we assume that either some people answer \enquote{wrongly}, or that some objects might actually be more similar to an object from another cluster than to ones
from their own cluster (maybe we have an image of a banana and an image of a zucchini that have both been shot in front of a beach and thus look similar), but most of the time, there is some kind of 
hidden way to compare items from entirely unrelated hierarchies. In general, apples might always be more similar to potatoes than to zucchinis for example. As a consequence, if we for example have a landmark
cut from an apple and a banana, it would contain all the apples and all the potatoes. 

In the case of hierarchy noise, we simply have some objects that are flat out more similar to one category than another. For example, in the set of apple images, we might have a lot of green apples, which 
are almost always more similar to zucchinis than to potatoes, and a lot of yellow apples, for which it is the other way around. For this reason, when we look at a landmark cut of a zucchini, this one
might contain all bananas and all green apples. As explained in \autoref{sec:adding-hierarchy-noise}, clustering this poses a problem for the tangles algorithm. The actual amount of noise added doesn't matter
(as long as it is smaller than the similarity decreases between distances),  as this information gets lost when turning the similarity matrix into triplets.

Overall, which of the two noise models sounds more realistic might depend completely on the problem setup, how we present the items, how we ask the triplet questions et cetera. Nontheless, even in the
hierarchical noise model, L-Tangles has a very good performance, meaning it can be used without thinking too much about whether the noise model used is appropriate.

\cleardoublepage

\chapter{Real World Data from Psychophysics}\label{real}
Psychophysics is a field of study that investigates the influence of physical stimuli on human perception. An example would be the relationship between the wavelength of 
light and the color sensation that the light produces, see \cite{shepardAnalysisProximitiesMultidimensional1962}. To investigate this relationship, researchers in psychophysics
often set up experiments to collect triplets from human participants. For the example of color perception, a researcher can place a participant in front of a computer 
screen, and repeatedly show them three different colors, $a$, $b$, $c$, together with the question \textit{is a more similar to b or c?} 
The answers of the participants then constitute a triplet data set, which is often analyzed using ordinal embeddings afterward.

In this chapter, we choose a data set consisting of triplets from psychophysics and analyze it using tangles. 
We compare the results of our analysis with the results that other researchers have achieved using more established methods in psychophysics, such as ordinal embeddings.
This chapter thus shows the strengths and weaknesses of tangles on real experimental data.

\section{Data background}
The data we use was collected by Schönmann during her bachelor thesis in \cite{inesschonmannSimilarityJudgementsNatural2021}. Schönmann originally
constructed the dataset to investigate how the formulation of a triplet question influences 
the perception of a person. It consists of the triplet data obtained from multiple
participants using different questions and image sets.  The resulting data sets were then analyzed to show differing similarity perceptions of the
participants depending on how the triplet questions were formulated and which image set was used.
To collect the data, the participants were presented with three images randomly drawn from the specific set of images, together with one out of four different questions. 

The possible questions are: \textit{Which is the odd one out?}, \textit{Which one is more similar?},
\textit{Which one looks more similar?} and \textit{Which concept is more similar?}, out of which we only study the question \textit{Which is the odd one out?} 
The images that are shown to the participants come from three different image sets: \textit{action}, \textit{taxonomic} and \textit{thematic}, out of which we only use the \textit{thematic} image set.
It consists of 5 classes which can be roughly divided into two themes: barn (straw, hay, pitchforks) and kitchen (forks, dishwashers).

The responses of the participants are converted to triplets as follows: if the participant is presented with images $a, b, c$ and signals that $a$ is the odd one out, 
we know that $b$ must be more similar to $c$ than to $a$ and $c$ must be more similar to $b$ than to $a$. Thus, we can gain two triplets from this answer: $(b,c,a)$ and $(c,b,a)$.
For each combination of a participant, image set, and question, the data set consists of $462$ unique triplets. 
Participant $2$ repeated the experiment over a month later, but we have decided not to include those triplets to stay faithful to the original evaluation.

The data set is particularly suited for our tangles experiment, as the image set consists of images from different classes (for example straw, hay, et cetera). 
We expect that this makes the resulting triplets particularly suited for clustering. Additionally, the analysis done by Schönmann already identified certain 
clusters in the data, which we can expect to see in the tangles analysis as well.

\section{Applying Tangles}
\subsection{Setup}
In this section, we show how to apply tangles to the triplet data set of Schönmann.
As there is a lot of data present, it is not feasible to repeat our evaluations for all possible data points.
We choose the triplets of participant 2 to step through rigorously, and then briefly repeat our evaluations for participant 3. We expect similar results for both observers,
as we keep the question and the image set the same.

For the image set, we use the \textit{thematic} one, as Schönmann has reported embeddings that can be cleanly 
separated into different categories.  We also use the \textit{odd-one-out} triplets, as these have been reported in the work by Schönmann as 
having the highest embedding accuracy. As the triplets are not in a landmark format, but uniformly sampled from the set of all triplets, we use majority cuts. 

We process the triplets to cuts using the majority cuts approach with a radius of 1 and apply tangles using an agreement of 3 and with the mean manhattan cost function (see \autoref{mmcf} in \autoref{sims-methods}). 
As a visualization aid, we plot the clustering onto a 2d-embedding from SOE. Care must be taken: the embedding
from SOE is not a ground-truth embedding and just serves as a visualization aid. If the elements of a cluster are far 
apart in this visualization, this does not mean that the clustering is wrong – it could just as well be that the distances in the embedding do not correctly 
represent the ground truth similarity of the data points.

In the analysis, Schönmann qualitatively identified different clusters for participant 2 on the \textit{thematic} image set with the \textit{odd-one-out} triplets. 
To do this, she identified a separating line in the SOE embedding of the triplets, which divided the objects into the two categories \textit{barn} (hay, pitchfork, straw) and 
\textit{kitchen} (dishwasher, fork). We expect to see a similar divide in the clusters produced by tangles. 
As a follow-up, we use the hierarchical clusterings and the explanations produced by tangles to get further insights into the produced clustering. 

\subsection{Embedding and clustering}
We first reproduce the results by Schönmann by embedding the data with SOE into two dimensions 
(see \autoref{fig:ines-embedding-a}). In this embedding, one can linearly separate two sets of clusters, which correspond to a divide between kitchen objects 
(dishwasher, fork) and barn objects (hay, straw, pitchfork), as reported by Schönmann in her thesis. Then we process the triplets to majority cuts, apply tangles 
and obtain a clustering. These cluster labels are visualised in \autoref{fig:ines-embedding-b}) on the embedding from SOE. 

\begin{figure}[h]
    \centering
    \subfloat[Original classes]{%
    \resizebox{0.5\textwidth}{!}{\input{figures/results/ines_vp2_odd_thematic_classes.pgf}}
    \label{fig:ines-embedding-a}
    }
    \subfloat[Tangles clusters]{%
    \resizebox{0.5\textwidth}{!}{\input{figures/results/ines_vp2_odd_thematic_tangles.pgf}}
    \label{fig:ines-embedding-b}
    }
    \hfill
    \subfloat[Embedding with images]{%
    \centering
    \resizebox{0.85\textwidth}{!}{\includegraphics{figures/results/vp2_embedding_thematic_stimuli_images.pdf}}
    \label{fig:ines-embedding-c}
    }
    \caption{
        Embedding of the \textit{odd-one-out} triplets from participant 2 on the thematic image set.
        In a), we see the original classes, and in b) we see the predictions that we receive by
        applying majority tangles with an agreement of 3 and a radius of 1.
        In c), we plotted the original images at the coordinates of their respective SOE embedding.
    }
    \label{fig:ines-embedding}
\end{figure}

We can see that the clustering from tangles (\autoref{fig:ines-embedding-b}) also produces a similar divide between kitchen (orange triangles) and barn (blue circles) items. 
However, we see a third cluster structure (grey squares), which is a mix between two pitchforks and a kitchen fork. We want to determine whether this is an erroneous clustering, which might arise
from too few triplets sampled, or possibly a new insight that we gained.

When we look at items in the grey squares cluster (depicted in \autoref{fig:thematic-images-forks-a}), we notice that they look more dissimilar to their
counterparts in the kitchen or barn cluster. The fork (left in \autoref{fig:thematic-images-forks-a}) has a design that is more reminiscent of pitchforks, and the two pitchforks
look cleaner than the other pitchforks in \autoref{fig:thematic-images-forks-c} (no dirt on them, not depicted lying in grass). Thus, it makes sense that these three items are judged
as more similar to each other than to their counterparts in the kitchen and barn cluster and thus get put into a separate cluster. 
Interestingly, that is an insight that could not be reached from the SOE embedding alone, as the items are relatively far away from each other in the embedding. Thus, tangles might provide valuable information for a researcher. 

\begin{figure}[h]
    \centering
    \subfloat[Third cluster of different forks (gray squares)]{%
        \label{fig:thematic-images-forks-a}
        \resizebox{0.333\textwidth}{!}{\includegraphics{figures/thematic_stimuli/fork_03s.jpg}}
        \resizebox{0.333\textwidth}{!}{\includegraphics{figures/thematic_stimuli/pitchfork_09s.jpg}}
        \resizebox{0.333\textwidth}{!}{\includegraphics{figures/thematic_stimuli/pitchfork_16s.jpg}}
    }
    \hfill
    \subfloat[Other kitchen forks]{%
        \resizebox{0.2\textwidth}{!}{\includegraphics{figures/thematic_stimuli/fork_05s.jpg}}
        \resizebox{0.2\textwidth}{!}{\includegraphics{figures/thematic_stimuli/fork_06s.jpg}}
        \resizebox{0.2\textwidth}{!}{\includegraphics{figures/thematic_stimuli/fork_many.jpg}}
        \resizebox{0.2\textwidth}{!}{\includegraphics{figures/thematic_stimuli/fork_plastic_light.jpg}}
        \resizebox{0.2\textwidth}{!}{\includegraphics{figures/thematic_stimuli/fork_wood.jpg}}
    }
    \hfill
    \subfloat[Other pitchforks]{%
        \label{fig:thematic-images-forks-c}
        \resizebox{0.25\textwidth}{!}{\includegraphics{figures/thematic_stimuli/pitchfork_02s.jpg}}
        \resizebox{0.25\textwidth}{!}{\includegraphics{figures/thematic_stimuli/pitchfork_03s.jpg}}
        \resizebox{0.25\textwidth}{!}{\includegraphics{figures/thematic_stimuli/pitchfork_04s.jpg}}
        \resizebox{0.25\textwidth}{!}{\includegraphics{figures/thematic_stimuli/pitchfork_15s.jpg}}
    }
    \caption{
        The images of all forks that are present in the thematic data set. In a), we show the pitchforks and forks that landed in
        the gray squares cluster in the tangles clustering (see \autoref{fig:ines-embedding}), which contained a mixture of kitchen and barn items.
        In b), we have depicted all other kitchen forks (orange triangle cluster, kitchen items) and in c) we depict all other pitchforks 
        (blue circle cluster, barn items).
    }
    \label{fig:thematic-images-forks}
\end{figure}

\FloatBarrier
\subsection{Hierarchical clustering}
Next, we explore what the hierarchy of the clusters looks like.
For this, we plot the hierarchy that we receive from the tangles algorithm in \autoref{fig:soft-tree-vp2}. 
As expected, we first see a coarse, thematic divide between the kitchen and the barn cluster, with 
the clean-looking pitchforks being placed together with the kitchen cluster. 
We then see a finer clustering of the kitchen cluster (node 2), which is split into a set of various kitchen items (node 5), and the cluster of special forks (node 6). 
The fact that the special forks belong to the coarse kitchen cluster and only get separated in a later step could be interpreted as the participant thinking of the 
clean-looking pitchforks as belonging more to a kitchen than a barn. This is an insight that we didn't see from the ordinal embedding alone,
highlighting a possible strength of tangles.

% contains the color scale
%TODO: Increase bar font, remove x,y font.
\begin{figure}[ht]
    \centering
    \resizebox{\textwidth}{!}{\includegraphics{figures/results/vp2_soft_tree.pdf}}
    \caption{
        The hierarchy produced by the tangles algorithm on the thematic image set for participant 2. 
        Each node represents a (soft) clustering, which we plot in its place. The points are embedded via SOE as a visualization aid.
        The color of a point corresponds to the probability that the point belongs to the given cluster, the darker, the more probable. 
    }
    \label{fig:soft-tree-vp2}
\end{figure}

\subsection{Explainability}\label{real-explain}
So far we have inspected the clusters that tangles produced. Now, we show how to use tangles to explain the clustering:
Why does a particular image belong to the kitchen or barn cluster? For this, we can look at the characterizing cuts of the clusters. As a reminder, 
these are the cuts at the splitting nodes that contribute to a meaningful decision between the left and the right subtree.

We visualize the characterizing cuts in \autoref{fig:characterising_cuts}, together with the images that induce the particular cuts. As our cuts are interpretable (a majority cut 
with anchor point $a$ contains points that are close to $a$), we can directly use this interpretation for our clustering. As we can see in \autoref{fig:characterising_cuts_a}, 
the items in the barn cluster are there because they are similar to the two straw/hay images shown in \autoref{fig:characterising_cuts_b}. 
%This is similar to our intuition, straw and hay intuitively belong in a barn setting and not in a kitchen, while pitchforks might be a bit more ambiguous.

Our interpretation is that the items that are in the cluster of other kitchen items are there because they are similar to
the dishwashers we have plotted in \autoref{fig:characterising_cuts_d}. This also makes sense, as we would interpret the dishwashers to very clearly belong in a kitchen environment, while the forks might be more ambiguous.

Overall, there is a small caveat to our explanation. In a majority cut, we only have a satisfying explanation in one direction: We know that if a point $b$ is in the majority cut that has anchor point $a$, 
then $b$ is close to $a$. However, the reverse direction might be unsatisfying: if a point $c$ is not in the majority cut, we know that it is not close to $a$. 
If we want to know how the cluster of forks and pitchforks formed, saying that they are dissimilar to the dishwashers plotted in \autoref{fig:characterising_cuts_d} is not a strong argument.
For example, we would rather know that they are similar to a certain item.  To remedy this, we can use more interpretable cuts. 
If the data would have been suitable for landmark cuts, we would have been able to make statements of the form: $a$ is in a certain cluster because it is closer to $b$ than to $c$.


\begin{figure}[ht]
    \centering
    \subfloat[Characterising cuts coarse cluster (kitchen-barn)] {
        \label{fig:characterising_cuts_a}
        \resizebox{0.333\textwidth}{!}{\import{figures/results}{node_18T_characterizing_12.pgf}}
        \resizebox{0.333\textwidth}{!}{\import{figures/results}{node_18T_characterizing_23.pgf}}
    }
    \hfill
    \subfloat[Images of data points point inducing characterizing cuts for coarse cluster] {
        \label{fig:characterising_cuts_b}
        \resizebox{0.25\textwidth}{!}{\includegraphics{figures/thematic_stimuli/straw_hay_01b.jpg}}
        \resizebox{0.25\textwidth}{!}{\includegraphics{figures/thematic_stimuli/hay_02s.jpg}}
    }
    \hfill
    \subfloat[Characterising cuts fine cluster (kitchen/special forks)]{
        \label{fig:characterising_cuts_c}
        \resizebox{0.333\textwidth}{!}{\import{figures/results}{node_19F_characterizing_0.pgf}}
        \resizebox{0.333\textwidth}{!}{\import{figures/results}{node_19F_characterizing_1.pgf}}
        \resizebox{0.333\textwidth}{!}{\import{figures/results}{node_19F_characterizing_3.pgf}}
    }
    \hfill
    \subfloat[Images of data points inducing characterizing cuts for fine cluster] {
        \label{fig:characterising_cuts_d}
        \resizebox{0.25\textwidth}{!}{\includegraphics{figures/thematic_stimuli/dishwasher_03s.jpg}}
        \resizebox{0.25\textwidth}{!}{\includegraphics{figures/thematic_stimuli/dishwasher_09s.jpg}}
        \resizebox{0.25\textwidth}{!}{\includegraphics{figures/thematic_stimuli/dishwasher_05s.jpg}}
    }
    \caption{
        Depiction of the characterizing cuts of all splitting nodes on the thematic image data set.  We draw the orientation that corresponds to the left subtree 
        (which is inverted for the right subtree). This means, that if a datapoint is often contained in the cuts that we depicted above, it is placed in the left subtree with high probability. In a), we plotted the characterizing cuts for the root node and in c) have plotted
        the characterizing cuts for the kitchen/special forks cluster. As these cuts come from our set of majority cuts, we have marked the data point that 
        induced the particular cut with a black X. Below the characterizing cuts in b) and d), we have plotted the images corresponding to the marked points.
    }
    \label{fig:characterising_cuts}
\end{figure}

\FloatBarrier
\subsection{Evaluation of another participant}
Next, we check if the results can be repeated for another participant. 
We select the odd triplets generated from the thematic image set of participant 3 and expect to see similar behavior to our evaluations for participant 2. 
In \autoref{fig:ines-embedding-vp3} we plot the embedding
from Schönmann again together with the tangles clustering. This time, we see three clusters (hay/straw, pitchforks and dishwashers/forks). These clusters coincide nicely
with our classes, aside from one fork being clustered together with the pitchforks. We note that this is not the fork from \autoref{fig:thematic-images-forks-a} that was clustered together 
with the pitchforks, so this might be a misclassification. 
If we look at the hierarchy in \autoref{fig:soft-tree-vp3}, we see that (contrary to participant 2) the pitchforks first get split off, 
and then the kitchen from the straw-hay cluster. This could indicate that participant 3 deems the straw/hay to be more similar to the kitchen items
than to the pitchforks, which could provide valuable insights for further analysis.
%TODO: Ask David
%This could again lead to interesting conclusions, which should probably be discussed by someone with more expert knowledge in the field, possibly using other evaluation data.

\begin{figure}[ht]
    \centering
    \subfloat[Original classes]{%
    \resizebox{0.5\textwidth}{!}{\input{figures/results/ines_vp3_odd_thematic_classes.pgf}}
    }
    \subfloat[Tangles clusters]{%
    \resizebox{0.5\textwidth}{!}{\input{figures/results/ines_vp3_odd_thematic_tangles.pgf}}
    }
    \hfill
    \subfloat[Embedding with images]{%
    \centering
    \resizebox{0.85\textwidth}{!}{\includegraphics{figures/results/vp3_embedding_thematic_stimuli_images.pdf}}
    }
    \caption{
        Analog to \autoref{fig:ines-embedding}. In a), we plot the original classes over an SOE embedding, in b) we plot the prediction of majority tangles with agreement 3 and radius 1 over 
        a 2-dimensional SOE embedding. In c), we see the original images at the coordinates of their respective SOE embedding.
    }
    \label{fig:ines-embedding-vp3}
\end{figure}

\begin{figure}[ht]
    \centering
    \subfloat[]{%
    \resizebox{\textwidth}{!}{\includegraphics{figures/results/vp3_soft_tree.pdf}}
    }
    \caption{
        Tangles produce similar clusters on participant 3 and participant 2. We plot the hierarchy of a tangles clustering on the triplets from participant 3 on the thematic image set. 
        On each node, we plot the soft clustering corresponding to it.
    }
    \label{fig:soft-tree-vp3}
\end{figure}
\FloatBarrier
\section{Discussion}
In this chapter, we evaluated tangles on real-world data.  
We clustered the data with majority tangles, which agreed well with the qualitative analysis that Schönmann has done on the data using more established methods. 
In addition, the more flexible clustering by tangles can show new dependencies between data points, as we 
made out a cluster of forks being perceived differently by participant 2. 

The hierarchy provided by tangles can help put these new dependencies into a better perspective, 
and allowed us to gather that the pitchforks in the special fork cluster were perceived to 
belong more in a kitchen setting than in a barn setting.
The explanations by tangles were then used to discover the more defining items of a cluster (straw/hay for
a barn, dishwashers for a kitchen). To our knowledge, hierarchical and explainable methods
have not been used on triplets in psychophysics to this date, allowing tangles to fill a potential niche.

\FloatBarrier

\cleardoublepage

\chapter{Conclusion}\label{conclusion}
\section{Where Tangles can be useful}
\section{Further developments}
\section{What this work has achieved}

\cleardoublepage

%%%%%%%%%%%%%%%%%%%%%%%%%%%%%%%%%%%%%%%%%%%%%%%%%%%%%%%%%%%%%%%%%%%%%%%%%%%%%
%%% Appendix
%%%%%%%%%%%%%%%%%%%%%%%%%%%%%%%%%%%%%%%%%%%%%%%%%%%%%%%%%%%%%%%%%%%%%%%%%%%%%
%\appendix
%
%%\setcounter{secnumdepth}{-1}
%%\section{Tables}\label{chap:App}
%\chapter{Further Tables and Figures}\label{chap:App}
%Viele Arbeiten haben einen Appendix. Besondere Sorgfalt muss beim Nummerieren der Tabellen und Abbildungen gewährleistet sein.
%\begin{table}[htb]
%\begin{tabular}{cc}
%Nummer & Datum \\
%\hline
%1 & 1.1.80\\
%2 & 1.1.90 \\
%\end{tabular}
%\caption{Erste Appendix-Tabelle}\label{tab:app1}
%\end{table}
%
%%\chapter{Figures}\label{chap:App2}
%
%\begin{table}[htb]
%\begin{tabular}{cc}
%Nummer & Datum \\
%\hline
%1 & 1.1.80\\
%2 & 1.1.90 \\
%\end{tabular}
%\caption{Zweite Appendix-Tabelle}\label{tab:app2}
%\end{table}
%%\end{appendices)
%
%\cleardoublepage

%%%%%%%%%%%%%%%%%%%%%%%%%%%%%%%%%%%%%%%%%%%%%%%%%%%%%%%%%%%%%%%%%%%%%%%%%%%%%
%%% Bibliographie
%%%%%%%%%%%%%%%%%%%%%%%%%%%%%%%%%%%%%%%%%%%%%%%%%%%%%%%%%%%%%%%%%%%%%%%%%%%%%

\addcontentsline{toc}{chapter}{Bibliography}
\bibliography{Masterthesis}
%% Obige Anweisung legt fest, dass BibTeX-Datei `mylit.bib' verwendet
%% wird. Hier koennen mehrere Dateinamen mit Kommata getrennt aufgelistet
%% werden.

\cleardoublepage
%%%%%%%%%%%%%%%%%%%%%%%%%%%%%%%%%%%%%%%%%%%%%%%%%%%%%%%%%%%%%%%%%%%%%%%%%%%%%
%%% Erklaerung
%%%%%%%%%%%%%%%%%%%%%%%%%%%%%%%%%%%%%%%%%%%%%%%%%%%%%%%%%%%%%%%%%%%%%%%%%%%%%
\thispagestyle{empty}
\section*{Selbst\"andigkeitserkl\"arung}

Hiermit versichere ich, dass ich die vorliegende Masterarbeit 
selbst\"andig und nur mit den angegebenen Hilfsmitteln angefertigt habe und dass alle Stellen, die dem Wortlaut oder dem 
Sinne nach anderen Werken entnommen sind, durch Angaben von Quellen als 
Entlehnung kenntlich gemacht worden sind. 
Diese Masterarbeit wurde in gleicher oder \"ahnlicher Form in keinem anderen 
Studiengang als Pr\"ufungsleistung vorgelegt. 

\vskip 3cm

Ort, Datum	\hfill Unterschrift \hfill 
%%%%%%%%%%%%%%%%%%%%%%%%%%%%%%%%%%%%%%%%%%%%%%%%%%%%%%%%%%%%%%%%%%%%%%%%%%%%%
%%% Ende
%%%%%%%%%%%%%%%%%%%%%%%%%%%%%%%%%%%%%%%%%%%%%%%%%%%%%%%%%%%%%%%%%%%%%%%%%%%%%

\end{document}

