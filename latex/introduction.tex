\chapter{Introduction}\label{Introduction}
We consider the task of clustering data $X = \{x_1, x_2, \ldots x_n \}$ for which neither 
explicit features nor concrete distances between the data points are known to us. 
The only form of information we have on the data is in the form of so-called 
\textit{triplet comparisons} or \textit{triplets} for short. A triplet 
on $X$ is a triple $(x_i, x_j, x_k)$ which tells us that $x_i$ is closer to $x_j$ than to $x_k$. 
This problem setting often arises when one works with data from human observers. 
An example, investigated among others by \citep{shepardAnalysisProximitiesMultidimensional1962}, 
is human color perception. It would be hard for human observers to accurately rate colors in 
terms of features, let alone define sensible features in the first place.  It would also be hard to rate colors in terms of concrete distances to each other.
Additionally, the experiment designer would have to deal with uniting differing internal scales of different observers during data evaluation. 
A preferred approach might be to gather triplets on the data. 
To obtain these triplets, the experiment designer can repeatedly draw
three different colors, which are presented to a human observer, together with a 
suitable question. For example, we might draw the colors violet, red, yellow,
and ask the the observer \textit{is violet more similar to red or to yellow?} These questions
are comparatively easy to answer for human observers. 
What remains is the question of how to evaluate the obtained triplet data.

A small research community has formed around the task of dealing with triplets.
Most of this community focusses on ordinal embeddings
\citep{agarwalGeneralizedNonmetricMultidimensional2007, tamuzAdaptivelyLearningCrowd2011,
laurensvandermaatenStochasticTripletEmbedding2012,   teradaLocalOrdinalEmbedding2014, jainFiniteSamplePrediction2016, ghoshLandmarkOrdinalEmbedding2019, andertonScalingOrdinalEmbedding2019}.
An ordinal embedding is an algorithm that aims to place the data points into a euclidean space
such that the euclidean distances between the embedded points respects as many of the original triplet comparisons as possible. 
However, this approach is not always perfect: we often cannot satisfy all triplet 
comparisons, no matter the dimension of the embedding space or how the points are placed in it.
% maybe leave this out?
This is because the euclidean distance is a proper metric, so it is symmetrical and has to obey the triangle inequality, limiting its flexibility. 

A big advantage of ordinal embeddings however is precisely that we obtain a euclidean representation of the data. For euclidean data, there are many good, readily available algorithms for almost all tasks. 
Thus, clustering on triplet data can be tackled by getting a euclidean representation of the data
from an ordinal embedding and applying a classical clustering algorithm (such as k-Means) on 
this representation. This approach was for example demonstrated as a baseline in \citep{kleindessnerLensDepthFunction2017}. However, as mentioned before, an ordinal embedding 
often cannot satisfy all triplet comparisons on the original data, and is therefore not 
necessarily a faithful representation. 
An alternative approach is devising an algorithm to solve the desired task directly 
using the triplets, which has shown promising results.
\citep{kleindessnerLensDepthFunction2017} estimated the lens-depth function
of the data using triplets (of a slightly altered format) and used this for 
medoid estimation, outlier identification, clustering and classification. 
\citep{kleindessnerKernelFunctionsBased2017} constructed a kernel function from
the triplets, which they demonstrated to work well with 
different kernel-based clustering algorithms.
\citep{ghoshdastidarFoundationsComparisonBasedHierarchical2019} used the 
triplets to estimate a similarity function between the data points and applied 
a linkage algorithm to obtain a hierarchical clustering from the data.

In a recent work, \cite{klepperClusteringTanglesAlgorithmic2021} proposed a novel framework
for clustering based on tangles, which are a mathematical tool originating from graph theory 
\citep{robertsonGraphMinorsObstructions1991}. The tangles algorithm has been shown to have 
interesting properties, such as inherent explainability (under certain conditions) and being suitable for hierarchical clustering.
One of the central objects in this framework are bipartions, ways of dividing a set into
two distinct, non-overlapping subsets. To cluster data using the tangles framework, we first
need to obtain a set of bipartitions on the data. We additionally require that these bipartitions
hold a little bit of information about the cluster structure of the data. The 
tangles framework can then aggregate the information contained in these bipartitions to a
clustering. 

In this work, we present two novel methods to generate bipartitions suitable for the
tangles algorithm, using only triplets. Using these methods, we 
demonstrate that the tangles algorithm can be successfully applied to (hierarchically) cluster data for which we only know triplets without creating an intermediate ordinal embedding. 
We evaluate our approach by simulated and experimental data and show that it is competitive 
in performance to ordinal embedding based approaches, while providing explainable results. 

This thesis is organized as follows: In \autoref{theory}, we give an introduction to tangles and ordinal data, which lays the necessary foundations for the rest of the thesis. 
In \autoref{methods}, we present our two extensions of the tangles framework, named \textit{landmark tangles} and \textit{majority tangles}.
We use simulated data in \autoref{simulations} to show the performance of our algorithms under different circumstances, such as the noise level or availability of the triplets. 
In \autoref{real} we showcase an example evaluation with real data from the domain of psychophysics, to establish tangles as a practical tool on triplet data.
Doing so, we also highlight the inherent explainability of the clustering produced by 
the tangles algorithm. 
