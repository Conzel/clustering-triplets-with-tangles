\chapter{Introduction}\label{Introduction}
\textit{Is item i closer to item j or item k?} is the central question in a triplet setting. As humans have problems ordering 
most things on absolute scales (what would be values of the features \textit{pop, rock, metal} of Rihanna?), a research community has formed
to investigate how to work with comparison or triplet data. Here, humans are only asked to rate the similarity of objects, instead
of their absolute values, which is intuitively easier (Rihanna being more similar to Britney Spears than to Metallica seems logical). 

Another interesting venue for triplet questions is research on human perception: purple and red might be very far removed in terms of their wavelength,
but we actually perceive them as very similar \citep{shepardAnalysisProximitiesMultidimensional1962}. Insights like these can be uncovered through triplet data 
with appropriate data analysis methods. 

Most of the research community has dealt with ordinal embeddings \citep{agarwalGeneralizedNonmetricMultidimensional2007, tamuzAdaptivelyLearningCrowd2011,
laurensvandermaatenStochasticTripletEmbedding2012,   teradaLocalOrdinalEmbedding2014, jainFiniteSamplePrediction2016, ghoshLandmarkOrdinalEmbedding2019, andertonScalingOrdinalEmbedding2019}, which 
are algorithms that try to embed the original data points into a euclidean space from the given triplet comparisons, with the goal that the euclidean distance between the new points
in the euclidean space respects the original triplet comparisons. 
This approach is not always perfect: the objects might have similarity values to each other that simply cannot be captured by euclidean distances alone. 
The euclidean distance is a proper metric, so it is symmetrical and has to obey the triangle inequality. 

With an ordinal embedding, the triplet data is brought into a more common format and accessible for further tasks, such as clustering or classification with classical methods \citep{kleindessnerLensDepthFunction2017}. However, this approach seems a bit lacking: if we already know that an ordinal embedding might introduce distortions in our data, why not try to solve the end tasks on
the triplet data directly? This approach has shown promising results in 
\citep{kleindessnerLensDepthFunction2017, kleindessnerKernelFunctionsBased2017, ghoshdastidarFoundationsComparisonBasedHierarchical2019} for clustering, classification and hierarchy reconstruction. 

In this thesis, we want to use the tangles framework on clustering by \cite{klepperClusteringTanglesAlgorithmic2021} and extend it to an algorithm that is capable of clustering and reconstructing
hierarchies on triplet data. First, in \autoref{theory}, we will give an introduction to tangles and ordinal data aimed at the unfamiliar reader, which lays
the necessary foundations for the rest of the thesis. 
In \autoref{methods}, we will present our two extensions of the tangles framework, which we have coined the \textit{landmark tangles} and \textit{majority tangles}
and discuss when and how to apply them.
We will use simulated data in \autoref{simulations} to show the performance of our algorithms under different environmental factors, such as the noise of the triplets on simulations or how many triplets are available. 
To establish tangles as a practical tool on triplet data, in \autoref{real} we will showcase an example evaluation with real data from the domain of psychophysics. 
Here, we will also highlight the inherent explainability of the tangles framework. 

In the end, we will have shown that tangles can make a viable alternative for clustering triplet data and will have identified use cases where it performs especially well.
