\chapter{Applying Tangles to Triplet Data}\label{methods}
As we have made out in \autoref{theory:tangles}, the tangles algorithm operates on bipartitions of data that contain some information about the cluster structure.
If we have obtained a set of triplet data, we are now faced with the task of processing this data to appropriate bipartitions. In this work, we have developed two
methods for this which we call \textit{landmark cuts} and \textit{majority cuts}. We will elaborate on the methods and intuitions in the following sections.

\section{Landmark cuts}\label{theory:landmark_cuts}
In recent years, there have been algorithms that hope to speed up ordinal embedding by focussing on so-called \textit{landmarks} \citep{ghoshLandmarkOrdinalEmbedding2019, andertonScalingOrdinalEmbedding2019}, which are objects in the dataset for which we know all (or all relevant) triplet comparisons.  
The definitions of what constitutes landmarks varies a bit, but we will use the one that is used in \cite{haghiriComparisonBasedFrameworkPsychophysics2019}.
Assume we have a set of objects $D$, as well as a set of $T$ triplet constraints of the form $(a,b,c)$, indicating that $d(a,b) < d(a,c)$. In a landmark setting, we have a set 
of $m$ objects $L \subset D$, for which 
\begin{align*}
\forall l_1, l_2 \in L \; \forall x \in D: (x, l_1, l_2) \in T \vee (x, l_2, l_1) \in T. 
\end{align*}
Landmarks make it very easy to define a set of bipartitions on triplet data:
for each combination of landmarks $l_1, l_2$, we can make a bipartition $P = \{A, \overline{A}\}$ by assigning all points closer to $l_1$ to $A$ and all 
those closer to $l_2$ to $\overline{A}$. Thus we can define the bipartition between landmark points $l_1, l_2$ as $A_{12} = \{ x \in D \mid \left( x, l_1, l_2 \right) \in T \}$, 
denoting as $P_L = \{A_{ij}  \mid i, j \in \{1\ldots m\},\, i < j\}$ all such possible bipartitions on $L$. 
These bipartitions are then called \textit{landmark cuts}.
Later in the simulations, as tangles is not reliant on having all triplet constraints for all landmark objects, we will simply sample a subset $P'_{L} \subset P_L$ of bipartitions,
which corresponds to repeatedly picking some objects $a, b$ and sampling all triplet comparisons to all other objects $x \in D$. \\

Landmark cuts intuitively capture a cluster structure: the more close $x$ is to $l_1$ according to $d$, the more likely that $A_{12}$ will contain $x$. In the end, $A_{12}$ will
consist of the points that are in some sense close to $l_1$ (how strong this closeness is depends on how close $l_1$ and $l_2$ are). 
In the euclidean space, this notion is very easily captured: A landmark cut between $l_1, l_2$ is
simply a linear cut between the two points, as illustrated in \autoref{fig:landmark_cut}.

    \begin{figure}[ht]
        \centering
        \includegraphics[width=0.8\textwidth]{figures/landmark_cut.png}
        \caption{Example of a landmark cut on a euclidean data set, with the two triangles being the landmarks $l_1, l_2$, where all blue items (left of the line) 
        would be assigned to $A$, all red items to $\overline{A}$ (or equivalently the other way around).}
        \label{fig:landmark_cut}
    \end{figure}

The landmark approach is a quite unusual way of sampling triplet questions.
In most triplet experiments, the triplets to sample are chosen randomly \citep{kleindessnerLensDepthFunction2017, haghiriEstimationPerceptualScales2020} 
or according to some metric, for example maximizing some measure of gained information \citep{roadsEnrichingImageNetHuman2021}. 
There was no experimental dataset available which is both sampled according to a landmark approach and exhibits a cluster structure, thus we rely on simulations
for testing landmark cuts.

\section{Majority cuts}\label{theory:majority_cuts}
As explained in \autoref{theory:landmark_cuts}, sampling triplet data in a landmark-fashion is not very widely used in current practice. Due to this, we aimed
to also develop a more general approach to processing tangles to cuts that can be applied to any set of triplet comparisons $T$ regardless of the sampling method.
For this, we again use the intuition that triplets tell us something about the closeness of data points, and thus about the cluster structure. 

Assume again that we have a set of objects $D$ and a set of triplet comparisons $T$. We fix two points $a,b \in D$. Assume that $a,b$ are close and we sample a random point $x$.
With high probability, it will be that $d(a,b) < d(a,x)$. The reverse holds if $a,b$ are far away. Now we can take this the other way around: 
if we observe a triplet $(a,b,c) \in D$, then we can more reasonably assume that $a,b$ are close than $a,c$ being close. This leads us to the following method of defining
a set of close points of $a$: Let $L_x = \{t \in T | t = (x, b,c), b,c \in D\}$ be the set of all available triplets where $a$ is in the left position, and equivalently $M_x$ and $R_x$ 
the set of triplets where $x$ is in the middle and right position. Then we define: $P_a := \{x \in D \mid \left|   L_a \cap M_x\right| < \left| L_a \cap R_x \right| \}$
which is the set of all points that are more often closer to $a$ than they are farther away. We refer to these bipartitions as \textit{majority cuts}, and to the point $a$ that
as the \textit{anchor point} of the cut $P_a$. These majority cuts capture
some cluster structure by assigning points that are close together to the same bipartition. Majority cuts can be made more flexible by including a ratio $r$ that controls the size of the cuts. 
We then define $P_a(r) := \{x \in D \mid \left|   L_a \cap M_x\right| < r \cdot \left| L_a \cap R_x \right| \}$ and call $r$ the \textit{radius} of the cut. 
The definition of majority cuts given before is then simply the case $r = 1$. 

Next, we want to gain some intuition about the form of the majority cuts. What we would expect, assuming that we had all triplets, is that each cut
$P_a(r)$ is a ball around $a$ that contains the $n \cdot \frac{r}{r+1}$ points that are closest to $a$ according to the distance measure $d$ that defined the triplets. In a euclidean setting for 
$r = 0.5$ we thus expect $P_a(1)$ to be a ball around $a$ that contains the $\frac{n}{3}$ points that are closest to $a$. To test this, we plot a majority cut with a fixed anchor point on a 
simple mixture of gaussians in \autoref{fig:majority_cut}. 
We can see that we get closer to the ideal form of a ball around $a$ (the point marked with a cross) with radius $\frac{n}{3}$ as we increase the amount of sampled triplets.
When we have fewer than all triplets available, our cut contains some noise in the form of points that in the cut together with $a$, but are outside of the $\frac{n}{3}$ ball around $a$ (and might in fact be very far away).

This intuition already gives us hints how we might want to choose the radius: if we pick a smaller radius, we will detect smaller clusters versus when we pick a larger radius. 
%TODO: Investigate this claim?
In particular, we should not pick a radius smaller than the smallest cluster we want to detect, else the tangles algorithm will quickly have troubles consistently aligning
the cuts. On the contrary, we are safe if we pick a radius that is a bit larger, as the tangles algorithm can work well with cuts that contain a cluster together with some additional data points 
(as long as they are not always the same points on all cuts).

\onecolumn
\begin{figure}[ht]
    \centering
    \subfloat[500 triplets]{%
      \resizebox{0.5\textwidth}{!}{\input{figures/results/majority-cut-8-n_triplets-500.pgf}}
  }
    \subfloat[5000 triplets]{%
      \resizebox{0.5\textwidth}{!}{\input{figures/results/majority-cut-8-n_triplets-5000.pgf}}}
    \hfill
    \subfloat[20000 triplets]{%
      \resizebox{0.5\textwidth}{!}{\input{figures/results/majority-cut-8-n_triplets-20000.pgf}}}
    \subfloat[All (106200) triplets]{%
      \resizebox{0.5\textwidth}{!}{\input{figures/results/majority-cut-8-n_triplets-106200.pgf}}}
    \caption{Depiction of how majority cuts look like on a simple mixture of gaussians, with a varying number of triplets available. 
        We plot just one bipartition $P_a$ with radius $r=0.5$, which was generated according to the procedure in \autoref{theory:majority_cuts}. The big X marks the anchor point $a$. 
        The bipartition $P_a$ consists of the orange triangles, which are the points that are twice as often closer to $a$ than they are not (according to the drawn triplets). 
        The blue points are those not in $P_a$.}
    \label{fig:majority_cut}
\end{figure}
