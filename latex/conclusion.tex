\chapter{Conclusion}\label{conclusion}
In this work, we demonstrated a suitable extension for the tangles algorithm to apply it to clustering and hierarchy reconstruction of triplets. This was done via
two different methods of generating cuts, \textit{majority cuts} and \textit{landmark cuts}.
We validated their properties on simulated and experimental triplet data.  
As we compared different algorithms in our experiments, we can also give recommendations as to when to use which clustering algorithm and which conditions on the data
have to be met for tangles to be effective. 

First off, if the goal is explainability, tangles can be a great choice. 
The importance of explainable algorithms is likely to increase, especially with 
a \textit{right to explanation} shifting more into the focus of policymakers and legal
scholars \citep{selbstMeaningfulInformationRight2017}, thus tangles could provide a substantial benefit over other algorithms for clustering triplets.
To our knowledge, there is no other clustering algorithm that both works with triplet data directly and is explainable. 
An alternative could be using an explainable algorithm such as \textit{explainable k-Means} \citep{moshkovitzExplainableKMeansKMedians2020} on the embedding created 
by an ordinal embedding directly, but this was out of the scope of this work. 

For standard clustering, we have to differ between the format of the triplets: sampled in a landmark fashion or uniformly at random.
For triplets in a landmark fashion, we observed a very good overall performance of landmark tangles.
If data is already present in this format, we can recommend
tangles as a clustering method, as it often shows better performance than the state-of-the-art algorithm SOE. This is especially true in the low-triplet regime. 
An experimenter that designs a new experiment, might think about whether it is feasible to sample data in a landmark format to utilize tangles, 
especially if its other properties (explainability and hierarchy reconstruction) are desired. 
If the data is expected to be very noisy, however, SOE might still outperform tangles.  
For triplets that are sampled uniformly at random, we only recommend using tangles (majority tangles in this case), if explainability and/or reconstructing hierarchies are desired, as SOE 
outperforms majority tangles in all our simulated experiments.

In the case of hierarchy reconstruction, we can give a clear recommendation of tangles. We have seen that landmark tangles perform best out of our evaluated algorithms. Majority tangles 
showed the second-best performance, roughly on par with SOE combined with average linkage. 
It can be problematic that tangles cannot construct a true dendrogram, which is what the hierarchical clustering literature focuses on (such as in \cite{ghoshdastidarFoundationsComparisonBasedHierarchical2019}).  For practical applications, however, the output from tangles can be good enough. 

In the setting of psychophysics, we have received good practical results with tangles on majority cuts,
confirming and expanding on the insights that the original author had gained through an ordinal embedding. 
Although, we think that tangles could perform better under different circumstances.
As the analyzed data set consisted of uniformly sampled triplets, we had to use majority tangles in our analysis.
However, in our simulations, landmark tangles performed much better than majority tangles on all data sets.
Also, as discussed in \autoref{real-explain}, landmark tangles can provide explanations that are more intuitive than majority tangles, as landmark cuts are inherently more explainable. 

Through our work, we also uncovered some problems and potential new research directions for clustering triplet data with tangles. 
The tangles framework by \cite{klepperClusteringTanglesAlgorithmic2021} is very flexible and has multiple areas where it can be expanded on or modified. 
As we demonstrated, changing how to preprocess triplets to cuts is an effective method of applying tangles to triplets. 
Further work could explore other ways of doing this preprocessing step, which is still lacking for non-landmark triplets. 
A part of the tangles algorithm that we have not touched on much is the cost function. 
One could imagine different cost functions than the mean manhattan cost function that we used. An idea could be to estimate a similarity on the data points using the triplets, for example, done in 
\cite{kleindessnerKernelFunctionsBased2017}, or use the average-linkage cluster similarity in \cite{ghoshdastidarFoundationsComparisonBasedHierarchical2019} and use this to calculate the cost 
of a cut.
Additionally, one could think about modifying parts of the tangles framework themselves, which could help with the problem of 
hierarchy noise we encountered in \autoref{sec:adding-hierarchy-noise}.

An addition to our research would be applying tangles to a more favorable real-world data set. This means that the data set contains both:
\begin{itemize}
    \item objects that exhibit a cluster structure, for example, distinct classes, such as the analyzed data set from Schönmann
    \item triplets sampled in a landmark format
\end{itemize}
To our knowledge, such a data set did not exist when we conducted our evaluations. However, this data could be readily generated in a controlled environment, 
for example, using the approach from \cite{inesschonmannSimilarityJudgementsNatural2021}.
