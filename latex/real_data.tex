\chapter{Case study: data from psychophysics}\label{real}
In this chapter, we want to use the tangles algorithm for real world data. 
The chapter aims to not only show the viability of tangles as a data evaluation tool, but also to serve as an 
instructive application for how a researcher might use tangles for their own experiments. We will use the data
from the thesis of \cite{inesschonmannSimilarityJudgementsNatural2021}
%TODO: Cite)
and will complement some of her analysis using tangles. This could be used to gain some more insight about the
data and their relations.

\section{Data background}
The data we will use was collected by Schönmann in her bachelor thesis. Schönmann originally
constructed the dataset to investigate how the formulation of a triplet question 
the perception of a person. It consists of the triplet data obtained from multiple
participants under different triplet questions and image sets.
To collect the data, the observers were presented with three images randomly drawn from the a specific set of images, and together with one out of four different questions. 
The \textit{odd} question, which we analyse in our setting is phrased: \textit{Which is the odd one out?}.  
As explained above, if out of $a,b,c$, $a$ is the odd one out, we transform this into two triplets $(b, c, a)$ and $(c, b, a)$.
Out of the three image sets, we will study the 
so-called thematic image set more closely. It consists of 5 classes which can be roughly divided into two themes: barn (straw, hay, pitchforks) and kitchen (forks, dishwashers).

For each observer, image set and question combination, the data set consists of $462$ unique triplets. 
Observer 2 has repeated the experiment again over a month later, but we have decided not to include those triplets to stay faithful to the original evaluation.

\section{Applying Tangles}
In the following, we will make an example of how to apply tangles to the obtained triplet data.
As there is a lot of data present, it is not feasible to repeat our evaluations for all possible data points.
We will choose one particular example to step through rigorously, and then we will quickly compare it to one other example. For our evaluations, we will use data points with the odd-one out triplets, as these have been reported in the work by Schönmann as 
having a higher embedding accuracy. We will also use the thematic image set, as Schönmann has reported embeddings that can be cleanly 
separated into different categories.

We will start out with observer 2. We have first reproduced the results by Schönmann by embedding the data with SOE into two dimensions 
(see \autoref{fig:ines-embedding-a}). One can linearly separate two sets of clusters, which correspond to a divide between kitchen objects 
(dish-washer, fork) and barn objects (hay, straw, pitch fork).  We have then applied majority tangles with an agreement of 3 and a radius of 1 (a lower radius produced only or two clusters) to the triplets and obtained a clustering. These cluster labels 
have been visualised in \autoref{fig:ines-embedding-b}) using the previously obtained embedding. Care must be taken: the embedding
from SOE is not a ground-truth embedding and just serves as a visualisation aid. If two clusters in the tangles-visualisation are far 
apart, this does not mean that the clustering is wrong – it could just as well be that the embedding is wrong or simply cannot capture
the cluster structure appropriately. This is similar to how T-SNE can produce very deceptive embeddings of data.

\onecolumn
\begin{figure}[ht]
    \centering
    \subfloat[Original classes]{%
    \resizebox{0.5\textwidth}{!}{\input{figures/results/ines_vp2_odd_thematic_classes.pgf}}
    \label{fig:ines-embedding-a}
    }
    \subfloat[Tangles clusters]{%
    \resizebox{0.5\textwidth}{!}{\input{figures/results/ines_vp2_odd_thematic_tangles.pgf}}
    \label{fig:ines-embedding-b}
    }
    \caption{
        We have embedded the obtained odd-one-out triplet data from observer 2 on the thematic image set.
        On the left, we see the original classes, on the right we see the predictions that we received by
        applying majority tangles with an agreement of 3 and a radius of 1.
    }
    \label{fig:ines-embedding}
\end{figure}


We can directly see that the clustering from tangles also produces a similar divide between kitchen (orange triangles) and barn (blue circles). 
However, we see a third cluster structure (grey squares), which is a mix between two pitch forks and a kitchen fork. 
When we look at these items (depicted in \autoref{fig:thematic-images-forks-a}), we can notice that they look more dissimilar to their
counterparts in the kitchen or barn cluster. The fork (left in \autoref{fig:thematic-images-forks-a}) has a design that is more reminiscent of pitch forks, and the two pitch forks
look more clean than the other pitchforks in \autoref{fig:thematic-images-forks-c} (no dirt on them, not depicted lying in grass). Thus, it can make sense that these three items are judged
as more similar to each than their counterparts in the kitchen and barn cluster and thus get put into a separate clustering. Interestingly, that is an insight that could not be reached from the 
SOE embedding alone and might provide valuable information for a researcher.

\onecolumn
\begin{figure}[ht]
    \centering
    \subfloat[Third cluster of different forks (gray squares)]{%
        \label{fig:thematic-images-forks-a}
        \resizebox{0.333\textwidth}{!}{\includegraphics{figures/thematic_stimuli/fork_03s.jpg}}
        \resizebox{0.333\textwidth}{!}{\includegraphics{figures/thematic_stimuli/pitchfork_09s.jpg}}
        \resizebox{0.333\textwidth}{!}{\includegraphics{figures/thematic_stimuli/pitchfork_16s.jpg}}
    }
    \hfill
    \subfloat[Other kitchen forks]{%
        \resizebox{0.2\textwidth}{!}{\includegraphics{figures/thematic_stimuli/fork_05s.jpg}}
        \resizebox{0.2\textwidth}{!}{\includegraphics{figures/thematic_stimuli/fork_06s.jpg}}
        \resizebox{0.2\textwidth}{!}{\includegraphics{figures/thematic_stimuli/fork_many.jpg}}
        \resizebox{0.2\textwidth}{!}{\includegraphics{figures/thematic_stimuli/fork_plastic_light.jpg}}
        \resizebox{0.2\textwidth}{!}{\includegraphics{figures/thematic_stimuli/fork_wood.jpg}}
    }
    \hfill
    \subfloat[Other pitch forks]{%
        \label{fig:thematic-images-forks-c}
        \resizebox{0.25\textwidth}{!}{\includegraphics{figures/thematic_stimuli/pitchfork_02s.jpg}}
        \resizebox{0.25\textwidth}{!}{\includegraphics{figures/thematic_stimuli/pitchfork_03s.jpg}}
        \resizebox{0.25\textwidth}{!}{\includegraphics{figures/thematic_stimuli/pitchfork_04s.jpg}}
        \resizebox{0.25\textwidth}{!}{\includegraphics{figures/thematic_stimuli/pitchfork_15s.jpg}}
    }
    \caption{
        The images of all forks that are present in the thematic data set. In a), we have put the pitch forks and forks that landed in
        the gray squares cluster in the tangles clustering (see \autoref{fig:ines-embedding}), which contained a mixture of kitchen and barn items.
        In b), we have depicted all other kitchen forks (all contained in the orange triangle cluster associated with kitchen items) and in c) we have depicted all other pitch forks 
        (contained in the blue circle cluster associated with barn items).
    }
    \label{fig:thematic-images-forks}
\end{figure}

Next, we want to see how one might use the hierarchy reconstruction and the explanatory power that tangles provides. For this, we first plot the hierarchy that we receive from the tangles
algorithm in \autoref{fig:tangles-nodes-tree}. To get a better idea of how the hierarchies look like, we plot the soft clustering at each node, which
can be seen in \autoref{fig:soft-clustering}. Here, each cut corresponds to one node of the hierachy and we can make out, which
clusters belong to which nodes. As expected, we first see a coarse, thematic divide between kitchen and barn (nodes 18T, 0F, see \autoref{fig:soft-clustering-a}, with 
the clean-looking pitchforks being placed together with the kitchen cluster. 
We then see a finer clustering of the kitchen cluster (node 0F), which is split into a set of various kitchen items (node 19F), and the cluster of special forks (node 4F). 
The fact that the special forks belong to the coarse kitchen cluster and only get separated off in a later step could be interpreted as the participant thinking of the 
clean-looking pitch forks as thematically belonging more in a kitchen than a barn. This is an insight that we couldn't have gotten from the ordinal embedding alone,
highlighting a possible strength of tangles.


\onecolumn
\begin{figure}[ht]
    \centering
    %TODO: Make this plot readable
    \resizebox{0.7\textwidth}{!}{\includegraphics{figures/results/tangle-nodes-tree.png}}
    \caption{
        Here, we have plotted the hierarchy we receive from the tangles algorithm. The hierarchy starts at the root and each other node represents a splitting cut, which
        separates the data further.  The label of the node is the number of the cut that caused the split to happen, together with the orientation (T means we have the cut in
        it's original arrangement, F means we have inverted the cut). This label can be used to identify the node in later processing steps.
    }
    \label{fig:tangles-nodes-tree}
\end{figure}

% contains the color scale

%TODO: Increase bar font, remove x,y font?
\onecolumn
\begin{figure}[ht]
    \centering
    \subfloat[Coarse cluster (barn - kitchen)] {
        \label{fig:soft-clustering-a}
        \resizebox{0.5\textwidth}{!}{\import{figures/results}{node_nb_1.pgf}}
        \resizebox{0.5\textwidth}{!}{\import{figures/results}{node_nb_2.pgf}}
    }
    \hfill
    \subfloat[Fine cluster (kitchen items - special forks)]{
        \label{fig:soft-clustering-b}
        \resizebox{0.5\textwidth}{!}{\import{figures/results}{node_nb_5.pgf}}
        \resizebox{0.5\textwidth}{!}{\import{figures/results}{node_nb_6.pgf}}
    }
    \caption{
        We have plotted the soft cuts that correspond to the nodes in \autoref{fig:tangles-nodes-tree}. We plot the SOE-embedding that we
        have calculated on the triplet data as a visualisation aid. Each cut corresponds to a node and thus to a cluster
        of a group of clusters. The color of a point corresponds to the probability that
        the point belongs to the given cluster(s), the darker, the more probable. 
        In a), we see the coarse clustering that corresponds to the nodes directly below the root. On the left, we see the barn cluster, 
        on the right the kitchen cluster. In b), we see the clusters that node 1F (the kitchen cluster) is made of. On the left, we have
        most of the kitchen items, on the right, we have the cluster of ktichen forks that look a bit more like pitchforks together with the clean-looking
        pitch forks (see \autoref{fig:thematic-images-forks} and the corresponding discussion).
    }
    \label{fig:soft-clustering}
\end{figure}

The last step showcased the strength of the hierarchical clustering of tangles. Next, we want to show how we can use tangles to explain the clustering:
why does a particular image belong to the kitchen or barn cluster? For this, we can look at the characterizing cuts of the clusters. As a reminder, if we look at a certain splitting node, its characterizing cuts are those that are always oriented in the same direction in the left subtree and in the other direction in the right subtree. 

We have visualised the characterising cuts in \autoref{fig:characterising_cuts}, together with the images that induced the particular cuts. As our cuts are interpretable (a majority cut 
with anchor point $a$ contains points that are in some sense close to $a$), we can directly use this interpretation for our clustering. As we can see in \autoref{fig:characterising_cuts_a}, 
the items that are in the barn cluster have landed there because they are similar to the two straw/hay images that we have plotted in \autoref{fig:characterising_cuts_b}. This is similar to our intuition, straw and hay intuitively belong 
in a barn setting and definitely not in a kitchen, while the pitchforks might be a bit more ambiguous.

Our interpretation is that the items that have landed in the cluster of other kitchen items are there because they are similar to
the dishwashers we have plotted in \autoref{fig:characterising_cuts_d}. This also makes sense, as we would interpret the dishwashers to very clearly belong into a kitchen environment, while the forks might be more ambiguous.

Overall, there is some small caveat to our explanation: In a majority cut, we only have a satisfying explanation in one direction: We know that if a point $b$ is in the majority cut that has anchor point $a$, 
then $b$ is close to $a$. However, the reverse direction might be a bit unsatisfying: If a point $c$ is not in the majority cut, we know that it is not close to $a$. If we want to know how the cluster
of forks and pitchforks formed, saying that they are dissimilar to the dishwashers plotted in \autoref{fig:characterising_cuts_d} is not a really strong argument, we would rather prefer to know that they are maybe
similar to a certain other item. To remedy this, we might want to use more interpretable cuts. If we would have sampled the data in a landmark format, we would have been able say something like: $a$ is in a certain cluster
because it is closer to $b$ than to $c$.


\onecolumn
\begin{figure}[ht]
    \centering
    \subfloat[Characterising cuts coarse cluster (kitchen-barn)] {
        \label{fig:characterising_cuts_a}
        \resizebox{0.5\textwidth}{!}{\import{figures/results}{node_18T_characterizing_12.pgf}}
        \resizebox{0.5\textwidth}{!}{\import{figures/results}{node_18T_characterizing_23.pgf}}
    }
    \hfill
    \subfloat[Images of data points point inducing characterising cuts for coarse cluster] {
        \label{fig:characterising_cuts_b}
        \resizebox{0.25\textwidth}{!}{\includegraphics{figures/thematic_stimuli/straw_hay_01b.jpg}}
        \resizebox{0.25\textwidth}{!}{\includegraphics{figures/thematic_stimuli/hay_02s.jpg}}
    }
    \hfill
    \subfloat[Characterising cuts fine cluster (kitchen items - special forks)]{
        \label{fig:characterising_cuts_c}
        \resizebox{0.333\textwidth}{!}{\import{figures/results}{node_19F_characterizing_0.pgf}}
        \resizebox{0.333\textwidth}{!}{\import{figures/results}{node_19F_characterizing_1.pgf}}
        \resizebox{0.333\textwidth}{!}{\import{figures/results}{node_19F_characterizing_3.pgf}}
    }
    \hfill
    \subfloat[Images of data points inducing characterising cuts for fine cluster] {
        \label{fig:characterising_cuts_d}
        \resizebox{0.25\textwidth}{!}{\includegraphics{figures/thematic_stimuli/dishwasher_03s.jpg}}
        \resizebox{0.25\textwidth}{!}{\includegraphics{figures/thematic_stimuli/dishwasher_09s.jpg}}
        \resizebox{0.25\textwidth}{!}{\includegraphics{figures/thematic_stimuli/dishwasher_05s.jpg}}
    }
    \caption{
        We have plotted the characterising cuts of our splitting nodes. We draw only the orientation that corresponds to the left subtree, the orientation for the
        right one would just be all cuts reversed. This means, if a datapoint is often contained in the cuts that we depicted above, it is placed in the left subtree
        with high probability. In a), we plotted the characterising cuts for the root node (coarse split) and in c) have plotted
        the characterising cuts for the fine split (node 0F). As these cuts come from our original set of majority cuts, we have marked the data point that 
        induced the particular cut with a black X. Below the characterising cuts in b) and d), we have plotted the images of the points that induced them.
    }
    \label{fig:characterising_cuts}
\end{figure}

Next, to show that the last evaluation wasn't just a particular edge case, we are going to repeat some of the basic evaluations for another observer.
We have selected the odd triplets and the thematic image set for person 3 and expect to see a similar behaviour than the one for person 3. In \autoref{fig:ines-embedding-vp3} we have plotted the embedding
from Schönmann again together with the tangles clustering. This time, we see three clusters (hay-straw, pitchforks and dishwashers-forks). These clusters coincide nicely
with our classes, aside from one fork being clustered together with the pitchforks. We note that this is not the fork from \autoref{fig:thematic-images-forks-a} that was clustered together 
with the pitchforks, so this might be a missclassification. 
If we look at the hierarchy (not plotted for space reasons) this time, we see that the pitchforks first get split off, and then the straw-hay from the kitchen dishwashers-forks cluster.
This could again lead to interesting conclusions, which should probably be discussed by someone with more expert knowledge in the field, possibly using other evaluation data.

\onecolumn
\begin{figure}[ht]
    \centering
    \subfloat[Original classes]{%
    \resizebox{0.5\textwidth}{!}{\input{figures/results/ines_vp3_odd_thematic_tangles.pgf}}
    }
    \subfloat[Tangles clusters]{%
    \resizebox{0.5\textwidth}{!}{\input{figures/results/ines_vp3_odd_thematic_classes.pgf}}
    }
    \caption{
        Analog to \autoref{fig:ines-embedding}, on the left we plot the prediction of majority tangles with agreement 3 and radius 1 over 
        a 2-dimensional SOE-embedding. On the right, we plot the original classes.
    }
    \label{fig:ines-embedding-vp3}
\end{figure}


\section{Discussion}
In this chapter, we have shown that tangles can be used as an additional tool in evaluating 
triplet experiments, providing valuable insights. In particular, the more flexible clustering 
can show new dependencies between data points than the euclidean embedding by SOE, as we could 
make out a cluster of forks being perceived differently by observer 2. The hierarchy 
provided by tangles can help put these new dependencies into a better perspective, 
allowing us to gather that the pitch forks in the special fork cluster were perceived to 
belong more in a kitchen setting than in a barn setting.
The explanations by tangles were used to make out the more defining items of a cluster (straw/hay for
a barn, dishwashers for a kitchen). To our knowledge, hierarchical and explainable methods
have not been used in psychophysics to this date, allowing tangles to eventually fill 
a gap.

However, we think that not all of the potential of tangles could be showcased here. 
As shown in our simulations, L-Tangles performs much better than M-Tangles on all data sets.
Also, L-Tangles can provide explanations that are more intuitive than those from M-Tangles,
as the cuts are inherently more explainble. A very interesting addition to our research would 
be applying tangles to our \textit{ideal} real-world data set. This means that we have objects
that exhibit a cluster structure, and that we have triplets sampled in a landmark format.
To our knowledge, such a data set does not exist yet.

